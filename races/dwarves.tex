\index{Dwarves}
\label{dwarvishHomes}
\renewcommand\npcsymbol{\Dw}

\subsection{Settlements}

Far underground, below the soil or coiled up within mountains, the underwyrms roam.
Some are as long as a castle, while others stretch only the length of eight horses.
Their head is that of a streamlined lizard, and they snake limblessly through the bowls of the world, pushing or chewing up raw earth and stone.
They feed on a combination of minerals, rocks and underground fungi.
And in their path they leave wide, wide tunnels.

After underwyrms form tunnels, little dwarves follow on -- strengthening them with properly placed stone arrayed into an arch or packing the tunnel with clay and then setting a fire of mushrooms, underwyrm droppings and underground oil. Then they carve and chisel for decades until they have a hall or room fit to house a dwarf, or a deep fungal garden, powered by an underground lake or river.

Almost all dwarvish communities begin by underground lakes -- many are boating folk, though they do not understand the open sea, or its wind and tides. You know where you stand with a dwarvish lake -- you stand still. It is often at the centre of the lake that one finds the day-bell, a massive bell which forms the pride and heart of any dwarvish community. The day bell rings after 20 hours to say that work has finished and then again 8 hours later to say that work has started again. Many communities buck this trend one way or the other, depending upon the whims of their queen.

The outsides of a dwarvish citadel (or `undertown') are reinforced with metals and very dense clays to discourage outsiders digging in.
Dwarves know exactly what might collapse, how to reinforce walls, and pull them down in a hurry.

Alcohol forms a massive part of dwarvish culture.
They use it primarily for light or cooking, as it produces less smoke than other fuels.
Dwarves, they say, can ferment anything -- living oozes, fungi, goblins.
All life underground eventually converts to light.%
\footnote{Dwarvish saying: \textit{`If it lives, it can die. When it dies, it rots. If it rots, you can burn it'}.}

Commonly, dwarvish tunnels to the outside will end in a gnome-warren.
Direct contact with the outside world, opening into a forest or plain, is seen as `letting the sun in',
and generally frowned upon, but if the dwarvish tunnel ends in a gnomish warren and those gnomes happen to let the sun in, well -- that's \emph{their} business.
This persistent crossing of paths means that the dwarvish and gnomish languages have many common words, and patient speakers of one can mostly understand the other.

\subsubsection{The Language}
\index{Languages!Dwarves}
has an official form, unchanged since the beginning of time.
None of them speak it, and they disagree on every kind of pronunciation, but all dwarves write in the official language.
\index{Dwarves!Names}
To do otherwise would invite shame, as written errors persist longer than spoken errors, and most dwarven speech is an error, at least according to the `official' dwarvish stance.
Dwarves avoid the problems embedded in their own `regional' names by simply translating the meaning when travelling beyond the mountain.

\subsubsection{The Structure}
\label{dwarven_structure}
of their society is heavily matriarchal -- only around one in every ten dwarves is female, so most never marry.
Women stand at the heads of their society and are generally considered too precious to go above ground for the menial tasks of trading for food or cutting down wood.
Rich males compete in fashioning the most exquisite jewellery in order to win the hand of a fair, dwarvish maiden (or indeed, any dwarvish maiden).

\subsection{Commerce}

Underground trade focusses on farming mushrooms, glow-worms for lanterns, underground jellies which feed on water and slime; all manner of underground delicacies are created deep below the earth (though it seems only dwarves actually find them palatable).

Dwarves are famed for their exceptional armour, being the first to invent full plate armour, and still the best at creating it.
They can enter combat fearlessly, knowing that little except an underwyrm can penetrate their thick, steel plates.

\subsection{Warfare}

Dwarves use a lot of smoke when fighting; any enemy coming from a smokey tunnel will invariably suffocate before pushing through defences.

When defending a large entrance, dwarves set themselves up with crossbows, then hand the crossbows back.
Others behind them reload the crossbows in a production line, then hand it back.

While rudimentary crossbow-string might be made from watchers' tendrils, the best comes from hemp.
Dwarves can construct the rest of the item from wood or umberhulk chitin.

When narrower tunnels eventually demand toe-to-toe combat, dwarves always fight with spears or swords (which humans irritatingly refer to as `short swords').
They bring all the nastiest, burnable material they can to a battlefield, such as specially dried mushrooms, or wood, and lay it around the start of a narrow tunnel where they intend to fight.
They stab a little with their spears, then retreat while lighting the fires underneath them.

Dwarves often wet their beards before battle, to protect them from flames.

\needspace{12\baselineskip}
\subsection{Inheritance}

\subsubsection[Tenacity: dwarves take only half the usual penalties from rotten food, poisons, or foul air.]{Tenacity}
\label{dwarvenInheritance}
is learnt with every meal, as dwarves grow up eating the most acrid substances -- tough mushrooms and acidic jellies (well cooked, of course).
Dwarven ales are classified as spirits by any sane human and dwarven spirits are generally classified as poisons by all other races.
The same applies to bad air.

Dwarves take half Damage or \glspl{ep} from any given poison or gas.
They suffer no ill effects from eating rotten food (though it may not count as being nutritious) and the \gls{gm} is encouraged to allow them to eat anything that might otherwise be damaging, within reasonable limits.

\subsubsection{Taciturn}
dwarves trust others slowly, and like to remain formal when first meeting people.
In gaming terms, they cannot spend \glspl{storypoint} during their first session.

\subsubsection{Fire \Glsfmtplural{ingredient}}
\index[mana]{Fire!Dwarves}
can be made from dwarven beards.
Of course, dwarves never like to speak about this, and often dismiss this as a rumour, designed to make trouble between dwarves and \glspl{doula}.

\subsection{Enlistment}

Dwarves join the \gls{guard} for the same reasons as anyone else -- a criminal background, a propensity for violence, and hope of gold.
And each one in the \gls{guard} carries some plan for that gold, even if they never voice it.
Because at the end of the day, it's never about the gold -- it's about what you plan to do with it.

\subsection{Roleplaying Dwarves}
\index{Dwarves!Roleplaying}

Check then double-check.

\begin{itemize}
  \item
  Does this person really know where the lost temple lies?
  Ask him about the rooves, doors, and other items made of wood.
  If people abandoned the temple three centuries ago, those constructions must have degraded.
  Does his story match?
  \item
  Have you really made your point clear?
  Tell him again what will happen if he fails to pay your money back, but \emph{louder}.
  \item
  Does the beer taste good?
  A really good beer still tastes good when you drink three in a row.
  \item
  When the guide says he will lead you all to the lost city, does he mean `within visual range', or `up to the gate', or `to the actual monument, in the centre'?
  Is that written in the contract?
  \item
  Do we have enough \glspl{torch} for this mission?
  If the last crew took two hours to journey down, and three hours back up, and if each \gls{torch} burns for one hour, then you will need at least five \glspl{torch} for the journey, and one to look around for an hour.
  Best bring ten.

  Share the \glspl{torch} among your companions, so that if you lose one, the group still has enough \glspl{torch}.
  \item
  Has the bandit really died?
  Stab him in the neck, just to make sure.
\end{itemize}

\emph{Write yourself a reminder to double-check this section at the start and end of every session, to make sure you have put it into practice.}


