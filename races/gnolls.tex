\index{Gnolls}
\label{GnollishGrounds}

\renewcommand\npcsymbol{\Nl}

\noindent
Only the gnolls have the strength and wits to live above-ground, without walls.
They keep and breed fierce hunting dogs, so a group of twenty gnolls will often have around fifty.
These dogs keep watch, sometimes prowling around a camp's outskirts, sometimes simply keeping their ears up, so the camp never lacks sentries.

Smaller groups hunt.
Larger groups generally herd animals, and can be heard a long way off, due to the combined noise of aurochs, goats, sheep, and gossip.

People change from one clan to another depending upon romantic partners or where they find themselves.
The various mobile clans sometimes fight, but always come together when an outsider invades their territory.

\subsubsection{The Language}
\index{Languages!Gnolls}
\index{Languages!Dragons}
sounds distinct from any other, due to the shape of gnolls' jaws.
Other people, with other jaws, struggle to make heads or tails of the gnolls' languages.
Tails present a particular problem, as gnolls communicate with their entire body, like a super-sign language.
Spoken words tend to relate to far-away communication, such as shouting for aid, or invitations to dinner; signs signal loyalty, subtle descriptions, or veiled warnings.

The Gnollish language shares a great deal of vocabulary with the standard speech of dragons.
According to legend, the gnoll hero Kshonk taught the dragons how to speak so that he could outwit them.

\subsubsection{The Structure}
emerges through hyperactive gossip.
It never ends -- the chatter is constant, but when serious decisions arise about where the tribe should go, or whether it should fight, the gossip reaches incredible speed.
Every gnoll speaks at once, to all sitting beside them, in a short, hurried fashion, acknowledging and expanding upon others' points.
This process goes on for anywhere for twenty minutes to a full night and day.

By the end, they reach a consensus.
Nobody knows exactly how the process works, or what kind of governance to call it.
Outsiders only know that when gnolls start talking, nobody else can keep up.

\subsection{Commerce}

Gnolls primarily trade meat with dwarves and humans, who can never get enough of their own.
They also trade hunting dogs, but charge a high price, and always make sure that the buyer promises to look after the animal properly.
The buyer should keep this promise, as gnolls take note of how buyers care for their animals.
Humans sometimes complain they don't really understand `property', and the gnolls generally don't sell to those humans.

Gnolls sometimes take coinage, but prefer jewellery, as it can be worn, and does not require additional preparation, like money-sacks.

\subsection{Warfare}

Gnolls almost universally employ guerilla tactics.
They set settlements on fire, attack supply lines, and generally poke at every weakness which comes from living in a fixed location.

Massive castle walls daunt gnolls deeply, so they prefer not to attack large civilizations, but if they must do so then they always focus on attacking supply lines, while moving in small groups around the area, encircling it with tiny groups.

\subsection{Inheritance}

\subsubsection[Teeth: grab and grapple in a single manoeuvre]{Fangs}
\label{gnollishInheritance}
mean that Gnolls, like wolves, can grab and damage in a single attack by sinking their teeth into a target.
This deals $1D6 + Str$ Damage.

\subsubsection{Air \Glsfmtplural{ingredient}}
\index[mana]{Air!Gnolls}
can be created from a gnoll's intestines.
Gnoll \glspl{witch} typically extract intestines from the dead.
Others refer to this as a `death ritual', but in fact, gnolls simply value practicality, and rarely bother with rituals.

\subsection{Enlistment}

Gnolls have no `criminals' within their own society (every crime has a fast ultimatum, and possible redemption or death), but those corrupted by human society still end up in the \gls{guard} when breaking the law.

Other gnolls sign up voluntarily.
Being faster and fitter than humans, they stand a better chance than most at survival.

\subsection{Roleplaying Gnolls}
\index{Gnolls!Roleplaying}

Let's go!

\centering{Gnolls get things done, then move on.}

{\raggedleft What's next?\par}


