\label{gnomishWarrens}

\renewcommand\npcsymbol{\Gn}

\widePic[t]{Roch_Hercka/illusion_trogdor}

\index{Gnomes}
\noindent
Gnomes live in little warrens, under the ground, but enjoy lots of sunlit openings near the edge of their warren.
Their network of tunnels and homes extend often up to fifty feet below the ground.
These little communities often keep two-level farms -- they tunnel beneath what others consider to be good farmland and then pull cabbages, potatoes, carrots and other rooting vegetables down from the ceiling rather than up from the earth.
They consider humans to be backwards, since root vegetables clearly grown downwards, to emerge at the bottom when ripe.

Gnomes take great pride in remaining `subtle' -- the openings to their houses are never glass but openings which can be closed in order to look as natural as possible -- the side of a hill may open to reveal a living room, or a large, apparently dead tree may have a door opening underground to a small pantry.
Often, the only way to spot a gnomish warren once the doors are closed is to note the bountiful fields of good crops.
Most gnomish gardens cannot support `heavy things', such as a human on horseback.
This leads to humans falling through the soil of a gnomish garden and into a warren, where a number of gnomes have to wonder what to do with a wounded horse and a bemused human rider, and whether or not to keep their warren's location a secret.

All warrens have many traps to secure them from predators and bandits.
\Glspl{woodspy} which probe a tentacle underground often find a razor-sharp edge which contracts as the tentacle withdraws.
\Glspl{crawler} searching the grounds will find myriad entrances, all of which have blades pointing harmlessly inwards\ldots until the creature turns back, and finds the blades far more of a problem when trying to leave the narrow corridor.
If rude dwarves decide to arrive fully-armoured, for a rude visit, they may find a hallway festooned with tiny hooks, just strong enough to snag on their helmets and distract from the cracks in the ceiling.

Gnomes make two distinct types of traps: those built for animals (which anyone with a little sense can see) and those build for people (which nobody can see, unless they understand how gnomes think).
Their \glspl{talisman} work similarly -- gnomes often write some activation work on their alchemical creations in the form of a riddle; this ensures that stupid people who don't speak their language cannot use the item, which functions to stop `bad people' using the item.%
\footnote{We all have our little prejudices, and the gnomish intellect does not make them an exception.}

The gnomish language is rather similar to dwarvish but can change almost as quickly as human languages.
They have three versions -- in addition to being able to speak and write, they can also whistle their language.
The language has a strict way of making sound shifts form normal sounds to whistling sounds.
This allows gnomes to communicate over massive distances -- over wide plains, mountains or through a mile or two of underground tunnels.
It also allows them to hold conversations between each other while standing right in front of people, as most people do not understand that when a gnome is whistling they are also probably saying something meaningful.
Or meaningless.
Gnomes are big fans of using language for its own sake. 

Upon greeting each other, gnomes do not give their names but ask for one -- customarily each person a gnome meets will have one name for them, and a group name will soon emerge for each different social circle. This causes no end of confusion when people ask a gnome what their name is, and the gnome takes this as a sign of an unimaginative companion, before giving the new friend a name without asking what they would like to be called.

\subsubsection{The Language}
has very little vocabulary.
Many assume that such an intelligent little people would develop an extremely complex and precise language.
In fact, the opposite case holds -- the Gnomish language has fewer than 200 words, which then create compound words.

$crazy + water = alcohol$

$flight + animal = bird$

$small + flight + animal = mosquito$

When gnomes become bored of making elaborate compound words, they revert to just one word per concept (or fewer) and expect people to just `get it'.

Despite the difficulties creating compounds, people learn the basics fast, so Gnomish has formed the basis of the \gls{tradeTongue}, which almost everyone uses while travelling.

\subsubsection{The Structure}
varies greatly from warren to warren.
Many Gnomish societies have complicated electoral systems where members cast differing numbers of votes in order to elect to create various positions of government.
These positions are then voted upon with different voting systems, and a third is in place to decide how often votes will take place and how to vote on bringing in new voting systems.
This can take place with warrens with as few as ten gnomes, and often every member of the warren will be in government in some sense or another.
Any time a decision is called upon, gnomes will be delighted to help, and will often return a month later with an ornately carved flowchart of exactly how to determine `Step A' in the `decision-optimization adventure'.
And if nearby dwarves and elves ignore this advice, it's just further evidence that the other races are both impatient and a little stupid.

\subsection{Warfare}

When gnomes can flee, they do so, but otherwise nobody knows what they might do ahead of time.
They dislike repeated tactics or methods.
They prefer unpredictable plans to reliable ones, and often rely on details that people think of as inconsequential, such as what the enemy's shoe-laces are made from, or what the maximum tunnel-size the enemy can comfortably run through.
They might stop and draw a perfect square into the dirt with their finger before running away, leaving any half-sensible enemy to conclude that some hidden trap lies on the ground.
And if they know illusion magic, they always mix together illusory common beasts for the area, illusions of spells they could theoretically cast (and in fact can cast), and ensure all of these illusion have far less detail than they could on the first casting.

\subsubsection{Gnomish traps}
\index{Gnomes!Traps}
\index{Traps}
are known across all of \gls{fenestra}, and demand a high price due to their rarity.
And every trap worth using comes with a Gnomish trademark, as a guarantee of its excellent quality.
Those who know the marks have been known to back away from an underground entrance in terror, rather than trying to navigate the warped cattle grids which might lurk inside.
Despite high demand, the traps usually weigh too much for small people to carry, and asking others to carry them would mean showing the big folk where the warren is.

\paragraph{Cattle grids}
are narrow bars, which humans can walk across but cows struggle with.
The same principle applies to gnomes and \glspl{crawler}, \textit{mutatis mutandis}.
Changes include sharpened blades, facing \emph{into} a tunnel, a pit with a single, narrow, log serving as a bridge, or just a long ladder.

\paragraph{Poison kites}
are kites with small nuggets of poisoned food.
If \pgls{griffin} sees the kite, it usually swoops down and eats it, but may alternatively attack the gnome.
As a result, only specialists fly kites, and gnomes consider the activity an extreme sport.

\paragraph{Water mazes}
are artificial ponds, with rusting spikes covering the bed.
A stone walkway, just inches under the surface, allows anyone who knows the path to enter freely.

\Glspl{basilisk} regularly walk away from them limping, \glspl{crawler} usually avoid bodies of water, and \glspl{digger} can't swim.
Unfortunately the mazes need someone to repeatedly defile the water, keeping it putrid and vile, or \pgls{woodspy} may move in.

\paragraph{Watermill spikes}
usually require two or three watermills chained together.
Once you get enough torque, the wheels can rotate a crank with a metal spike facing \emph{upwards}, through a hole in the ceiling.

Setting up a watermill spike takes a lot of effort for a whole community, but the result is a tunnel with an automated spear-stab-trap, rising from the ground to impale intruders.
Some of them also grind wheat.

\paragraph{\Gls{woodspy} chests}
have two or three latches which lock the box the moment \pgls{woodspy} enters.
Small pieces of wood keep the chest open until pushed aside.
The chests don't have any bait -- no meat or wriggling rodent.
Instead, the chest simply sits near a path or door, as if to say `gee, I sure hope nobody goes into my treasure chest and ambushes me later'.

These chests usually have a small opening at the back next to a bell.
The opening leaves just enough room for a flustered \gls{woodspy}'s  tentacle to squeeze the tip out, and feel around for any way out.
At this point, the tentacle-tip rings the bell, which lets the gnomes know they've caught something.

Trappers must empty the trap quickly, because \glspl{woodspy} are semi-intelligent, and complete psychopaths.
If they see another \gls{woodspy} caught in a chest, they will not only understand the trap, but wait beside it, in the hopes of catching a gnome.

\subsection{Commerce}

Gnomes use treasure maps as currency, as none of them want to walk long distances with bags of heavy coins.
Humans who try to use these maps to hunt down the treasure are always disappointed, as they lift the treasure-chest's lid, and find it's full of maps.
This is because, as covered earlier, gnomes use treasure maps as currency.

Some say that the maps comprise a grand treasure-hunt, which stretches across all of \gls{fenestra}.
Others say that someone has probably found every real treasure which once existed.
It makes no difference, because as far as gnomes are concerned, treasure maps represent value.

\needspace{12\baselineskip}
\subsection{Inheritance}

\subsubsection[Attentiveness: roll 2D6+3 for resting actions]{Attentiveness}
\label{gnomishInheritance}
means Gnomes often have a hard time focussing on things, but once they successfully do so, they focus to the exclusion of all else, often with amazing results.
When gnomes take a Resting Action, rather than rolling $1D6$ and adding +6, they roll $2D6+3$.
If they want to change a failed action into a Resting Action, they add $1D6-3$ to their roll.

\subsubsection{Earth \Glsfmtplural{ingredient}}
\index[mana]{Earth!Gnomes}
can be harvested from Gnome-bones with enough grinding.
Many theorize that this explains why \glspl{griffin} seem so eager to eat them.%
\footnote{Gnomes have their own theory, but it makes for a bad story, so nobody can remember their claims.}

\subsection{Enlistment}

Many join the \gls{guard} to find rare \glspl{ingredient} for \glspl{talisman}.
Others join the \gls{paperGuild}, then steal some valuable book when they think nobody can see them, and end up in the \gls{guard}.

\subsection{Roleplaying Gnomes}
\index{Gnomes!Roleplaying}

{\raggedleft Think sideways.\par}
\noindent
If Human things are `Human', and Dwarven things are `Dwarven', is my hat `Gnome' or `Gnomen'?
Can we apologize to the \gls{witch} and make amends instead of killing her?
Can you use a hammer to communicate?
What else do shoes do?

Gnomes see the world from a different perspective.
They look up people`s noses all day.
Gnomes see the ceiling while others look down at the ground.

Gnomes travel slowly but it looks like a large space to them.
From a relative perspective, a travelling Gnome has travelled farther than the rest of the troupe.
Are we counting footsteps or miles?
Did you know that every mile has 5.280 feet?

Where did the \gls{witch} commission her traps?
Is the architect still alive?
Does he have standard schematics for his traps in a workshop where he builds traps for people?

What kind of contract do you make when you sell someone a trap to guard their labyrinth?
What happens if I roll a boulder down the stairs?
Have these traps killed before?
Where do the bodies go?
Does someone climb down to get them out and do they use a ladder?
If we dig out the stream nearby, we could flood the labyrinth.


