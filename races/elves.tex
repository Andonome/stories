\index{Elves}
\label{elvenGlades}

\widePic{Studio_DA/elf_stalker}

\renewcommand\npcsymbol{\El}

\noindent
Elven settlements depend on what kinds of \gls{witchcraft} their eldest enjoy most.
Their \glspl{spell} dominate and define the settlement.

If an elder knows the Force \gls{sphere} well, they can create magical gateways connecting a cavern to a treetop, which connects to a valley through a briar.
Visitors who don't know about the latest changes to the local space become hopelessly lost quickly, as they pass under a tree and find a nearby mountain has suddenly shifted from their left to their right.
They can communicate throughout the land by making every fire take the shape of little apparitions, who tell a tale through interpretive dance, and then indicate `bed-time' by taking a bow and going `snuff'.

Elders who focus on Life \glspl{spell} rearrange the local wildlife to better serve the younger elves.
They make birds grow massive so they can lay massive eggs.
Fish grow a single, enormous tooth with a razor-sharp edge, so anyone catching one can pull a pre-made knife from its skull, then use the skin as durable boots with minimal preparation.
And when trees shed their bark, it falls into perfect strips of paper.
And since Life comes from mixing Earth and Water \glspl{spell}, these elders can turn snow into ice palaces, or craft underground houses with glass rooves, lit by fireflies.

Younger elves who grow under the care of the Mind \gls{sphere} have a much harder time.
Mind \glspl{spell} let the elves communicate through messenger-crows (who relay messages verbatim, to the requested location), and ride obedient \gls{griffin}-mounts.
But anyone who spends their centuries building up and tearing down others' minds cannot retain any true respect for others; Mind \glspl{spell} always result in a manufactured and false consent.
These lands remain free from crime in the worst possible way.

Younger elves leave their settlements so they can learn self-reliance, beyond the false paradise that raised them, and to practice some \glspl{spell} without an elder using all the mana first.
Older elves leave because enough \gls{witchcraft} changes anyone, and little by little, elves stop being what they were, and become something else.

This natural ageing process is not some kind of decay, just raw change.
Some elves enjoy twisting their own bodies, and reshaping minds, until they find themselves breathing the water at the bottom of a swamp, with tentacles instead of legs, talking to local sparrows, and occasionally eating a passing human.
After all, humans are just another short-lived animal.

While female elves often become \glspl{dryad}, most male shape-shifter end up as trees.
If you consider `shape-shift until rebirth' to be the real adulthood, then elves are highly sexually dimorphic, despite what people say.

\subsubsection{The Language}
\index{Languages!Elves}
does not change much, as elven elders maintain old forms of speech for as long as they live.
The similarities between the various Elven languages suggests they come from a single, united source -- a `proto-Elvish'.
But in fact elves just swap songs so often that common elements become inevitable.

\subsubsection{The Structure}
follows expertise, which often follows age.
In matters concerning hunting, the master hunter will make all group decisions.
In matters concerning statues, the master carver will make communal decisions.
Each expert has their own strict domain of influence.
Many elves translate these `masters' as `king' or `land warden' when speaking with human, and as a result nearly half the elves abroad in human lands claim to be the children of royalty -- exactly how accurate this is depends upon one's interpretation.

\subsection{Commerce}

People think that elves won't trade with anyone due to snobbery.
But in truth, the elves rarely trade with anyone, because nobody has the time.

\subsubsection{Jewellery}
shows how wealthy an elf is, so more wealth means more piercings.
Typically these will be in the ears, but torso piercings are also common.
Rings, necklaces, brooches and all manner of other precious art pieces adorn most elves with any interest in commerce.

The value of jewellery depends on its history -- having a famous maker increases the value, as does a history of being worn during a famous battle, or just by a famous elf.
While trading, elves will explain the complete history of each item, in order to ascertain its worth.
During this time, the seller expects the buyer to sit silently and listen, without reaction.
This custom arose so that buyers who already know about the history of an item can corroborate what the seller tells them, and then inform other elves if the seller's information matches, or does not match.
This chain of listening, comparing, and noting others' reputations keeps the system consistent.%
\footnote{Consistency is often more important than truth.}

While most traders don't have a day or two to simply look at people's wares, some of them manage to break the system.
Gnomes often set up deals through a series of letters (which they then show to others, rather than repeating a long conversation), and some gnolls manage to conclude deals simply by {\small speaking-incredibly-fast} and hurrying the seller along.

\subsubsection{Songs}
form a kind of second currency, as jewellery is rarely traded for music, and vice-versa.
Instead, elves trade a songs for songs.

Elves often store information in songs, including area-knowledge, gossip,%
\footnote{What human refer to as `history', elves refer to as `gossip'.}
recipes, hunting techniques and spells.%
\exRef{core}{Core Rules}{ritualCaster}
They do this partly to make something beautiful, but mostly to solidify their idea, and ensure nobody changes what they have made.
So while elves can change a recipe themselves, they can't pass those changes on to someone else unless they can fit their changes to a new rhyme.

\subsection{Warfare}

As a rule, elves have a wickedly individual mindset, which makes wars difficult.
They understand group-thinking in the abstract, but lack the instinct.

For a band of elves to instigate a battle, each one must have an individual reason why this fight is worth risking their long life.
Of course, battle still occurs, but the groups tend to be small, and usually rely on long-range spells.

Even when defending their own turf from an encroaching enemy, elves usually think about how they -- as an individual -- can deal with the situation before asking others.
They can still work together in defence, and take orders from a central point of command, but the order of their instincts matters a great deal, and often results in elves fleeing when they would likely be victorious.

Despite these limitations, elven \glspl{witch}, with their centuries of practice, ensure the safety of most settlements.
With sufficient motivation, superannuated casters can destroy entire settlements.

\needspace{12\baselineskip}
\subsection{Inheritance}

\subsubsection[Thermal Apathy: take no penalties from natural weather.]{Thermal Apathy}
\label{elvenInheritance}
means elves are immune to \glspl{ep} from natural heat levels -- they can sleep outside in the snow or wander deserts without sunburn.

Within their own land they might wander naked, or put on clothes just for the joy of adornment.
However, when visiting abroad, they always put on something, as a minimal effort to acclimate to others' cultures.%
\footnote{Nobody ever thanks an elf for all the effort they put in to purchase and maintain clothing, which the elves take as yet another sign that outsiders are all barbarians.}

\subsubsection{Longevity}
means the elves do not degrade, but they do change over the centuries, becoming progressively more fay-looking and alien.
Their minds sharpen, but their bodies degrade.
After 100 years, an elf's maximum Strength Bonus decreases from +2 to +1 but their maximum Dexterity increases to +4.
At 200 years old the elf's maximum Strength score becomes 0 but their maximum Speed Bonus raises to +4.
At 300 the elf's maximum Strength Bonus is -1 but they can move their Intelligence up to +4.
Finally, at 400 years old the elf's Charisma Bonus becomes +4 and their maximum Strength becomes -2.

  \begin{boxtable}[XcX]

    Age & Max. Strength & Increase \\\hline

    100 & +1 & Dexterity \\

    200 & 0 & Speed \\

    300 & -1 & Intelligence \\

    400 & -2 & Charisma \\

  \end{boxtable}

\subsubsection{Water \Glsfmtplural{ingredient}}
\index[mana]{Water!Elves}
can be harvested from elven tears.
Unfortunately, shedding tears also drains elves of all \glspl{mp}, so elves quickly learn to withdraw from unpleasant feelings.

\Glspl{pc} can elect to cry on command by spending \pgls{storypoint}, in order to call upon a tragic memory.

\subsection{Starting Characters}
Elven characters begin young, without the experience, keen intellect and amazing skill-set of their elders.
Some join the \gls{guard} in order to gain the experience they see in their elders.
Others want to learn a specific skill, perhaps to master the rapier or an elemental Sphere.
Most just want to see what the world has to offer.

Elves tend to view their own young as expendable.
They do not reproduce rapidly, but over long centuries a single elf can easily have many children.
Since the youth tend to be stronger than their elders, these young things are encouraged to perform the most dangerous of tasks such as hunting large animals or defending a village through m\^{e}l\'{e}e rather than with a bow.
As a result of this attitude, elves encourage their young to go out into the world and seek knowledge before they become old, delicate and strange.

\index{Elves!Starting \glsfmtlongpl{xp}}
When starting \pgls{campaign}, the \gls{gm} may allow elves to spend all of the \glspl{xp} they would gain from \glspl{storypoint} when creating the character (this usually totals 25~\glspl{xp}).
The character can spend \glspl{storypoint} as usual, but gains no further \glspl{xp} for using them.
Spending all \glspl{xp} up-front allow elven characters to begin with evidence of their lived-experience, then leaves the character broadly unchanged, while the other \glspl{pc} show growth-spurts as they spend \glspl{storypoint}.

\subsection{Roleplaying Elves}
\index{Elves!Roleplaying}
\label{elfRoleplaying}

Elves have no words for `good', `bad', or `evil'.
The language they speak makes no difference -- the absence is cultural.
As a result, elves do not fully understand or use these words, even when speaking other languages.

Bread cannot `go bad' -- it has mould.
They will never call a song `good' -- the song feels lively, or sounds like a Sunrise, or makes one think of home.
They would never call someone `evil' -- they might say `destructive' or `useless', or `selfish', but never use language which characterizes anything with such a wide notion as `good' or `bad'.

If someone says `your plan sounds good', make sure to clarify if they mean that they want the results of the plan, or if the plan seems likely to succeed, or if the plan has been stated clearly.
And when you hear something is `bad', clarify that too.


