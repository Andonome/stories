\index{Humans}
\label{humanTowns}

\renewcommand\npcsymbol{\Hu}

\noindent
Humans are a massive, war-like race with round ears, who all live above ground, despite the dangers.
Their strong arms let them wield long, iron weapons for battle, and build high wooden walls and dig deep ditches around their settlements.

Their large size and tall walls don't protect all of the humans from being eaten by large creatures of the forest, so populations must survive by having many children.
Humans reproduce at an alarming rate -- instead of simply replacing themselves with two or three more humans, a couple might make as many as fourteen.
A good many will practice the hunting bow daily to protect their animals from forest predators.

The repetitive activities of their \glspl{village} hide a subtle chaos.
No matter how long one lives with them, each one does the same thing, at the same time of day, every day.
But when people return a dozen years later, they find half the humans have a different routine, and the other half have died.

When enough humans form together, a city often springs up in the centre, organically.
None of them organize where to build the city, it simply emerges around the middle of anywhere they can defend.
Once a city establishes itself in the area, humans will start to raise specialists who can practice at woodworking, book-binding, and other specialized skills.

\widePic[t]{Leonard/next_day}

\subsubsection{The Language}
\index{Languages!Humans}
looks very unified in speech, but with different styles of writing.
This illusion comes from the fact that every time humans travel, they pick up a few local words, and copy a little of the local accent.
By the time someone has travelled a thousand miles, they have arrived at a very different language, without learning anything entirely new at any step.

Exactly what counts as a new language depends a little on shared words, and a lot on the speed of travel.
Simply put:

$$ comprehension = \frac{cromulance}{velocity} $$

The local human languages share enough words with Gnomish that the two are mutually intelligible, as long as both speakers have some patience.
However, when humans speak their local dialects quickly, nobody can understand them except their relatives and others who come from a nearby \gls{village}, or from the same town.
Many use their dialects as a secret language, or `cant', when they want to speak privately.

\subsubsection{The Structure}
of any \gls{village} or town is extremely important, because humans love hierarchies and become confused about what they are doing if they cannot identify a nearby leader.
As a result, specialized decision-makers arise, usually inside cities, called `\glspl{warden}', who dictate what happens in a city, and distribute justice to criminals in a large court-house.

\subsection{Commerce}

Humans' massive feet and their habit of following each other creates massive roads.
Additionally, they trade live animals more often than hunted game, which creates more roads as cows, sheep, and goats trample down every possible route between human settlements.

They cannot weave quality spells, or make long-lasting armour, but the sheer quantity of goods they have to trade always lets them purchase these goods from others.

\subsection{Warfare}
Humans always rely on numbers in battle.
Coupled with their incredible size, they make a formidable force without much need for additional tactics.

Due to their slow minds, humans need to use simplified signals for battles, such as trumpets or flags, which can signal where everyone should go.

\needspace{12\baselineskip}
\subsection{Inheritance}

\subsubsection[Marching Legs: every \glsentrytext{ep} spent to march adds 2 miles]{Marching Legs}
\label{humanInheritance}
mean humans can walk a long way before they feel tired.
Instead of taking \pgls{ep} to cover an extra mile, they can take \pgls{ep} to cover an additional 2~miles.

Humans may seem slow and clumsy, and may not run terribly well.
But when time is measured in days, they are the fastest in \gls{fenestra}.

\subsubsection{Fate \Glsfmtplural{ingredient}}
\index[mana]{Fate!Humans}
can be made from human blood, if distilled into an ink.
Sometimes it makes for two \glspl{ingredient}, if the human had enough blood.

This process takes a long time, but the results speak for themselves, so many spell-casters will refuse to enter battle until they have enough humans write some bloody \glspl{boon}.

\subsection{Enlistment}

The \gls{guard} exists to ferry all the excess humans into the forest, to protect those who can plant and create.
They never need much of a reason to join the guard.

\subsection{Roleplaying Humans}
\index{Humans!Roleplaying}

If something doesn't work, humans just try a different method.
If they can't buy what they want, they often just steal.
When they can't figure out a spell to make the plants grow, they turn to studying natural cultivation.
And when they can't open a door, they start hammering the walls.

Humans may seem dim when you watch them work, but come back a year later and you can often find them with the same goal, using a new technique.


