\chapter{Witchcraft}

The spells from the \textit{Core Rules} each express a single formula, and spellcasters can use that formula to craft their own spells, or create \glspl{talisman}.
Those with the knack for it can craft spells on the fly, while others require weeks of practice and focus to create their own spells.

A spell's heart comes from a primitive sentence, usually left implied rather than spoken.
The actual language could express the spell as chatter, sign-language, theatre or song.

\vspace{\baselineskip}

\begin{wideTable}[|l|>{\scshape\raggedright\arraybackslash}p{.15\linewidth}>{\scshape\arraybackslash}l>{\scshape\arraybackslash}l|>{\small\arraybackslash}L|]%
  {The Anatomy of Magic}
  & \textbf{Descriptors} & \textbf{Action} & \textbf{Target} & \normalsize\scshape\textbf{Result} \\
  \hline
  \hint{1} & ---       &  Wax            & Air             &
                       A sharp breeze tugs. \\
  \hint{2} & Detailed  &  Wane           & Air             &
                       Mist scatters away from the caster, as a feint vortex follows them, and pushes a dog-shaped impression into the chalky air. \\
  \hint{2} & Duplicated&  Wane           & Earth           &
                       Snow melts all around, forming a small river on a bed of ice. \\
  \hint{3} & Duplicated, Detailed
                       &  Wane           & Fate            &
                       Two-dozen men lose their luck with archery.  They don't know it, but some suspect, while others argue that the issue is simply aptitude. \\
  \hint{4} & Detailed, Distant, Duplicated
                       &  Warp           & Fate            &
                       The target was destined to have an upcoming encounter with a chitincrawler, the next day.  But now their destiny is twisted into a loop -- they will encounter another chitincrawler each day for a week. \\
  \hint{2} & Detailed  &  Warp           & Fate, Water &
                       The shepherd dreams of herding sheep and banal conversations all night, but by daylight he will think he can fly, then lives in a castle which is also his bedroom. \\
  \hint{2} & Divergent &  Warp           & Fire, Water     &
                       The hearth-fire spills out, and the floods across the floor; the soup in the cauldron above it swirls like a slow whirlwind, and dances over the cauldron's rim, flashing liquid fans straight up like a flame.  The house begins to burn, as nobody knows how to kill a fire that move like liquid.  \\
  \hint{1} & ---       &  Witness        & Water       &
                       Her ear to the ground, the witch confirms that water lies a few steps below the house.  The \gls{village} will build a well right here, once they move the house.  Or perhaps the witch will say nothing.  \\

\end{wideTable}

\section{Weaving Spells}

\begin{multicols}{2}
\index{Spheres}

\begin{figure*}[t!]
  \centering
  \speltogram
  \label{speltogram}
  \index{Elements}
\end{figure*}

\subsection{Step by Step}

\mapentry{Elemental Mana}

Every spell starts with an elemental Sphere -- Earth, Water, Fate, Air, and Fire.
Casters can pour a number of \glsentrylongpl{mp} into a spell, up to their level in that Sphere, and every \gls{mp} goes a long way.

Spellcasters have precisely four way to agitate the world.

\paragraph{Waxing}
spells encourage the element to express its basic nature.

\begin{itemize}
  \item
  Stone, or ice hardens, and sand clumps together into a solid mass;
  \item
  streams rush faster down hill, and even ale in a mug may begin to swirl and froth gently in an effort to escape to the ground;
  \item
  fate wraps its attentions on someone, increasing their \glspl{fp};
  \item
  air rushes round and up, violently;
  \item
  and fire burns.
\end{itemize}

\paragraph{Waning}
magic does the opposite -- it stifles, scatters, or reverses the element's essential nature.

\begin{itemize}
  \item
  Snow melts easily, and ice melts with a little effort (\gls{tn} 7-9).
  Stone puts up far more resistance, but skill and patience may soften slate to the consistency of thick, wet, clay (\gls{tn} 12-14).
  Metals provide far more resistance, and may prove impenetrable to casters without plenty of preparation and \pgls{boon} (\gls{tn} 16-20).

  More complex stuff, like wood, or flesh, does not respond to Earth mages -- true, elemental Earth has only one form, running through it.
  Even a metal alloy can raise a spell's \gls{tn} significantly.
  \item
  Water stagnates and evaporates.
  Small puddles and mugs of ale may vanish quickly (\gls{tn} 6 and 9)
  but a flowing river or loch always requires a spell large enough to cover a massive chunk of its body (and a roll at \gls{tn} 12).
  \item
  Waning Fate does very little to the myriad wretches of \gls{fenestra}, who never had much chance to begin with.%
  \footnote{Most \glspl{npc} have no \glspl{fp}.}
  Those lucky enough to have luck to lose usually resist with \roll{Charisma}{Deceit}.
  \item
  Dead air stands still and milky, blocking vision (\gls{tn} 7);
  and once smokey enough it begins to choke people.
  Underground, these spells can blind or kill (\gls{tn} 9), but on a stormy mountain, casters will need to create a full cyclone%
  \footnote{See \vref{magic:duplicated}.}
  of thick smoke or simply fail altogether (\gls{tn} 16).
  \item
  Spells cannot destroy, but once a candle shrinks and scatters its energy, the fire dies a natural death (\gls{tn} 7).
  Hearth fires put up more of a fight (\gls{tn} 10), and torches made of pitch may grow dim.
\end{itemize}

\paragraph{Warping}
an element alters one of its fundamental aspects.

\begin{itemize}
  \item
  Twisted Earth may become brittle -- both extra-hard, and liable to shatter into sharp points.
  \item
  Water should want to spread out and down, and if that changes, it thickens into ice;
  or if its clarify fails, it can become poisonous, or even acidic.
  \item
  Warping Fate affects the \gls{gm}'s random encounters, sending them on loops, or bunching a week's encounters into a single day.%
  \footnote{Once \pgls{doula} curses someone, nobody wants to travel with them.  This is the real curse behind the spell, and doesn't even require a spell to be cast.}
  \item
  Air should be formless, but insistent witches can demand a few clear breaths of air stick close by, or coax mist into forming simple shapes.
  \item
  Blue flames, moving in slow motion, decorate the festivals every \gls{Laiquea}.
\end{itemize}

\paragraph{Witnessing}
work differently to the other actions -- the caster stops speaking, and starts to listen.
No spell is cast, but a receptive attitude must be adopted throughout the entire body, and mind.
And despite no `casting' occurring, mana must flow out, in order to form a bond with an element which might lie in that direction.

When \pgls{pc} wants to Witness an element, the player must ask a `yes/no' question, which indicates what the \gls{pc} has attuned to.

Walls go a long way towards blocking anyone from Witnessing an element, which makes the activity close to pointless, at least with the spell's basic form.
The observer can look at a fire, and confirm that it is fire.
These spells become rather more useful once combined with spell `Descriptions'.

\mapentry{Descriptors}

Descriptors increase a spell's properties, giving it a longer range, or copying the spell across \pgls{area}.
Each Descriptor makes the effects more powerful, and makes every other Descriptor more powerful.

Pour a little more mana into a spell, and it blossoms quickly in one direction.
Give a little more to some other part of the spell, and both areas swell up even more.

The spell-patterns outlined above each cost 1~\gls{mp}.
Casters can spend a number of \glspl{mp} equal to their rating in a Sphere, so someone with Earth~2 can spend 1~\gls{mp} to cast a \textit{Wax Air} spell to push someone back with a blast of air, then spend another \gls{mp} to make the spell \textit{Duplicated}, affecting many more people with the blast.

\paragraph{Detailed}
spells let you control the appearance and minute properties of a spell.
Where most spells have no more shape than one might expect of a gust of wind or stalagmite, Detailed spells look like something.

\begin{boxtable}
  \textbf{Level} & \textbf{Visages}       \\
  \hline
      1          & A lumpy pile of substance. \\
      2          & A candle-flame which seems reminiscent of a face.  A section of river freezing into the shape of a bridge. \\
      3          & \textit{Wane Earth} spells shatter ice, but leave behind impressions of people or beasts which the caster knows well. \\
      4          & Summoned mist looks like the caster's face (or how they think of their own face). \\
      5          & \textit{Warp Fate} spells breed encounters that resemble pivotal moments of the caster's life.  \textit{Wax Death} spread pustules across a target in the shape of a map where the caster lives.  \\
      6          & Every \textit{Warp Water} spell looks like the caster, or an intimate face. \\
\end{boxtable}

A little details can help a spell a precise tool, or target only the caster's enemies, and never their allies.
However, as the spells gain more detail, they become restricted to the caster's most vivid memories, and eventually cannot represent anything but people or places they love or hate.%
\footnote{Spellcasters who end a close relationship abruptly often haunt themselves for decades through their own spells.}

A couple of \glspl{mp} spent to observe with \textit{Detailed Witness} allow a witch to ascertain if an \textit{oak} fire burns inside a building, or if a casket contains wine.
As before, all questions will only receive a `yes/ no' answer.

\begin{enumerate}
  \item
  \begin{description}
    \item[Distant]
    spells throw your magic far into the distance, and no less.
    These spells cannot target shorter distances.

    \begin{boxtable}[cLL]
      \textbf{Level} & Standard Distance & Enhanced Distance       \\
      \hline
                  1              & 20 \glspl{step} & ---               \\
                  2              & 15 \glspl{step} & throwing distance \\
                  3              & 10 \glspl{step} & shouting distance \\
                  4              &  5 \glspl{step} & horizon           \\
                  5              &  ---            & line of sight     \\
    \end{boxtable}
    \item[Divergent]
    spells produce two effects from opposing Spheres, at the same time.
    \begin{itemize}
      \item
      A \textit{Divergent Wane Fire, Fate} spell would make a torch dim to nothing, while another target loses \glspl{fp}.
      \item
      While a \textit{Divergent, Duplicated, Wane Mind} spell would create a lot of shadow, while making a lot of people feel confused.
    Check the Speltogram \vpageref{speltogram} to see all opposing Spheres.
      \item
      Any Spheres which do not neighbour each other, oppose each other.
      \item
      As usual, all Spheres used must be equal in level.
    \end{itemize}
    \item[Duplicated]
    spells fork like lightning, affecting every available target nearby.
    The caster does not select the remaining targets.
    The total number of targets equals $L^L$, where $L$ is the spell's level.
    \begin{itemize}
      \item
      Fire spells target all nearby fires.
      \item
      Earth spells usually target a large patch of rock or ice, as all the substance around makes for other viable targets.
      \item
      \begin{boxtable}[YY]
        \textbf{Level} & Targets \\
        \hline
                    1              & 1 \\
                    2              & 4 \\
                    3              & 9 \\
                    4              & 256 \\
                    5              & 3125 \\
      \end{boxtable}
    \end{itemize}
  \end{description}
  \item
  Roll to resolve, with \roll{Charisma}{Sphere}, while stating your intention!
  \begin{itemize}
    \item
    The element's resistance may raise or lower the \gls{tn}.
    \item
    Breaking hard earth is difficult, but snow is easy.
    Blasting wind in a house is difficult, but redirecting a storm is not.
    \item
    If the effects target a person, then the person can \emph{also} resist the effects with any appropriate combination of Traits.
  \end{itemize}
  \item
  Apply effects.
  \label{sphereEffects}
  \begin{itemize}
    \item
    Low Spheres create mechanical effects equal to their level +2.
    \item
    High Spheres create a mechanical effect equal to their level +1.
  \end{itemize}
\end{enumerate}

\subsection{Principles of Magic}

Once you read enough spells, they should start to feel repetitive.
The steps above were used to create each one, so once the spells in the Core Rules start to feel familiar, making spells on-the-fly should become quickly intuitive.

On top of those steps, keep the general principles below in mind when crafting new spells.

\subsubsection{Plan or Die}

Sooner or later someone will fail to see a spell's natural consequences.
Someone may cast a spell to make their enemies shrink and become weak, then notice the spell's range is `to the horizon', which means the spell cannot actually target those enemies.
Or perhaps a spell intended to kill all seven bandits has twenty-four targets, and therefore targets the bandits, all of the \glspl{pc}, every ally, and the \gls{doula}'s pet magpie.
The system intends for players and the \gls{gm} to make these mistakes.

Magic creates unintended side-effects, but they don't fall out of random rolls and a `random magic effect' table.
They come from the \ref{sphereEffects} principles above.

\subsubsection{Potent, Yet Barren}

Magic cannot create.
It warps, waxes, and wanes.
It gives information.
But nothing in the world comes from nothing, so spellcasters make the world pulse, or transform, but cannot remove or add the sum of all that is the case.
Witches may control fire, but only a fire striker can \textit{make} it.

\subsubsection{Spells Last Until Death}

Spells have no set `duration'.
Fires burn until their fuel runs out.
The undead burn through the spell which animates them every time they move.
Life spells which bolster a target's Bonuses often demand the target burn through calories in order to keep the spell going, and then fade to nothing once the spell ends.

The first principle here is that if something seems like it could stop the effects of a spell, it stops the spell.
The second principle is that spells can last forever, but they do not \emph{produce} anything forever.

\subsubsection{Spells Stack as Usual}

Stacking spell effects should work exactly like any other Stacking.%
\exRef{core}{Core Rules}{stacking}
Casting two Life spells, which both inflict a -2 Penalty to someone's Speed results in a -2~Penalty, because these spells do the same thing.
However, two different Spheres which create a Speed Penalty would Stack as usual.

So spell Stacking works exactly like armour -- wearing chainmail on top of plate will not work well, but a character could gain \gls{dr} through a touch hide, and then add armour to gain half the \gls{dr} for the lower of the two protection-types.

\end{multicols}

\section{\Glsfmttext{ingredient} Workshop}

\begin{multicols}{2}

\noindent
`\Glspl{ingredient}' means anything which can help with arcane practices, and each \gls{ingredient} aligns to one of the Low Spheres.
Most \glspl{ingredient} come from the bodies of beasts, such as a basilisk's gullet, or griffin feathers.

\subsubsection{\Glsfmttext{boon} Creation}

Casters can add a little extra power to their spells by taking raw \glspl{ingredient}, and crafting them into \glspl{boon}.
\Glspl{boon} come in the form of powders or liquids, to be thrown in the air at the point of casting, and add +1 to any Sphere.

\begin{itemize}
  \item
  \Glspl{boon} demand at least 1 \gls{ap} to use from the caster's hand, as usual.
  \item
  \Glspl{boon} remain suffused in the air until someone (anyone) consumes the energy by casting a spell.
  \item
  \Glspl{boon} can increase multiple Spheres by combining multiple \glspl{ingredient}, but only by 1 level each.
\end{itemize}

\noindent
With the help of a Fire \gls{boon}, someone without any magical ability could cast a level 1 Fire spell, or a caster with Fire 2, Earth 3 could cast a level 3 Force spell.

\subsubsection{\Glsfmttext{talisman} Creation}

With the right \glspl{ingredient}, one can create \pgls{talisman}, which stores a spell, and releases it once the conditions have been met.

\Glspl{talisman} require \glspl{ingredient}, rather than magical Spheres to cast.
Anyone can make one, with the right knowledge, and the ability to coax a little life into the material world with the right words.

By default, \pgls{talisman} will target the nearest thing it can.
However, many casters add a \textit{Witness} spell to the \gls{talisman}, to let it target something else.

\begin{enumerate}
  \item
  The creator starts by arranging their \glspl{ingredient} -- these will give effective access to spell Spheres.
  Casting with Fire 2, Earth 2 would require 2 Earth \glspl{ingredient} and 2 Fire \glspl{ingredient}.
  \item
  Crafters must make \glspl{talisman} from materials aligned to the Spheres in use.
  \begin{description}
    \item[Air]
    uses contained gasses, often held in a phial.
    \item[Earth]
    requires any solid materials dug from the ground.
    \item[Fate]
    talismans require the remains of someone who had \glspl{fp} in life (at least 1 per \gls{ingredient} used in the spell).
    \item[Fire]
    requires flammable material.
    \item[Water]
    can use any liquid.
  \end{description}
  \item
  Next, the \gls{talisman} needs a spell to cast, once it detects something.
  Fire spells will, by default, target the nearest fire they can; Air spells will almost certainly target surrounding air.
  \begin{itemize}
    \item
    The \gls{talisman} casts its spell with a bonus equal to the spell's level, so a level 3 Air spell with a cost of 3 \glspl{mp}, made into \pgls{talisman}, casts its spell with a +3 bonus, rather than the normal roll of \roll{Charisma}{Air}.
    \item
    The creator may include a \textit{Witness} spell which the \gls{talisman} will use to detect the target.
    It will then target the nearest thing the \textit{Witness} spell finds.
    \item
    This \textit{Witness} spell requires further \glspl{ingredient}, but it does not require another roll to construct.
  \end{itemize}
  \item
  The creator then decides upon the talisman's activation conditions -- will it activate once it sees starlight?
  Or awaken to any loud noises?
  Or activate when someone solves a riddle, engraved on its surface?
  \item
  The alchemical process requires an \roll{Charisma}{Academics} check, \tn[10] plus double the spell's cost.
\end{enumerate}

\subsubsection*{Examples}

\begin{exampletext}
Courtblight wants to make \pgls{talisman} to help her companions flee when they're in trouble.

\begin{itemize}
  \item
  She can't cast Force spells herself, but she has two Earth \glspl{ingredient} and two Fire \glspl{ingredient}, so that can make \pgls{talisman} with a level 2 \textit{Wax Duplicate Force} spell.
  \item
  She takes the \glspl{ingredient} (two woodspy eyes, and a lot of basilisk hide) and boils both for several days, reducing them to an inky paste.
  \item
  She takes her materials (paper for the Fire, crushed amethyst for Earth) and works the dust into the paper, then writes across it with the inky paste from her pot.
  \item
  Finally, she whispers the activation words, `let me fly', into the scroll, over and over, with some doodles of griffins around the margins.

  With this, the player rolls \roll{Charisma}{Academics} at \tn[14].
\end{itemize}

Once done, she casts a spell to detect the mana locked in the scroll, and finds nothing\ldots the ritual awoke nothing, the item will remain fruitless, and the \glspl{ingredient} wasted.

She decides that next time, she will ask for help creating the item.

\end{exampletext}

\end{multicols}

