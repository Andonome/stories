\chapter{Witchcraft}
\label{witchcraft}

\noindent
The spells from the \textit{Core Rules} each express a single formula.
Once \pgls{witch} has sufficient practice with the basic spells, they can begin to make their own \glspl{spell} during \gls{downtime}.
A spell's heart comes from \pgls{invocation} -- a primitive sentence, never spoken but heavily implied.
The actual \gls{casting} could express the spell as chatter, humming, sign-language, theatre or song.

\vspace{\baselineskip}

\begin{wideTable}[l|>{\scshape\raggedright\arraybackslash}p{.12\linewidth}>{\scshape\arraybackslash}l>{\scshape\arraybackslash}l|>{\small\arraybackslash}X]%
  {The Anatomy of Magic}
  & \textbf{\Glsfmtplural{descriptor}} & \textbf{Action} & \textbf{Target} & \normalsize\scshape\textbf{Result} \\
  \hline
  \hint{1} & ---       &  Wax            & Air             &
                       A sharp breeze tugs. \\
  \hint{2} & Detailed  &  Wax           & Air             &
                       Mist scatters away from the caster, as a feint vortex follows them, and pushes a dog-shaped impression into the chalky air. \\
  \hint{2} & Duplicated&  Wane           & Earth           &
                       Snow melts all around, forming a small river on a bed of ice. \\
  \hint{3} & Duplicated, Detailed
                       &  Wane           & Fate            &
                       Two-dozen men lose their luck with archery.  \\
  \hint{4} & Detailed, Distant, Duplicated
                       &  Warp           & Fate            &
                       The target was destined to have an upcoming encounter with \pgls{crawler}, the next day.  But once they hear that foreboding song, their destiny is twisted into a loop -- they will encounter another \gls{crawler} each day for a week. \\
  \hint{2} & Detailed  &  Warp           & Mind &
                       The shepherd dreams of banal conversations and herding sheep all night, but by daylight he will think he can fly, then lives in a castle which is also his bedroom. \\
  \hint{2} & Divergent &  Warp           & Fire \& Water     &
                       The hearth-fire spills out, and the floods across the floor; the soup in the cauldron above it swirls like a slow whirlwind, and dances over the cauldron's rim, flashing liquid fans straight up like a flame.  The house begins to burn, and nobody knows how to kill a fire that moves like liquid.  \\
  %\hint{1} & ---       &  Witness        & Water       &
  %                     Her ear to the ground, the witch confirms that water lies a few steps below the house.  The \gls{village} will build a well right here, once they move the house.  Or perhaps the witch will say nothing, if they like the homeowner.  \\

\end{wideTable}

\section{Weaving Spells}
\label{spellWeaving}

\begin{multicols}{2}

\subsection{Step by Step}

These are the conceptual steps you can use to create spells during \gls{downtime}.
\iftoggle{core}{%
  Once you have a spell on-paper, check the \textit{Core Rules}, \autopageref{make_spell}, to see if your character can manage the task, or what might happen if they fail.
}{}

\mapentry{Elemental Mana}
determines the target.
Every spell starts with an elemental \gls{sphere} -- Earth, Water, Fate, Air, or Fire.
Casters can pour a number of \glsentrylongpl{mp} into a spell, up to their level in that \gls{sphere}, and every \gls{mp} goes a long way.

Spells have precisely four way to agitate the world.

\paragraph{Waxing}
spells encourage the element to express its basic nature.

\begin{itemize}
  \item
  Stone, or ice hardens, and sand clumps together into a solid mass;
  \item
  streams rush faster down hill, and even ale in a mug may begin to swirl and froth gently in an effort to escape to the ground;
  \item
  fate wraps its attentions around someone, giving them good luck;
  \item
  air rushes violently;
  \item
  and fire burns.
\end{itemize}

\paragraph{Waning}
magic does the opposite -- it stifles, scatters, or reverses the element's essential nature.

\begin{itemize}
  \item
  Snow melts easily, and ice melts with a little effort (\gls{tn}~7-9).
  Stone puts up far more resistance, but skill and patience may soften slate to the consistency of thick, wet, clay (\gls{tn}~12-14).
  Metals provide far more resistance, and may prove impenetrable to casters without plenty of preparation and \pgls{boon} (\gls{tn} 16-20).

  More complex stuff, like wood, or flesh, does not respond to Earth magic -- pure, elemental Earth has a pure consistency.
  Even a metal alloy can raise a spell's \gls{tn} significantly.
  \item
  Water stagnates and evaporates.
  Small puddles and mugs of ale may vanish quickly (\tn[6] and 9)
  but a flowing river or loch always requires a spell large enough to cover a massive chunk of its body (and a roll at \tn[12]).
  \item
  Waning Fate does very little to the myriad wretches of \gls{fenestra}, who never had much chance to begin with.%
  \footnote{Most \glspl{npc} have no \glspl{fp}.}
  Those lucky enough to have luck to lose usually resist with \roll{Charisma}{Deceit}.
  \item
  Dead air stands still and milky, blocking vision (\tn[7]);
  and once smokey enough it begins to choke people (\tn[9]).
  Underground, these spells can blind or kill, but on a stormy mountain, casters will need to create a full cyclone
  of thick smoke or simply fail altogether (\tn[16]).
  \item
  Spells cannot destroy, but once a candle's flame shrinks and scatters its energy, the fire dies a natural death (\gls{tn}~7).
  Hearth fires put up more of a fight (\gls{tn}~10), and \glspl{torch} made of pitch may grow dim.
\end{itemize}

\paragraph{Warping}
an element alters one of its fundamental aspects.

\begin{itemize}
  \item
  Twisted Earth may become brittle -- both extra-hard, and liable to shatter into sharp points.
  \item
  Water should want to spread out and down, and if that changes, it thickens into ice;
  or if its clarify fails, it can become poisonous, or even acidic.
  \item
  Warping Fate affects the \gls{gm}'s random encounters, sending them on loops, or bunching a week's encounters into a single day.%
  \footnote{Once \pgls{doula} curses someone, nobody wants to travel with them.  This is the real curse behind the spell, and doesn't even require a spell to be cast.}
  \item
  Air should be formless, but insistent witches can demand a few clear breaths of air stick close by, or coax mist into forming simple shapes.
  \item
  Blue flames, moving in slow motion, decorate the Frozen Festival every \gls{cFive}.
\end{itemize}

\paragraph{Witnessing}
work differently to the other actions -- the caster stops speaking, and starts to listen.
No spell is cast, but a receptive attitude must be adopted throughout the entire body, and mind.
And despite no `casting' occurring, mana must flow out, in order to form a bond with any target which might lie in that direction.

When \pgls{pc} wants to Witness an element, the player must ask a `yes/no' question, which indicates what the \gls{pc} has attuned to.

Far-removed targets, or targets behind a wall receive a higher \gls{tn}, which makes a basic Witnessing close to useless (since the witch effectively has a `Detect Fire' or `Detect Air' spell).
However, with a couple of \glspl{descriptor} (vref{descriptors}), a witch can Witness many secrets.

\mapentry{\Glsfmtplural{descriptor}}
\label{spellDescriptors}
increase a spell's properties, giving it a longer range, or copying the spell across \pgls{area}.
Each \gls{descriptor} makes the effects more powerful, and makes every other \gls{descriptor} more powerful.

Pour a little more mana into a spell, and it blossoms quickly in one direction.
Give a little more to some other part of the spell, and both properties swell up even more.

The spell-patterns outlined above each cost 1~\gls{mp}.
Characters can spend a number of \glspl{mp} equal to their rating in \pgls{sphere}, so someone with Earth~2 can spend \pgls{mp} to cast a \textit{Wax Air} spell (and push someone back with a blast of wind), or spend 2~\glspl{mp} to cast a \textit{Duplicated Wax Air} spell (and push many people back, with a more powerful blast of wind).

\paragraph{Detailed}
\glspl{casting} let you control the appearance and distinctions of \pgls{spell}.
Where most \glspl{spell} have no more definition than one might expect of a gust of wind or stalagmite, \textit{Detailed} spells look like something.

\begin{boxtable}
  \textbf{Level} & \textbf{Visages}       \\
  \hline
      1          & A lumpy pile of substance. \\
      2          & A candle-flame which seems reminiscent of a face.  A section of river freezing into the shape of a bridge. \\
      3          & \textit{Wane Earth} spells shatter ice, but leave behind impressions of people or beasts which the caster knows well. \\
      4          & Summoned mist looks like the caster's face (or how they think of their own face). \\
      5          & \textit{Warp Fate} spells breed encounters that resemble pivotal moments of the caster's life.  \textit{Wax Death} spread pustules across a target in the shape of a map where the caster lives.  \\
      6          & Every spell looks like the caster, or an intimate face. \\
\end{boxtable}

As the spells gain more detail, they become restricted to the caster's most vivid memories, and eventually cannot represent anything but people or places they love or hate.%
\footnote{Spellcasters who end a close relationship abruptly often haunt themselves for decades through their own spells.  Every storm looks like their hair, when they want to warp light to create the appearance of a dragon, the dragon looks all too human, and always like the same human.}
This restricts the possible forms of the spell to people and places the caster knows intimately, and provides a strong clue about the caster's identity, making it difficult to cast spells secretly.

A couple of \glspl{mp} spent on a \textit{Detailed Witness} allow a witch to ascertain if an \textit{oak} fire burns inside a building, or if a casket contains wine.
As before, all questions will only receive a `yes/ no' answer.

\paragraph{Devious}
spells are hinted and murmured, but don't require the same flashy gestures as the others, so people often cannot tell when someone casts them.
However, they also take a long time to reach fruition.

\sidebox[20]{
  \begin{boxtable}
    \textbf{Level} & \textbf{Time} \\
    \hline
      2            &  \Pgls{interval} \\
      3            &  A day      \\
      4            &  A week      \\
      5            &  \Pgls{cycle} \\
      6            &  A decade     \\
  \end{boxtable}
}
The higher the level, the longer the spell will take to reach full power.
They sit dormant for half of the required time, and during the other half, they rise slowly -- one point at a time.

A \textit{Devious Wax Fate} spell which grants $1D6$~\glspl{fp} will wait for half \glspl{interval}, then add half the rolled \glspl{fp}, then the second half thereafter.
A \textit{Devious, Detailed, Wane Fire} spell could wait for two \glspl{interval}, then make a hearth go dim and throw out smoke in the form of a black wolf.
A \textit{Devious, Detailed, Distant, Warp Water} spell cast on a stream could wait for three days, then slowly push an icy replica of the castle which hold the caster prisoner.

If someone notices one of these spells growing, they can put a stop to the spell just like any other.

\paragraph{Distant}
magics can strike across vast distances, beyond where even the best archers can reach, and even beyond vision.
\set{spellCost}{3}\setRange\toggletrue{Distant}%
However, the caster cannot limit or contain these spells; a \textit{Detailed,  Distant, Wax Air} spell begins at `\spellRange'.
Not `up to a maximum of ``\spellRange'', if you please', but at that minimum and maximum.
Spells all have some leeway in their range (as the vague descriptor of distance ought to imply), but they cannot be expanded and later limited.
\togglefalse{Distant}

\begin{boxtable}[cLL]
  \textbf{Level} & \textbf{Standard Distance} & \textbf{Enhanced Distance}        \\
  \hline
    \setcounter{spellCost}{1}
    \arabic{spellCost}
      & \setRange\spellRange & ---                  \\
    \stepcounter{spellCost}
    \arabic{spellCost}
      & \setRange\spellRange & \toggletrue{Distant}\setRange\spellRange \\
    \stepcounter{spellCost}
    \arabic{spellCost}
      & \setRange\spellRange & \toggletrue{Distant}\setRange\spellRange \\
    \stepcounter{spellCost}
    \arabic{spellCost}
      & \setRange\spellRange & \toggletrue{Distant}\setRange\spellRange \\
    \stepcounter{spellCost}
    \arabic{spellCost}
      & \setRange\spellRange & \toggletrue{Distant}\setRange\spellRange \\
    \stepcounter{spellCost}
    \arabic{spellCost}
      & ---                  & \toggletrue{Distant}\setRange\spellRange \\
\end{boxtable}

Of course, casters must clearly perceive something in order to cast upon it, so anything outside of clear vision demands a \textit{Distant Witness} precursor.
Once casters can \textit{Witness} someone, they can cast upon them, for the short moment when while the awareness lasts.
Of course, guessing where rivers and houses in the distance might be does not have a high success rate, so setting up these \textit{Distant} spells requires patience, planning, and a way to keep precise records about trajectories.

These \textit{Distant Witness} deductions and maps can take weeks or years of planning for targets far beyond sight.

\begin{exampletext}
  The ability to gain knowledge from a distance allows \glspl{doula} and elves to produce excellent maps.
  Combining this \textit{\gls{descriptor}} with \textit{Distant} and \textit{Duplicated}, allows them to inquire about a large area, and then narrow down the space they want to know about.
  Of course, this process requires a lot of skill -- simply locating \pgls{village} from a distance might require listening for something \textit{Detailed} (for the specific type of rock), \textit{Distant}, and \textit{Duplicated}.
  One might then discover the location of a particular person by seeking out the sapphire pendant they wear, using \textit{Detailed Witness Earth}.

  These spells also make long-ranged warfare possible, as casters can send storms, bad luck, or poisoned water after each other.
  Two witches having a duel to the death looks less like a spark-filled inferno of spell-slinging, and more like a protracted game of battleship, with plenty of involvement for bystanders any time one guesses wrong.
\end{exampletext}

\begin{figure*}[t!]
  \centering
  \speltogram
  \label{speltogram}
  \index{Elements}
\end{figure*}

\paragraph{Divergent}
spells channel mana through opposing \glspl{sphere} to create two parallel, but divergent, effects at the same time.
Check the chart \vpageref{speltogram}, and notice that each \gls{sphere} has two neighbouring \glspl{sphere}, and two opposing \glspl{sphere}.
Divergent spells construct a single spell-sentence to be used by two opposing \glspl{sphere} at the same time.

For example, Water sits between its neighbours Earth and Fate, but opposes Air and Fire.
So a caster with Water 2 and Air 2 could create a \textit{Divergent, Wax Water \& Air} spell, which would make surrounding waters jump and thrash, while wind rages.

These two spells will affect different substances, and the caster can direct them at different targets.

\begin{itemize}
  \item
  A \textit{Divergent Wane Fire, Fate} spell would make \pgls{torch} dim to nothing, while the torch-bearer target loses \glspl{fp}.
  \item
  Encountering some bandits, \pgls{doula} uses a \textit{Duplicated, Divergent, Warp Fire \& Fate} spell.
  Their campfire burns green, and their luck shifts -- the possible forest encounters swing strongly in favour of \glspl{basilisk}.
  \item
  Using \textit{Divergent Witness Water \& Air}, \pgls{pc} could find out if a cavern of air or water existed beneath them (although they would not know which -- only that `yes, one of them exists').
  \item
  While a \textit{Divergent, Duplicated, Wane Mind \& Light} spell would create a large patch of darkness, and plunge anyone inside into confusion.
\end{itemize}

As usual, casters make a single roll to resolve the spell, and apply the result to everything.
A \textit{Divergent, Distant, Wane Earth \& Fate} spell may require a roll of \tn[7] to give someone bad luck, and \tn[14] to make the rock-face above them crumble.
If the caster rolls a total of 13, they would inflict the bad luck, but not make the rock-face crumble.

\paragraph{Duplicated}
\label{duplicatedDesc}
spells fork like lightning, affecting every available target nearby.
The caster selects where the spell begins, but cannot decide on the remaining targets, nor can they limit the number.%
\footnote{Each number is simply $n^n$.  So spending 3~\glsfmtplural{mp} means $3^3 = 3\times 3\times 3 = 27$ targets.}

When \textit{Duplicated} spells seek out discrete targets (like Fate spells increasing people's \glspl{fp}, or a Fire spell suffocating many candles) each copy of the spell can jump the same range as the original spell's range.
This can present problems; if a caster makes three enemy \glspl{torch} explode in their faces, but the spell requires four targets, then the spell may return to unleash itself on the caster's \gls{torch}.
Spells which cannot find an appropriate target within range just fizzle and die, without effect.

\sidebox[17]{
  \begin{boxtable}[cX]
    \textbf{Level} & \textbf{Targets} \\
    \hline
           1       & 1      \\
           2       & 4      \\
           3       & 27     \\
           4       & 256    \\
           5       & 3,125  \\
           6       & 46,656 \\
  \end{boxtable}
}

When \textit{Duplicated} spells seek out contiguous targets, they simply affect a wider area
(Water spells can target large sections of a loch, and Earth spells might collapse a cavern).%
\footnote{The incredible range and power of high-level spells means that any attempt at wide-scale wars runs the risk of a skilled caster targeting one, or both, armies with a crippling spell.
  This simple fact has stopped a number of wars before they began.

  However, someone who can fight on, despite blinding fog, a snowstorm, or their \glspl{torch} and camp-fire suddenly going out, will not be more troubled by a spell simply because it affects a thousand more people.
  As a result, a few specialist trackers, or veteran \glspl{guard}, can often fare better against powerful spell-casters than an army of field-workers with swords.}


Casters often try to predict whether a spell will treat a target as discrete or contiguous, but they have no control over the result.

\subsubsection{Resolution \& Effects}
may sound very open-ended, or even badly-defined, and they are!
However, the results are not.
The five elemental \glspl{sphere} have a mechanical, numerical result equal to the number of \glspl{mp} spent~+2.

\begin{itemize}
  \item
  A \textit{Wax Earth} spell which solidifies snow over a door, and stops anyone entering, costs 1~\gls{mp}, so it inflicts a -3 penalty to break down the door.
  \item
  A \textit{Duplicated Wane Air} spell which blinds archers with mist requires 2~\glspl{mp}, so it inflicts a -4 penalty to using arrows at any kind of range.
  \item
  A \textit{Divergent, Detailed Wax Fate, \& Fire} spell could bless the caster's ally with a +1 Bonus to using the Melee Skill over their next 5 actions, and make an enemy's \gls{torch} transform into a monstrous face which attempts to eat them.
  The burning \gls{torch} inflicts 5~Damage, which translates to \dmg{5}~Damage.%
  \exRef{core}{Core Rules}{stackingDamage}
\end{itemize}

The final part of the spell is the resistance -- the \gls{gm} stipulates some standard force which stands in the way of the spell.
Sometimes spells have obvious forces acting against them (such as the strength of a rock, when one wants it to crumble), while other spells have less obvious barriers (good-luck spells often have trouble taking effect when the recipients are not paying attention).

\subsection{The High \glspl{sphere}}

As you'll see on the chart (\vpageref{speltogram}), casters can access five `high \glspl{sphere}' by combining two elemental \glspl{sphere}.
Using the high \glspl{sphere} doesn't require a new Skill -- casters simply combine their the elemental \glspl{sphere}, and have an effective level equal to the lowest of the two.

\begin{exampletext}
For example:

\begin{boxtable}[XX]
  \textbf{Elemental \glspl{sphere}} & \textbf{High \Gls{sphere}} \\
  \hline
  {\normalfont Casters with these:} & {\normalfont can also use these:} \\
  \begin{itemize}
    \item
    Air~3
    \item
    Fire~2
    \item
    Earth~1
  \end{itemize}
  &
  \begin{itemize}
    \item
    Light~2
    \item
    Force~1
  \end{itemize}
  \\
\end{boxtable}

If this caster later learned Fate~1, they would also gain Death~1.

\end{exampletext}

The spells listed across the \textit{Core Rules} don't list the high \glspl{sphere}, so players can simply look at any spell's requirements and note whether or not they have the right \glspl{skill}.

\subsubsection{Resolution \& Effects}
\label{highResolution}
work much like the elemental \glspl{sphere}, but the mechanical result equals the \glspl{mp} spent +1 (not `+2', like the elemental \glspl{sphere}).
They have less raw power, but generally target people more directly.
Instead of relying on targets being in water, or under a crumbling cliff-face, \pgls{witch} can combine Water + Earth to use the Life \gls{sphere}, and twist targets' bodies directly.
Where elemental \glspl{sphere} can remove someone's \glspl{fp}, and use blasts of wind to knock them back, a caster with both could combine the effects into the Death \gls{sphere}, and make the targets die.

\subsection{Principles of Magic}

\subsubsection{Bodies in Motion}
run out of `spell-power' if the spell would provide `free fuel', so fire will always burn out without wood, and waves will calm down in time.
The undead burn through the spell which animates them every time they move.
Life spells which bolster a target's Bonuses often demand the target burn through calories in order to keep the spell going, and then fade to nothing once the spell ends.

If you need to track a spell's duration, assume it has a number of points equal to the \glsentrylongpl{mp} used to cast it,%
\footnote{Duplicated spells would divide their initial \glspl{mp} among every copy, but each copy can still rejuvenate itself as if it had been created alone.}
then remove a point every \gls{interval} in which it exerts some force or performs some action.

\subsubsection{Bodies at Rest}
remain unchanged, and the magic remains unspent.
A castle full of deadly Light and Force \glspl{spell} will remain deadly until the \glspl{spell} kill someone; until then, they wait.

Life spells which make tentacles grow out of someone's back will require food to keep the tentacles moving, but if the target simply grows horns, then the horns require no maintenance (although their \gls{weight} will still create a burden).

\subsubsection{Stacking \Glsfmtplural{spell}}
never works with the same kind of spell effect.
Using two Light \glspl{spell} to blind someone has no additional effect -- only the highest counts.
However, a Light spell to blind someone \textit{can} stack with an Air \gls{spell} which blurs someone's vision with mist.

The same principle applies to casters who use Life magic to increase their \gls{dr} -- the second-highest Bonus only counts for half.

\subsubsection{\Glsfmttext{bandAct} \Glsfmtplural{casting}}
work exactly like any other \gls{bandAct}.%
\exRef{core}{Core Rules}{banding}
All casters must be able to cast the spell, and they all spend the \glsentrylongpl{mp}.
This does not increase the spell's potency, it only increases the \gls{casting}~Bonus.

\subsubsection{Potent, Yet Barren}
magic cannot create.
It warps, waxes, and wanes.
It gives information.
But nothing in the world comes from nothing, so spellcasters make the world pulse, or transform, but cannot remove or add the sum of all that is the case.
Witches may control fire, but only a fire striker can \textit{make} it.

\subsubsection{Witnessing is Guesswork}
because all of these spells ultimately produce `yes/ no' answers.
However, they can become wide-spread, and detailed.

In general, a non-Detailed Witness spell can only ask `is there a mind?', or `does this \gls{area} have fire?'.
While a Detailed Witness spell can ask `is there any dwarf over there?', at lower levels.
At the highest levels, the caster can ask about extremely specific targets, but only concerning objects or people they know extremely well.

A sixth-tier Witness spell may only say if one's child is feeling curious about breakfast, but not about whether or not they are alive.

\subsubsection{Think before You Cast!}
because sooner or later someone will fail to see a spell's natural consequences.
Someone may cast a spell to make their enemies shrink and become weak, then notice the spell's range is `to the horizon', which means the spell cannot actually target those enemies.
Or perhaps a spell intended to kill all seven bandits has twenty-four targets, and therefore targets the bandits, all of the \glspl{pc}, every ally, and the \gls{doula}'s pet magpie.

Magic creates unintended side-effects, but these side-effects don't fall out of a `random magic effect' table.
They come from players (including the \gls{gm}) misusing magic.

So be forewarned: magic is dangerous.

\end{multicols}

\section{\Glsfmttext{ingredient} Workshop}

\begin{multicols}{2}

\noindent
Spellcasters and alchemist can benefit from rare substances with magical power, called `\glspl{ingredient}'
These include \gls{griffin}~feathers, auroch hooves, and dragon eggs.
They can cure diseases, boost spells, and even lock a spell into an item, ready to cast at the right prompt.
They just need prepared correctly.

Preparation needs a clean space, and a clear mind, but never a second set of hands to manage.
Characters cannot use \glspl{restingaction} or Banding actions.

A good space for preparing \glspl{ingredient} grants a Bonus to any roll.

\begin{boxtable}[Lr]
  \textbf{Area} & \textbf{Bonus} \\
  \hline
  Campfire with a pot & +0       \\
  Clean kitchen with a cookery book and larder       & +1       \\
  Grand workshop with three fires, an assortment of pots, and a well-indexed grimoire & +2       \\
  Castle wing with library, glass-blowing room, and a dozen small crushing devices & +3       \\
\end{boxtable}

\subsubsection{Creating \Glsfmtplural{boon}}
requires reducing an \gls{ingredient} into a powder or liquid, slowly, then sealing it quickly to avoid rot; many use cheese-cloth or a phial.

After \pgls{interval}, the character rolls \roll{Intelligence}{Crafts}, at \tn[10].
Success means that the \gls{boon} can boost anyone's rating in the appropriate elemental \gls{sphere} by 1.
A tie indicates that the \gls{boon} emerges useable, but rank.
It will go rotten within $1D6$ days.

With the help of a Fire \gls{boon}, someone without any magical ability could cast a level 1 Fire spell, or a caster with `Fire 2', and `Earth 3' could cast a level 3 Force spell.

A single container might mix multiple \glspl{boon} together, so one container might infuse the air with Fire and Earth.
Someone could use this to boost their Fire Skill, then their Earth Skill; or they might use it in a single Force spell.

\paragraph{Failure}
to make \pgls{boon} properly means the energy releases instantly, resulting in a spell from that \gls{sphere}.
The type of spell depends on the Failure Margin.

\label{randomSpellFailure}
\begin{nametable}[ccL]{\Gls{ingredient} Failures}
  \textbf{Margin} & \textbf{Action} & \textbf{Example} \\
  \hline
    1             & Warp            &  Nearby fires glow blue. \\
    2             & Wane            &  Water evaporates, luck turns bad. \\
    3             & Wax             &  Hearth's ash turns solid, or air blows all the windows open. \\
\end{nametable}

\subsubsection{\Glsfmttext{alchemy}}
\label{alchemicalSpells}
\index{Spells!Alchemical}
only needs \glspl{ingredient} and knowledge to craft spells; it does not need any magical Skill.
Once the alchemist has a `recipe' for the spell, it works like this:

\begin{enumerate}
  \item
  The alchemist gathers a number of \glspl{ingredient} equal to a spell's \gls{sphere} requirement.

  If a spell requires Fate 3, and Water 3, the alchemists needs 3 Fate \glspl{ingredient} and 3 Water \glspl{ingredient}.
  \item
  The player rolls \roll{Intelligence}{Academics}.
  The \gls{tn} starts at 6, and each spell level adds 3.
  Rolling a tie means the spell succeeds, but it casts with a +0 Bonus.
  \item
  The spell takes effect the very moment the recipe has completed.
  \begin{itemize}
    \item
    The spell rolls with a Bonus equal to its level, so level 1 spells have a +1~Bonus, and resisting a level 2 spell would require a roll at \tn[9].
    \item
    The alchemist can try to flee a moment before the spell is completed, but this increases the \gls{tn} by~1.
    \item
    Players can spend \pgls{storypoint} to automatically pass, using the `Random Fact' story (\autopageref{randomFact}).
  \end{itemize}
\end{enumerate}

\paragraph{Failure}
means the spell's energies recombine in the wrong way, producing some random effect, just like \glspl{boon} (check the failures chart \vpageref{randomSpellFailure}).
However, \gls{alchemy} spells can produce much worse failures when they use more mana.

The \glspl{descriptor} are always selected in this order:

\begin{multicols}{2}
  \begin{enumerate}
  \raggedright
    \item
    Divergent (when~possible)
    \item
    Duplicated
    \item
    Detailed
    \item
    Distant
    \item
    Devious
  \end{enumerate}
\end{multicols}

\subsubsection{Creating \Glsfmtplural{talisman}}
works as above, but adds sentience to the mixture, to give the spell a goal.
All spells translate intention into change -- \pgls{talisman} blossoms that intention into a full intelligence with a \textit{Detailed Wax Mind} spell stuffed into the mixture.

The system works as above, with two more steps:

\begin{enumerate}
  \item
  The alchemist needs two Fate \glspl{ingredient}, and two Water \glspl{ingredient} to encourage the spark of mind into rudimentary sentience.
  \item
  A second roll, at \tn[9] must succeed, or the spell casts instantly.
\end{enumerate}

The mind created within \pgls{talisman} does not work like a regular person.
It has extremely limited value and interests.
It will not usually consider its own survival to be interesting -- it only wants to achieve the one thing its creator pushed into it when creating it.
If someone somehow communicates with it, it will not necessarily want to respond, or may only respond by talking about its special interest.
They do not understand deals, and if they did, would have no reason to trust them.

\end{multicols}
