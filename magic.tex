\chapter{Spell Weaving}

The spells from the core rulebook each express a single formula, and you can use that formula to craft your own spells.
The same system also lets those who understand the world craft \glspl{talisman}, whether or not they have innate magical abilities.

\section{The Rules}

\begin{multicols}{2}

\noindent
Magic cannot create.
It warps, waxes, and wanes.
It gives information.
But nothing in the world comes from nothing, so spellcasters have to work with the environment, but they can only change what their Spheres govern.

Casters create a spell's foundation by deciding what \emph{action} to perform on which element.
Fire casters may \textit{Wax Fire}, causing it to roar and explode.
Water casters can \textit{Wane the River}, making it evaporate.
\textit{Witness Fate} spells allow the caster to know if fate has plans for someone nearby.
So each of the five Spheres allows the caster to create four basic spells -- wax, wane, warp, and witness.

More advanced casters can add enhancements, though they come with complications and costs.
The `duplicated' enhancement lets spells affect a wider area, or more targets, but the caster may inadvertently target allies.
`Distant' spells can affect anyone far away, but not someone close by.
Each enhancement added increases the spell's potency, at the potential cost of unwanted complications.

On top of the basic five low Spheres, casters can employ high Spheres by combining two of the Low Spheres together.
Air and Fate create Death magic, while Water and Earth create Life magic.

\subsection{Step by Step}

\begin{enumerate}
  \item
  Select an elemental Sphere.
  \begin{itemize}
    \item
    Each of the Low Spheres can combine with two others to make a High Sphere, equal to the lowest of the two elements creating it.
    \item
    For example, a witch with Fire 1 and with Air 2 could cast \textit{Light} magic at level 1.
  \end{itemize}
  \item
  Select one of the four Weaves:
  \begin{description}
    \item[Waxing]
    spells encourage the element to express its basic nature.
    Stone hardens, water rushes, et c.
    \item[Waning]
    magic does the opposite -- it reduces the element's nature, and often destroys the target.
    Water evaporates, and Light turns to darkness.
    \item[Warping]
    an element alters some fundamental aspect, promoting strange behaviour and effects from the target.
    \item[Witnessing]
    means to find out whether or not the element exists somewhere.
    They always come in the form of `yes/ no' answers.
  \end{description}
  \item
  Spend \glspl{mp} to power your spell, up to your level in the Sphere.
  \begin{itemize}
    \item
    The spell's level equals the \gls{mp} cost.
    \item
    If you have 0 \glspl{mp}, each extra levels inflicts \pgls{fatigue}.
  \end{itemize}
  \item
  For each \gls{mp} spent beyond the first, you \emph{must} select another Enhancement.
  \begin{description}
    \item[Detailed]
    spells let you control the appearance and minute properties of a spell, or let you specify details of an element with \textit{Witness} spells. 
    \item[Distant]
    spells throw your magic far into the distance, and no less.
    These spells cannot target shorter distances.

    \begin{boxtable}[cLL]
      \textbf{Level} & Standard Distance & Enhanced Distance       \\
      \hline
                  1              & 20 \glspl{step} & ---               \\
                  2              & 15 \glspl{step} & throwing distance \\
                  3              & 10 \glspl{step} & shouting distance \\
                  4              &  5 \glspl{step} & horizon           \\
                  5              &  ---            & line of sight     \\
    \end{boxtable}
    \item[Divergent]
    spells produce two effects from opposing Spheres, at the same time.
    \begin{itemize}
      \item
      A \textit{Divergent Wane Fire, Fate} spell would make a torch dim to nothing, while another target loses \glspl{fp}.
      \item
      While a \textit{Divergent, Duplicated, Wane Mind} spell would create a lot of shadow, while making a lot of people feel confused.
    Check the Speltogram \vpageref{speltogram} to see all opposing Spheres.
      \item
      Any Spheres which do not neighbour each other, oppose each other.
      \item
      As usual, all Spheres used must be equal in level.
    \end{itemize}
    \item[Duplicated]
    spells fork like lightning, affecting every available target nearby.
    The caster does not select the remaining targets.
    The total number of targets equals $L^L$, where $L$ is the spell's level.
    \begin{itemize}
      \item
      Fire spells target all nearby fires.
      \item
      Earth spells usually target a large patch of rock or ice, as all the substance around makes for other viable targets.
      \item
      \begin{boxtable}[YY]
        \textbf{Level} & Targets \\
        \hline
                    1              & 1 \\
                    2              & 4 \\
                    3              & 9 \\
                    4              & 256 \\
                    5              & 3125 \\
      \end{boxtable}
    \end{itemize}
  \end{description}
  \item
  Roll to resolve, with \roll{Charisma}{Sphere}, while stating your intention!
  \begin{itemize}
    \item
    The element's resistance may raise or lower the \gls{tn}.
    \item
    Breaking hard earth is difficult, but snow is easy.
    Blasting wind in a house is difficult, but redirecting a storm is not.
    \item
    If the effects target a person, then the person can \emph{also} resist the effects with any appropriate combination of Traits.
  \end{itemize}
  \item
  Apply effects.
  \label{sphereEffects}
  \begin{itemize}
    \item
    Low Spheres create mechanical effects equal to their level +2.
    \item
    High Spheres create a mechanical effect equal to their level +1.
  \end{itemize}
\end{enumerate}

\subsection{Principles of Magic}

Once you read enough spells, they should start to feel repetitive.
The steps above were used to create each one, so once the spells in the Core Rules start to feel familiar, making spells on-the-fly should become quickly intuitive.

On top of those steps, keep the general principles below in mind when crafting new spells.

\subsubsection{Plan or Die}

Sooner or later someone will fail to see a spell's natural consequences.
Someone may cast a spell to make their enemies shrink and become weak, then notice the spell's range is `to the horizon', which means the spell cannot actually target those enemies.
Or perhaps a spell intended to kill all seven bandits has twenty-four targets, and therefore targets the bandits, all of the \glspl{pc}, every ally, and the \gls{doula}'s pet magpie.
The system intends for players and the \gls{gm} to make these mistakes.

Magic creates unintended side-effects, but they don't fall out of random rolls and a `random magic effect' table.
They come from the \ref{sphereEffects} principles above.

\subsubsection{Spells Last Until Death}

Spells have no set `duration'.
Fires burn until their fuel runs out.
The undead burn through the spell which animates them every time they move.
Life spells which bolster a target's Bonuses often demand the target burn through calories in order to keep the spell going, and then fade to nothing once the spell ends.

The first principle here is that if something seems like it could stop the effects of a spell, it stops the spell.
The second principle is that spells can last forever, but they do not \emph{produce} anything forever.

\subsubsection{Spells Stack as Usual}

Stacking spell effects should work exactly like any other Stacking.%
\exRef{core}{Core Rules}{stacking}
Casting two Life spells, which both inflict a -2 Penalty to someone's Speed results in a -2~Penalty, because these spells do the same thing.
However, two different Spheres which create a Speed Penalty would Stack as usual.

So spell Stacking works exactly like armour -- wearing chainmail on top of plate will not work well, but a character could gain \gls{dr} through a touch hide, and then add armour to gain half the \gls{dr} for the lower of the two protection-types.

\end{multicols}

{
  \centering
  \speltogram
  \label{speltogram}
}

\section{\Glsfmttext{ingredient} Workshop}

\begin{multicols}{2}

\noindent
`\Glspl{ingredient}' means anything which can help with arcane practices, and each \gls{ingredient} aligns to one of the Low Spheres.
Most \glspl{ingredient} come from the bodies of beasts, such as a basilisk's gullet, or griffin feathers.

\subsubsection{\Glsfmttext{boon} Creation}

Casters can add a little extra power to their spells by taking raw \glspl{ingredient}, and crafting them into \glspl{boon}.
\Glspl{boon} come in the form of powders or liquids, to be thrown in the air at the point of casting, and add +1 to any Sphere.

\begin{itemize}
  \item
  \Glspl{boon} demand at least 1 \gls{ap} to use from the caster's hand, as usual.
  \item
  \Glspl{boon} remain suffused in the air until someone (anyone) consumes the energy by casting a spell.
  \item
  \Glspl{boon} can increase multiple Spheres by combining multiple \glspl{ingredient}, but only by 1 level each.
\end{itemize}

\noindent
With the help of a Fire \gls{boon}, someone without any magical ability could cast a level 1 Fire spell, or a caster with Fire 2, Earth 3 could cast a level 3 Light spell.

\subsubsection{\Glsfmttext{talisman} Creation}

With the right \glspl{ingredient}, one can create \pgls{talisman}, which stores a spell, and releases it once the conditions have been met.

\Glspl{talisman} require \glspl{ingredient}, rather than magical Spheres to cast.
Anyone can make one, with the right knowledge, and the ability to coax a little life into the material world with the right words.

By default, \pgls{talisman} will target the nearest thing it can.
However, many casters add a \textit{Witness} spell to the \gls{talisman}, to let it target something else.

\begin{enumerate}
  \item
  The creator starts by arranging their \glspl{ingredient} -- these will give effective access to spell Spheres.
  Casting with Fire 2, Earth 2 would require 2 Earth \glspl{ingredient} and 2 Fire \glspl{ingredient}.
  \item
  The talisman itself must be comprised of materials which align to the elemental Spheres of two spells in use.
  \begin{description}
    \item[Air]
    uses contained gasses, often held in a phial.
    \item[Earth]
    requires any solid materials dug from the ground.
    \item[Fate]
    talismans require the remains of someone who had \glspl{fp} in life (at least 1 per \gls{ingredient} used in the spell).
    \item[Fire]
    requires flammable material.
    \item[Water]
    can use any liquid.
  \end{description}
  \item
  Next, the \gls{talisman} needs a spell to cast, once it detects something.
  Fire spells will, by default, target the nearest fire they can; Air spells will almost certainly target surrounding air.
  \begin{itemize}
    \item
    The \gls{talisman} casts its spell with a bonus equal to the spell's level, so a level 3 Air spell with a cost of 3 \glspl{mp}, made into \pgls{talisman}, casts its spell with a +3 bonus, rather than the normal roll of \roll{Charisma}{Air}.
    \item
    The creator may include a \textit{Witness} spell which the \gls{talisman} will use to detect the target.
    It will then target the nearest thing the \textit{Witness} spell finds.
    \item
    This \textit{Witness} spell requires further \glspl{ingredient}, but it does not require another roll to construct.
  \end{itemize}
  \item
  The creator then decides upon the talisman's activation conditions -- will it activate once it sees starlight?
  Or awaken to any loud noises?
  Or activate when someone solves a riddle, engraved on its surface?
  \item
  The alchemical process requires an \roll{Charisma}{Academics} check, \tn[10] plus double the spell's cost.
\end{enumerate}

\subsubsection*{Examples}

\begin{exampletext}
Courtblight wants to make \pgls{talisman} to help her companions flee when they're in trouble.

\begin{itemize}
  \item
  She can't cast Force spells herself, but she has two Earth \glspl{ingredient} and two Fire \glspl{ingredient}, so that can make \pgls{talisman} with a level 2 \textit{Wax Duplicate Force} spell.
  \item
  She takes the \glspl{ingredient} (two woodspy eyes, and a lot of basilisk hide) and boils both for several days, reducing them to an inky paste.
  \item
  She takes her materials (paper for the Fire, crushed amethyst for Earth) and works the dust into the paper, then writes across it with the inky paste from her pot.
  \item
  Finally, she whispers the activation words, `let me fly', into the scroll, over and over, with some doodles of griffins around the margins.

  With this, the player rolls \roll{Charisma}{Academics} at \tn[14].
\end{itemize}

Once done, she casts a spell to detect the mana locked in the scroll, and finds nothing\ldots the ritual awoke nothing, the item will remain fruitless, and the \glspl{ingredient} wasted.

She decides that next time, she will ask for help creating the item.

\end{exampletext}

\end{multicols}

