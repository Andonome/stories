\begin{multicols}{2}

\subsection*{Overview}

The \textit{Book of Stories} covers everything BIND players need to build a \glsentrytext{pc}, and all the rules of more interest to the players than the \glsentrytext{gm}.

\begin{description}
  \item[\nameref{randomCharacterCreation}]
  lets you build a character with seven rolls of $2D6$.
  \item[\nameref{playerchosen}]
  details how to make your own character, with decisions at every stage.
  \item[\nameref{listOfCodes}] introduces the Codes of Belief that players can use to gain additional \glspl{xp}.
  \item[\nameref{listOfStories}] shows how to spend \glspl{storypoint} to flesh out a character's backstory, and introduce allies.
  You can create these allies just like your \gls{pc}, and when your \gls{pc} dies, you can select a new one from this pool of known accomplices.
  \item[\nameref{races}] details the cultural norms of the peoples of \gls{fenestra}.
\end{description}

\columnbreak

\subsection*{Special Thanks \ldots}

\paragraph{To Matija}
for editing out my brain-fog.

\subsubsection*{to the Artists}

\paragraph{Roch Hercka} for the myriad wonderful pencil sketches, pages 
\pageref{Roch_Hercka/five_races}, 
\pageref{Roch_Hercka/xp-1}, 
\pageref{Roch_Hercka/xp-2}, 

Find him at artstation.com/hertz.

\paragraph{Studio DA}
for the elf stalker image
(page \pageref{Studio_DA/elf_stalker}).

\subsection*{Licence}

BIND is open source, and available under the {\tt GNU General Public License 3} or (at your option) any later version.

You have full access to all the source files, including art, and the right to change anything and share those changes with others.
BIND will never have any `house rules', because anyone can place their alterations directly into the book and make their rules official.

\end{multicols}
