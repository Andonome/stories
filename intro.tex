\begin{multicols}{2}

\subsection*{A World of Predators}

You've probably played games where your character dies, then you try again, with the same scenario with the same character.
But in BIND, every death means you start the next scenario with a new character.

Your first character will be entirely random -- one of the many lowlifes, cut-throats or political agitators the \glspl{warden} sentence to guard the \gls{edge} of civilization against the ever-curious creatures of the forest.
But even low-life nobodies come from somewhere, so you can spend up to 5 \glspl{storypoint} to introduce friends, family, and allies.

You can roll these allies randomly, or spend \glspl{xp} to select their Traits, then have them join for a single session.
Each ally which survives enters your \gls{characterPool} at the end of the session.

Besides expanding your \gls{characterPool} with allies, you can spend \pgls{storypoint} to explain how you know a language, or some special skill, or even ask the \gls{gm} if your character might know a nearby safe-house to hide in when things get dicey.
Each little explanation adds to a growing, in-world tale.

Once your first character dies, write down the cause of death, and select another sheet from your \gls{characterPool}.
Each god governs a different type of death, so the cause of death determines their afterlife.

Characters don't begin with excellent weapons or supplies.
And the creatures which wander across \gls{fenestra} don't understand what `level-appropriate' means, they just want to eat people.
\Gls{fenestra} will always remain larger than any of your characters, but with a little cunning, you can keep them out of the gods' hands for too long.

Can't enter town due to your rank?
Stand outside, and ask traders to purchase something for you.
Can't fight properly with the dagger?
Use a shovel, or make a club from a tree-branch and rusty nails!
Too many beasts about to stay safe?
Buy some pitch, and start burning trees!

So take care, and don't play fair.

\columnbreak

\subsection*{Overview}

The \textit{Book of Stories} covers everything BIND players need to build a \glsentrytext{pc}, and all the rules of more interest to the players than the \glsentrytext{gm}.
For the complete rules, find a copy of BIND \textit{Core}.

\begin{description}
  \item[\nameref{randomCharacterCreation}]
  lets you build a random character with seven rolls of $2D6$.
  \item[\nameref{playerchosen}]
  details how to craft allies, with decisions at every stage.
  \item[\nameref{listOfStories}] shows how to spend \glspl{storypoint} to flesh out a character's backstory, then introduce allies.
  You can create these allies just like your \gls{pc}, and when your \gls{pc} dies, select a new one from your pool of allies.
  \item[\nameref{listOfCodes}] introduces the Codes of Belief that players can use to gain additional \glspl{xp}.
  \item[\nameref{races}] details the cultural norms of the peoples of \gls{fenestra}.
\end{description}

\subsection*{Special Thanks \ldots}

\paragraph{To Matija}
for developing the spell-casting system with me, and editing out my brain-fog.

\subsubsection*{to the Artists}

\paragraph{Roch Hercka} for the myriad wonderful pencil sketches, pages 
\pageref{Roch_Hercka/five_races}, 
\pageref{Roch_Hercka/xp-1}, 
\pageref{Roch_Hercka/xp-2}, 

Find him at artstation.com/hertz.

\paragraph{Studio DA}
for the elf stalker image
(page \pageref{Studio_DA/elf_stalker}).

\paragraph{Leonard}
for the `Next Day' image (\autopageref{Leonard/next_day}).

\subsection*{Licence}

BIND is open source, and available under the {\tt GNU General Public License 3} or (at your option) any later version.

You have full access to all the source files, including art, and the right to change anything and share those changes with others.
BIND will never have any `house rules', because anyone can place their alterations directly into the book and make their rules official.

\end{multicols}
