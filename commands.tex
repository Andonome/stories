\makeindex[name=spells,title={Spell Summaries},columns=2]

\newcommand\namesOfHumans{
  \begin{boxtable}[c|YY]
  \Hu & \textbf{Prefix} & \textbf{Suffix} \\\hline
  \dicef{1} & \humanNamePrefix & --\humanNameSuffix \\
  \dicef{2} & \humanNamePrefix & --\humanNameSuffix \\
  \dicef{3} & \humanNamePrefix & --\humanNameSuffix \\
  \dicef{4} & \humanNamePrefix & --\humanNameSuffix \\
  \dicef{5} & \humanNamePrefix & --\humanNameSuffix \\
  \dicef{6} & \humanNamePrefix & --\humanNameSuffix \\
  \end{boxtable}
}

\newcommand\raceChart{
  \mapPic[\Large]{b}{Roch_Hercka/five_races}{
    {\Huge\Gn}/07/83,
    2/11/82,
    {$-2$ Strength}/02/24,
    {$+1$ Dexterity}/02/16,
    {$-1$ Speed}/02/08,
    {$+1$ Intelligence}/02/0,
    {\Huge\Dw}/24/87,
    3/28/86,
    {$+1$ Dexterity}/25/24,
    {$-1$ Speed}/25/16,
    {\Huge\Hu}/42/99,
    4-10/47/99,
    {$+1$ Strength}/49/24,
    {$-1$ Wits}/49/16,
    {\Huge\El}/60/95,
    11/64/94,
    {$-1$ Strength}/65/24,
    {$+1$ Wits}/65/16,
    {\Huge\Nl}/77/99,
    12/81/99,
    {$+1$ Strength}/82/24,
    {$+1$ Speed}/82/16,
    {$-1$ Intelligence}/82/08,
    {$-2$ Charisma}/82/00,
  }
}

\newcommand\raceAbilitiesChart{
  \begin{boxtable}[lL]
    Dwarves & \nameref{dwarvenInheritance}, \autopageref{dwarvenInheritance} \\
    Elves & \nameref{elvenInheritance}, \autopageref{elvenInheritance} \\
    Gnolls & \nameref{gnollishInheritance}, \autopageref{gnollishInheritance} \\
    Gnomes & \nameref{gnomishInheritance}, \autopageref{gnomishInheritance} \\
    Humans & \nameref{humanInheritance}, \autopageref{humanInheritance} \\
  \end{boxtable}
}

\newcommand\raceAgeChart{
  \begin{boxtable}[lL]
    \textbf{Race} & \textbf{Age} \\
    \hline
    Dwarves & $30 + 2\times$ positive \Glsfmtplural{attribute} \\
    Elves & $60 + 4\times$ positive \Glsfmtplural{attribute} \\
    Gnolls & $10 + 2\times$ positive \Glsfmtplural{attribute} \\
    Gnomes & $30 + 4\times$ positive \Glsfmtplural{attribute} \\
    Humans & $15 + 4\times$ positive \Glsfmtplural{attribute} \\
  \end{boxtable}
}

\newcommand\poshNames{
  \begin{nametable}[c|Y|Y]{Aristocratic Names}
    \textbf{Roll} & \M & \F \\\hline
    \ifcase\value{r3}\relax
    \or%
      \dicef{1} & Cuffmain   & Anodyne     \\
      \dicef{2} & Narcos     & Detritus    \\
      \dicef{3} & Lambatives & Soporific   \\
      \dicef{4} & Lincture   & Throng      \\
      \dicef{5} & Dispen     & Disring     \\
      \dicef{6} & Acrostic   & Endema      \\
    \or%
      \dicef{1} & Grawlix & Senofage \\
      \dicef{2} & Topiary & Cuspadore \\
      \dicef{3} & Coreolis & Carnyx \\
      \dicef{4} & Caligin & Maledict \\
      \dicef{5} & Artix & Unctious \\
      \dicef{6} & Elegiac & Carntol \\
    \else
      \dicef{1} & Fractious & Ethel \\
      \dicef{2} & Calyx & Fextol \\
      \dicef{3} & Seance & Tepal \\
      \dicef{4} & Voynich & Sefal \\
      \dicef{5} & Aspid & Suetin \\
      \dicef{6} & Perspic & Auspic \\
    \fi
  \end{nametable}
}

% for CS callout boxes

\newcommand\commentary[3]{
  \ifstrempty{#3}{}{\stepcounter{track}}
  \node[text=white, overlay, rectangle callout, callout relative pointer={(#2)}, fill=black,] (randomNode) at (#1) [text width=4cm]{#3};
}

\newcommand\showConcept[6]{
  \filbreak
  \setcounter{knacks}{0}
  \randomize
  \item
  \textbf{#1:} % Attribute
  \underline{%
  \textit{#2!}% Concept
  }
  #3% Description
  \begin{itemize}
    \item[\textbf{Code:}] #4

    \item[\textbf{Skills:}] #5

    \item[\textbf{Stuff:}] Begin play with #6

    \bigLine
  \end{itemize}
  \filbreak
}

\newcommand{\racechart}{

\begin{nametable}[clX]{Race}
  
  \textbf{Roll} & \textbf{Race} & \textbf{Adjustments} \\\hline

  2-3 & Gnome & \mbox{+1 Intelligence, +1 Dexterity,} \mbox{Strength -2, Speed -1} \\

  4-5 & Dwarf & +1 Dexterity, -1 Speed \\

  6-8 & Human & +1 Strength, -1 Wits \\

  9-10 & Elf & +1 Wits, -1 Strength \\

  11-12 & Gnoll & \mbox{+1 Strength, +1 Speed,} \mbox{-1 Intelligence, -2 Charisma} \\

\end{nametable}
}

\newcommand\attributeChart{
  \begin{boxtable}[YYX]

    \textbf{Result} & \textbf{Bonus} & \textbf{Meaning} \\
    \hline

    2 & -3 & Pathetic \\

    3 & -2 & Useless \\

    4-5 & -1 & Weak \\

    6-8 & 0 & Normal \\

    9-10 & +1 & Exceptional \\

    11 & +2 & Excellent \\

    12 & +3 & Peak \\

  \end{boxtable}
}

\newcommand\XPchart{
  \begin{wideTable}[XYYYY]{\Glsfmttext{xp} Costs}
    \textbf{Level} &  \textbf{Attributes}  &  \textbf{Skill}  & \textbf{Combat Skills}  & \textbf{Knacks} \\
    \hline
    <1             &         5             &      ---         &       ---                &        ---       \\
    1st            &         10            &       5          &        10                &         5        \\
    2nd            &         20            &       10         &        20                &        10        \\
    3rd            &         40            &       15         &        30                &        15        \\
    \hline
    \textit{Total} &         75            &       30         &        60                &        30   \\

  \end{wideTable}
}


\newcommand\exampleRandomCharacter{
  \renewcommand\csComments{
      \setcounter{track}{0}
      \label{exampleRandomCharacter}

      \commentary{[xshift=-4em,yshift=-4em]TCBPOSTER@title.east}{-16em,3em}{{\LARGE \ref{ccRaceRoll}: }Roll a random race (\vpageref{raceRoll}) and a name (\vpageref{randomNames}).}
      \commentary{[xshift=3em,yshift=-1em]TCBPOSTER@attributes.north}{0em,0em}{{\LARGE \ref{ccAttributeRoll}: }Roll to determine each of your \glsfmtplural{attribute} \vpageref{randomAttributes}, applying modifiers for race.}
      \commentary{[xshift=8em,yshift=-6em]TCBPOSTER@Derstats.east}{0em,6em}{}
      \commentary{[xshift=8em,yshift=-6em]TCBPOSTER@Derstats.east}{-19em,-2em}{{\LARGE \ref{ccConcept}: }Your lowest and highest Attributes determine your Concept and Code (see \nameref{enlistment}, \vpageref{enlistment}).  Write down the Skills, equipment, and weapons.}
      \commentary{[xshift=3em,yshift=0em]TCBPOSTER@knacks.east}{-18em,28em}{}
      \commentary{[xshift=3em,yshift=0em]TCBPOSTER@knacks.east}{-10em,12em}{{\LARGE \ref{ccDerived}: }Place coins in the circles to keep track of the current scores, then fill in the derived score for combat.}
  }

  \begin{filledCS}%
    {Proskuff}% NAME
    {Human}% RACE
    {Loner}% CONCEPT
    {Conqueror}% CODE
    {{0}{-1}{0}}% BODY
    {{1}{0}{1}}% MIND
    {%
      \renewcommand\characterDebt{100~\glsfmtplural{sp}}
      \renewcommand\rank{Fodder}
      \setcounter{Academics}{2}
      \setcounter{Cultivation}{1}
      \setcounter{Air}{2}
      \shortsword
    }% SKILLS
    {\specialist{locks}}% KNACKS
    {a bag of flour, chalk, and an unopened letter from home.}% EQUIPMENT

  \end{filledCS}

  \renewcommand\csComments{}

}

\newlength{\magicCircle}
\setlength{\magicCircle}{.32\textwidth}
\newlength{\sphereBack}
\setlength{\sphereBack}{18pt}

\newcommand\speltogram{
  \setcounter{track}{161}
  \begin{tikzpicture}[very thick]

    \foreach \x in {1,...,5} {
      \node (\x) at (\arabic{track}:\magicCircle) {};
      \addtocounter{track}{72}
    }

    \foreach \x in {1,...,5} {
      \setcounter{enc}{\x}
      \addtocounter{enc}{2}
      \ifnum\value{enc}>5
        \addtocounter{enc}{-5}
      \fi
      \draw[fill=\pageSideColor] (\x) -- (\arabic{enc});
    }

    \foreach \x in {1,...,5} {
      \draw[fill=\pageSideColor] (\x) circle(\sphereBack){};
      \addtocounter{track}{72}
    }

    \addtocounter{track}{36}
    \foreach \x in {6,...,10} {
      \node (\x) at (\arabic{track}:.38\magicCircle) {};
      \draw[fill=\pageSideColor] (\x) circle(\sphereBack){};
      \addtocounter{track}{72}
    }

    \node (X) at  (6) {\Large\outline{\titleFont{Earth}}};
    \node (X) at  (7) {\Large\outline{\titleFont{Water}}};
    \node (X) at  (8) {\Large\outline{\titleFont{Fate}}};
    \node (X) at  (9) {\Large\outline{\titleFont{Air}}};
    \node (X) at (10) {\Large\outline{\titleFont{Fire}}};
    \node (X) at  (1) {\huge\outline{\titleFont{Force}}};
    \node (X) at  (2) {\huge\outline{\titleFont{Life}}};
    \node (X) at  (3) {\huge\outline{\titleFont{Mind}}};
    \node (X) at  (4) {\huge\outline{\titleFont{Death}}};
    \node (X) at  (5) {\huge\outline{\titleFont{Light}}};
  \end{tikzpicture}
}
