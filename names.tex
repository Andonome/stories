\subsection{Names}

Roll a die twice -- once for the prefix, and once for the suffix.

\begin{multicols}{2}
\begin{nametable}[c|lY]{Gnollish Names}
\textbf{Roll} & \textbf{Prefix} & \textbf{Suffix} \\\hline
1  & Ksha & --dz  \\
2  & Ko   & --g   \\
3  & Sya  & --h   \\
4  & Tso  & --d   \\
5  & Yo   & --sh  \\
6  & Riye & --tse \\
\end{nametable}

\columnbreak

Gnoll names are short, to-the-point, and never require difficult lip-movements (assuming you have canine lips).
The meanings generally relate to the gnoll's primary joy, such as `hunting', or `biscuits'.

\end{multicols}

\begin{nametable}[l|lYY]{Elven Names}
  & \textbf{Prefix} & \textbf{Suffix (F)}   & \textbf{Suffix (M)} \\\hline
1 & Sind    & --\"e    & --on      \\
2 & Atar    & --ink\"e & --inkon   \\
3 & Ciry    & --inw\"e & --iel     \\
4 & Tarin   & --\'ote  & --or      \\
5 & Fin     & --uin    & --acil    \\
\ifodd\value{r3}
6 & It\'ar    & --w\"e   & --il      \\
\else
6 & Itar    & --il     & --ill\"e  \\
\fi
\end{nametable}

Elven names represent long stretches of their lives -- generally as long as the language survives.

Roll once for the prefix, and again for a female or male suffix.

The `\"e' symbol means the sound must be pronounced fully, as in `f\textbf{ai}rie', or `sel\textbf{e}ct'.
The stress goes on the last syllable of the prefix.

\begin{nametable}[c|YY]{Human Names}
\textbf{Roll} & \textbf{Prefix} & \textbf{Suffix} \\\hline
\ifodd\value{r4}
1 & Lex    & --ograf \\
\else
1 & Gla    & --dun   \\
\fi
2 & Steer  & --kuff  \\
\ifodd\value{r3}
3 & Choir  & --nail  \\
4 & Flick  & --bor   \\
\else
3 & Pros   & --flay  \\
4 & Cart   & --pike  \\
\fi
\ifodd\value{page}
5 & Gors  & --meen  \\
\else
5 & Keel   & --frak  \\
\fi
6 & Moc    & --drag  \\
\end{nametable}

Human names remain static throughout their lives, so they never have any relation to the person, or their accomplishments.

\begin{multicols}{2}

\begin{nametable}[l|Y]{Gnomish Names}
1  & ni    \\
2  & lawa  \\
3  & noka  \\
4  & en    \\
5  & ante  \\
6  & alasa \\
8  & yan   \\
9  & mu    \\
10 & kala  \\
12 & yelo  \\
15 & musi  \\
16 & ma    \\
18 & leta  \\
20 & nanpa \\
24 & mute  \\
25 & wan   \\
30 & open  \\
36 & tu    \\

\end{nametable}

Roll $1D6 \times 1D6$, and re-roll on doubles to add another part to the name.

Gnomes receive names from everyone around them, so their name depends on context.
Reflexively, this therefore means they give others a name rather than asking.
However, if the gnome feels generous, and does not want to trouble any of the `big folk' with the task of creating two or three syllables, they may provide a name.

\end{multicols}

\begin{nametable}[l|XYY]{Dwarven Names}

\textbf{Roll} & \textbf{Prefix} & \textbf{Suffix (M)} & \textbf{Suffix (F)} \\\hline
1 & Th   & --alin   & --oshell    \\
2 & D    & --urg    & --ragsi     \\
3 & M    & --eel    & --well      \\
4 & G    & --imlen  & --itten     \\
5 & B    & --enlak  & --lot       \\
6 & K    & --rindal & --lasi      \\

\end{nametable}

Dwarven names show to their position in society, and generally relate to their mother's occupation.

\end{multicols}
