\section{Self Made Characters}

\begin{multicols}{2}

\label{playerchosen}

If players prefer, they can design their own characters by simply setting all Attributes to 0, then applying the racial modifiers.
They can choose to take a single -1 penalty to any Attribute of their choice in return for an additional 5 \gls{xp}.

\begin{enumerate}
  \item\label{sumCCrace}
  Select a race.
  \item\label{sumCCatt}
  Set each of your Attributes to 0, then apply the racial bonuses and penalties (\autopageref{raceRoll}).
  \item\label{sumCCconcept}
  Write down a concept and background culture from your campaign.
  \item\label{sumCCcode}
  Select a Code to follow, so you can gain \glspl{xp} (\autopageref{codes}).
  \item\label{sumCCxp}
  Spend 50 \gls{xp} on Attributes, Skills, and Knacks\iftoggle{core}{ (Core rules, \autopageref{knacks}.)}{}.
  \item\label{sumCCequip}
  Take 1 item per Skill level your character has from the list of starting equipment.
  \item\label{sumCCcoin}
  Starting money is $(3D6-5)\times 2^S$\gls{cp}, where S = combined levels in all Skills.
  \item
  Fill in the derived stats.
  \begin{itemize}
    \item
    \glspl{hp} are equal to 6 plus your Strength.
    \item
    \glspl{fp} are equal to total \glspl{xp} divided by 10 (rounded up), plus Charisma.
    \item
    \glspl{mp} are equal to the number of spheres you have times 3, plus your Wits Bonus.

  \end{itemize}
\end{enumerate}

\subsection{Concept}
\index{Character Concept}

Have a look at your character's Attributes and consider what kind of person they are.
Strength might indicate working on a farm.
A good Charisma may indicate a creative past, such as poetry, or working with the public, trading or selling items.
Intelligent characters may have required to plan a lot -- perhaps working as a seneschal for a town master, or planning a safe route for a travelling circus.
Low intelligence means never having to think, so they may come from a noble family, or always worked in a stonemason's guild, where someone else could plan their day for them.
Dexterous characters could have been an expert weaver, before joining the \gls{guard}; or perhaps they lived in a major city and stole money from others.

How did you end up in the \gls{guard}?
Did you naively sign up for a thrilling adventure, and now regret it?
What happened to your farm?
Torched by goblins, or did too many men leave the village?
Most join simply because they did not inherit their parents' land, so they hope to make some money, and perhaps be rewarded with land.
Some stole or robbed from people, and managed to plead to a judge to let them join the \gls{guard} instead of hanging.

Take your character's history, and condense it into two words.

\begin{multicols}{2}
\begin{itemize}
  \item
  Betrayed Guildsman
  \item
  Clairvoyant Herder
  \item
  Crypto Zoologist
  \item
  Dispossessed Farmer
  \item
  Dishonoured \gls{miracleworker}
  \item
  Dauntless Youth
  \item
  Failed Priest
  \item
  Lazy Thief
  \item
  Lost Writer
  \item
  Reformed Bandit
  \item
  Zealous Guildsman
\end{itemize}
\end{multicols}

\subsection{Experience}

\begin{itemize}
  \item
  Increasing a negative Attribute costs 5 \glspl{xp}.
  After that, the price increases sharply.
  \item
  Each Knack costs 5 \glspl{xp} more than the last.
  \item
  Projectiles, Combat, Brawl, and all magic spheres are covered under `Martial Skills'.
\end{itemize}

\XPchart

\subsection{Starting Equipment}
\label{start_equipment}

\index{Starting Equipment}
\label{adventuringequipment}
\index{Adventuring Equipment}

Characters each start with one items per Skill level, and each item can be worth 10 \glspl{sp} or less.
This might include a sword, dagger, a donkey, or anything else worth 10 \gls{sp} or less.
Have a look at the goods available in \autoref{coreRules}: \nameref{coreRules}, \autoref{goods}.
If your character has Combat 1, and Caving 2, they can select 3 items.

The player can decide to replace any of these items with a generic item called mission equipment, and decide exactly what it is later in the game.
Mission equipment cannot be given away or acted upon in any way without stipulating exactly what it is.

Mission equipment can include any of the following items:

\begin{multicols}{2}
\begin{itemize}

\item{Chalk}
\item{Flour}
\item{Lock picking set}
\item{Medical equipment}
\item{Mirror}
\item{Rations for a day}
\index{Food}\index{Rations}
\item{Rope}
\item{Tinder box}
\item{Torch}
\item{Wine}
\item{Writing equipment}

\end{itemize}
\end{multicols}

\subsubsection{Starting Money}

The amount of bare money a character starts out with depends upon their Skills.
Starting money is $3D6-5$ \glspl{cp}, which doubles for every level the characters has in a Skill.

For example, a character with Empathy 1 and Tactics 1 might roll a total of 7.
$7\times2\times2 = 28$, so the character starts out with 28 \glspl{cp}.

\end{multicols}

\renewcommand\csComments{
    \commentary{[xshift=22,yshift=-9em]TCBPOSTER@title.north}{-2em,2em}{{\huge\ref{sumCCrace}} Select a race}

    \commentary{[xshift=4em,yshift=-6em]TCBPOSTER@attributes.south east}{-4.1em,2.8em}{{\huge\ref{sumCCatt}:} Fill in Attributes.}

    \commentary{[xshift=13em,yshift=-6em]TCBPOSTER@title.west}{-2em,2em}{{\huge\ref{sumCCconcept}:} Write a concept and name.}

    \commentary{[xshift=4em,yshift=-1em]TCBPOSTER@gumption.west}{-20em,9em}{}
    \commentary{[xshift=4em,yshift=0em]TCBPOSTER@gumption.west}{1em,-28em}{}
    \commentary{[xshift=4em,yshift=0em]TCBPOSTER@gumption.west}{-4em,6.5em}{{\huge\ref{sumCCxp}:} Spend 50 \glspl{xp} to purchase Attributes, Skills, and Knacks.}

    \commentary{[xshift=-7em,yshift=-4em]TCBPOSTER@title.east}{-2em,1em}{{\huge\ref{sumCCcode}:} Select a Code (\autopageref{gods_codes}) and culture.}

    \commentary{[xshift=3em,yshift=2em]TCBPOSTER@derived.south}{3em,2em}{\small Place a coin on the \glspl{ap} tracker to keep track of your \glspl{ap} during combat.}

    \commentary{[xshift=5em,yshift=5em]TCBPOSTER@armoury.north west}{4em,-25em}{}
    \commentary{[xshift=5em,yshift=5em]TCBPOSTER@armoury.north west}{-1em,-2.5em}{{\huge\ref{sumCCequip}:} Place any weapons and armour, then fill in their stats.  The rest go in the box below.}
}


\setcounter{str}{1}
\setcounter{dex}{0}
\setcounter{spd}{0}
\setcounter{int}{0}
\setcounter{wts}{-1}
\setcounter{cha}{1}

\renewcommand\concept{Knightly Poet}
\renewcommand\race{Human}
\renewcommand\culture{Quennome}
\renewcommand\name{Sean}
\renewcommand\code{Experience}

\setcounter{Academics}{1}
\setcounter{Athletics}{0}
\setcounter{Caving}{0}
\setcounter{Crafts}{0}
\setcounter{Deceit}{0}
\setcounter{Empathy}{1}
\setcounter{Medicine}{0}
\setcounter{Performance}{1}
\setcounter{Larceny}{0}
\setcounter{Seafaring}{0}
\setcounter{Stealth}{0}
\setcounter{Tactics}{0}
\setcounter{Vigilance}{0}
\setcounter{Wyldcrafting}{0}
\setcounter{Combat}{2}
\setcounter{Projectiles}{0}

\setcounter{fp}{5}

\renewcommand\characterWeapons{\longsword}
\renewcommand\characterArmour{Chain Mail & 4 & P & 1}
\renewcommand\characterEquipment{Mission Equipment \Repeat{3}{{\Large\sqn} }}
\setcounter{gold}{288}

\settoggle{examplecharacter}{true}
\settoggle{bestiarychapter}{true}
\input{config/CS.tex}
\settoggle{bestiarychapter}{false}
\settoggle{examplecharacter}{false}
\renewcommand\csComments{}


