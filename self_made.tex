\section[Induction at the \Glsfmttext{templeOfBeasts}]{Induction at the \Glsfmttext{templeOfBeasts}~\gls{beasts}}
\label{ngIntroductions}

\begin{multicols}{2}

\noindent
Introduce your character to the group, and describe what they look like.
The \gls{guard} have no uniform exactly, but they do have a standard `look', due to commonalities:

\begin{itemize}
  \item
  Darkened leather (or a black tabard), hides them in the dark.
  \item
  A special scabbard with a `v-split' along one side, allows the \glspl{guard} to carry a sword on their back, and comfortably draw it.
  \item
  Long hair means they can pull out a strand, to be used as a suture.
  \item
  Large, black, backpack with side-pouches.
\end{itemize}

\subsubsection{Your Rank}
\label{ngRank}
\index{Rank in the \glsentrytext{guard}}
equals the highest rank of the troupe, because life's not fair.
If everyone's new, then everyone begins as `fodder'.

Welcome to the \gls{guard}!

\subsubsection{Requesting Equipment}
is your first task.
You have arrived in \pgls{broch}, ready for your first mission.
It's time to ask the \gls{jotter} if you can have one more piece of equipment from the \nameref{start_equipment}, \vpageref{start_equipment}.
Roll \roll{Charisma}{Empathy} at \tn[7].

\paragraph{If you succeed,}
think of how your character would do well in a social situation, and describe the result before writing the item on your character sheet.

\paragraph{If you fail,}
describe how your character might fumble the request.
Do they misunderstand the situation, or do they just have a bad attitude?

\paragraph{If you tie,}
think of how your character might fumble, then reverse course.
The jotter gives them the next item on the list, and explains why you should \emph{in fact} use that.

During the game, you can use a dice-roll to indicate that you have something to say.
Just roll, and use the result as a guide to your character's reaction.%
\exRef{core}{Core Rules}{rollForRoles}
Of course, the \gls{gm} may give you a rather different \gls{tn}\ldots

\subsubsection{Mission Equipment}
\label{start_equipment}
\index{Equipment, Standard}
\index{Mission Equipment}

Most \glspl{broch} hold most of these items, most of the time, but wandering monsters make supply-lines difficult to maintain.

\begin{multicols}{2}
\begin{itemize}
\raggedright
  \item
  Longsword
  \item
  Maul
  \item
  Shortsword
  \item
  Partial leather armour
  \item
  Partial chain armour
  \item
  Chalk
  \item
  Medical equipment
  \item
  Mirror
  \item
  \rations
  \item
  \rations
  \item
  \rations
  \item
  Rope
  \item
  Tinder box
  \item
  Torch
  \item
  Wine
  \item
  Writing equipment
  \item
  Bag pipes
\end{itemize}

\end{multicols}

\commonArmourChart

\commonWeaponsChart %from config/rules/charts.tex
\label{commonWeapons}
\index{Weapons Chart}

\subsubsection{The Derived Traits}
on your character sheet say what they do, so fill them in.
Remember that you begin with 50~\glspl{xp} `spent' already, which gives you 5~\glspl{fp} + Charisma Bonus.

\end{multicols}

\section{Forging Allies}

\glsresetall

\begin{multicols}{2}

\label{playerchosen}

\noindent
Over \pgls{campaign} you can add a characters to your \gls{characterPool} by spending \pgls{storypoint},
(covered in \autoref{stories})
or spend two \glspl{storypoint} and design your own character.%
\footnote{Covered \vpageref{designCharacter}.}

Of course, you can decide to use this method for your first character, but that character may not make much sense if you don't understand \gls{fenestra}.

\begin{enumerate}
  \item\label{sumCCrace}
  Select a race, and note your \nameref{racialAbility} (\vpageref{racialAbility}).
  \item\label{sumCCatt}
  Set each of your Attributes to 0, then apply the racial bonuses and penalties (\autopageref{raceRoll}).
  \item\label{sumCCconcept}
  Write down a concept.
  Is this character in the \gls{guard}, or from another temple?
  \item\label{sumCCcode}
  Select a Code to follow, so you can gain \glspl{xp} (\autopageref{codes}).
  \item\label{sumCCxp}
  Spend 50 \gls{xp} on Attributes, Skills, and Knacks (see \autopageref{xpCosts} for the costs).
  \item\label{sumCCequip}
  Take 1 item per Skill level your character has from the list of \nameref{start_equipment} (\vpageref{start_equipment}).
  \item\label{sumCCcoin}
  Starting money is $(3D6-5)\times 2^N$\gls{cp}, where N = combined levels in all Skills.
  \item\label{sumCCder}
  Fill in the derived stats, like \glspl{ap}, \glspl{hp}, and (if you have armour) \gls{dr}.
  \begin{itemize}
    \item
    \glspl{hp} are equal to 6 plus your Strength.
    \item
    \glspl{fp} are equal to total \glspl{xp} divided by 10 (rounded up), plus Charisma.
    \item
    \glspl{mp} are equal to the number of Spheres you have times 3.
  \end{itemize}
\end{enumerate}

\subsection{Concept}
\index{Concept}

\subsubsection{\Glsfmtplural{attribute}}
indicate your past.
Strength might indicate working on a farm.
A high Charisma Bonus may indicate a creative past, such as poetry, or working with the public, trading or selling items.
Intelligent characters may have required to plan a lot -- perhaps working as a seneschal for a town \gls{warden}, or planning a safe route for a travelling circus.
Low intelligence means never having to think, so they may come from \pgls{warden} family, or started as a high-ranking \gls{server}, with someone else doing the book-keeping.
Dexterous characters could have been an expert weaver, before joining the \gls{guard}; or perhaps they lived in a major city and stole money from others.

\subsubsection{Joining the \glsfmttext{guard}}
happens for all number of reasons.
Did you naively sign up for a thrilling adventure, and now regret it?
Or are you another, standard, criminal, in the only place without standards?

\subsubsection{Other Temples}
sometimes send people out beyond the \gls{edge}, but be careful with these concepts.
Such a character must have a reason to routinely join \gls{guard} missions, and listen to \glspl{jotter}.

Take your character's history, and condense it into two words.

\begin{multicols}{2}
\begin{itemize}
  \item
  Betrayed \Gls{server}
  \item
  Crypto Zoologist
  \item
  Dispossessed Farmer
  \item
  Dishonoured \Gls{doula}
  \item
  Dauntless Youth
  \item
  Failed \Gls{scribe}
  \item
  Lazy Thief
  \item
  Lost Writer
  \item
  Reformed Bandit
  \item
  \Gls{warden}'s bastard
\end{itemize}
\end{multicols}

\end{multicols}

\renewcommand\csComments{
    \commentary{[xshift=22,yshift=-9em]TCBPOSTER@title.north}{-2em,2em}{{\huge\ref{sumCCconcept}} Select a two-word concept.}

    \commentary{[xshift=3em,yshift=0em]TCBPOSTER@attributes.south east}{-3em,2.8em}{{\huge\ref{sumCCatt}:} Adjust Attributes with the racial bonuses (\autopageref{raceRoll}).}

    \commentary{[xshift=15em,yshift=-6em]TCBPOSTER@title.west}{-2em,2em}{{\huge\ref{sumCCrace}:} Write a name and race.}

    \commentary{[xshift=-4em,yshift=-1em]TCBPOSTER@armoury.north east}{-18em,10em}{}
    \commentary{[xshift=-4em,yshift=0em]TCBPOSTER@armoury.north east}{1em,-23em}{}
    \commentary{[xshift=-4em,yshift=0em]TCBPOSTER@armoury.north east}{-4em,6.5em}{{\huge\ref{sumCCxp}:} Spend 50 \glspl{xp} to purchase Attributes, Skills, and Knacks.}

    \commentary{[xshift=0em,yshift=1em]TCBPOSTER@equipment.south}{0em,0em}{{\huge\ref{sumCCcoin}:} Note starting money, if any.}

    \commentary{[xshift=-7em,yshift=-4em]TCBPOSTER@title.east}{-2em,1em}{{\huge\ref{sumCCcode}:} Select a Code (\autopageref{codes}) and culture.}

    \commentary{[xshift=3em,yshift=2em]TCBPOSTER@derived.south}{0em,0em}{{\huge\ref{sumCCder}:} Fill in the derived stats.}

    \commentary{[xshift=5em,yshift=5em]TCBPOSTER@armoury.south west}{4em,-22em}{}
    \commentary{[xshift=5em,yshift=5em]TCBPOSTER@armoury.south west}{4em,.5em}{{\huge\ref{sumCCequip}:} Take a number of items equal to your total Skills, then fill in any stats for weapons and armour (\autopageref{commonWeapons}).}
}

\input{config/reset_cs.tex}

\setcounter{Strength}{1}
\setcounter{Wits}{-1}
\setcounter{Intelligence}{1}

\renewcommand\concept{Lockpicking Proteg\'e}
\renewcommand\race{Human}
\renewcommand\rank{Guildsman}
\renewcommand\name{Keelfrak}
\renewcommand\code{Chronicler}

\setcounter{Academics}{1}
\setcounter{Athletics}{0}
\setcounter{Caving}{0}
\setcounter{Crafts}{2}
\setcounter{Deceit}{0}
\setcounter{Empathy}{0}
\setcounter{Medicine}{0}
\setcounter{Performance}{0}
\setcounter{Larceny}{1}
\setcounter{Seafaring}{0}
\setcounter{Stealth}{0}
\setcounter{Tactics}{0}
\setcounter{Vigilance}{0}
\setcounter{Wyldcrafting}{0}
\setcounter{Combat}{1}
\setcounter{Projectiles}{0}

\renewcommand\knackOne{\specialist{locks}}

\setcounter{fp}{5}

\renewcommand\characterWeapon{\longsword}
\renewcommand\characterArmour{\partialchain}
\renewcommand\characterEquipment{Lock picking set, torch, tinder box}
\setcounter{gold}{288}

\settoggle{examplecharacter}{true}
\settoggle{genExamples}{true}
\input{config/CS.tex}
\settoggle{genExamples}{false}
\settoggle{examplecharacter}{false}
\renewcommand\csComments{}

