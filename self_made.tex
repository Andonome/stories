\section[Induction at the \Glsfmttext{templeOfBeasts}]{Induction at the \Glsfmttext{templeOfBeasts}}
\label{ngIntroductions}

\begin{multicols}{2}

\noindent
Introduce your character to the group, and describe what they look like.
The \gls{guard} have no uniform exactly, but they do have a standard `look', due to commonalities:

\begin{itemize}
  \item
  Darkened leather (or a black tabard), helps hiding in the shadows on the roads (+1~Bonus).
  \item
  Standard swords come with a special scabbard with a `v-split' along one side, which allows the \glspl{guard} to carry the sword on their back, and comfortably draw it.
  \item
  Long hair (or beard) means each strand can function as a suture.
  \item
  Large, black, backpack with side-pouches.
\end{itemize}

\subsubsection{Your Rank}
\label{ngRank}
\index{Rank in the \glsentrytext{guard}}
equals the highest rank of the troupe, because life's not fair.
If everyone's new, then everyone begins with the rank `\gls{fodder}'.

Welcome to the \gls{guard}!
You owe us 100~\glspl{sp}.
You can earn these coins by felling beasts while on your missions.
Only the \gls{templeOfBeasts} has the privilege of selling the bodies of \glspl{monster}, so if you see anyone else selling \pgls{basilisk}'s hide, be sure to report them to \pgls{keeper}, and prosecute them in the local \gls{court}.

\subsubsection{Requesting Equipment}
is your first task.
You have arrived at \pgls{broch} with nothing but a standard black tabard, ready for your first mission.
It's time to ask the \gls{jotter} if you can have one more piece of equipment from the \nameref{start_equipment}, \vpageref{start_equipment}.
Roll \roll{Charisma}{Empathy} at \tn[7].

\paragraph{If you succeed,}
think of how your character would do well in a social situation, and describe the result before writing the item on your character sheet.

\paragraph{If you fail,}
describe how your character might fumble the request.
Do they misunderstand the situation, or do they just have a bad attitude?

\paragraph{If you tie,}
think of how your character might fumble, then reverse course.
The jotter gives them the next item on the list, and explains why you should \emph{in fact} use that.

During the game, you can use a dice-roll to indicate that you have something to say.
Just roll, and use the result as a guide to your character's reaction.%
\exRef{core}{Core Rules}{rollForRoles}
Of course, the \gls{gm} may give you a rather different \gls{tn}\ldots

\subsubsection{Mission Equipment}
\label{start_equipment}
\index{Equipment, Standard}
\index{Mission Equipment}
just means whatever the \glspl{broch} has available.
This one comes well-stocked, but many others lack essential supplies, so prepare well.

\begin{multicols}{2}
\begin{itemize}
\raggedright
  \item
  Longsword
  \item
  Maul
  \item
  Shortsword
  \item
  Partial leather armour
  \item
  Partial chain armour
  \item
  Chalk
  \item
  Medical equipment
  \item
  Mirror
  \item
  Salted \rations
  \item
  Spiced \rations
  \item
  Stale \rations
  \item
  Rope
  \item
  Tinder box
  \item
  Torch
  \item
  Wine
  \item
  Writing equipment
  \item
  Bag pipes
\end{itemize}

\end{multicols}

\commonArmourChart

\glsadd{weapon}
\commonWeaponsChart %from config/charts.tex
\label{commonWeapons}
\index{Weapons Chart}

\subsubsection{The Derived Traits}
on your character sheet say what they do, so fill them in.
Remember that you begin with 50~\glspl{xp} `spent' already, which gives you 5~\glspl{fp} + Charisma Bonus.

\subsubsection{The Troupe Leader}
is decided by whoever has the highest rank.
Ties are broken by a \roll{Charisma}{Melee} roll, though many of the \gls{guard} step back from the roll quickly.

On a successful mission, the lowest ranking member of the troupe get a promotion.
If everyone has equal rank, the leader gains the promotion.
And if the mission fails, the troupe's leader bears full responsibility.

Since higher-ranks bring more dangerous missions, many of the \gls{guard} try to stay as unadorned as they can.

\end{multicols}

\section{Forging Allies}
\index{Allies}

\begin{multicols}{2}

\label{playerchosen}

\noindent
Over \pgls{campaign} you can add allies to your \gls{characterPool} by spending \pgls{storypoint},
(covered in \autoref{stories})
or spend two \glspl{storypoint} and design your own character as an ally.%
\footnote{Covered \vpageref{designCharacter}.}

Of course, you can decide to use this method for your first character, but that character may not make much sense if you don't understand \gls{fenestra}.

\begin{enumerate}
  \item\label{sumCCrace}
  Select a race, and note your \nameref{racialAbility} (\vpageref{racialAbility}).
  \item\label{sumCCconcept}
  Write down a concept.
  Is this character in the \gls{guard}, or from another temple?
  \item\label{sumCCcode}
  Select a Code to follow, so you can gain \glspl{xp} (\autopageref{codes}).
  Write the Code on the back page.
  \item\label{sumCCxp}
  Spend 50 \gls{xp} on Attributes, Skills, and Knacks (see \autopageref{xpCosts} for the costs).
  \item\label{sumCCequip}
  Take 1 item per Skill level your character has from the list of \nameref{start_equipment} (\vpageref{start_equipment}).
  \item\label{sumCCcoin}
  Starting money is $(3D6-5)\times 2^N$\gls{cp}, where N = combined levels in all Skills.
  \item\label{sumCCder}
  Fill in the derived stats, like \glspl{ap}, \glspl{hp}, and (if you have armour) \gls{dr} and \gls{covering}.
  \begin{itemize}
    \item
    \Glspl{hp} are equal to 6 plus your Strength.
    \item
    \Glspl{fp} are equal to total \glspl{xp} divided by 10 (rounded up), plus Charisma.
    \item
    \Glspl{mp} are equal to the number of Spheres you have times 3.
    \item
    \Glspl{storypoint} are equal to the number of \glspl{fp} your last character had.
  \end{itemize}
  \item\label{sumCCatt}
  Apply the racial Bonuses and Penalties after spending all \glspl{xp} (\autopageref{raceRoll}).
\end{enumerate}

\columnbreak

\subsection{Concept}
\index{Concept}

\subsubsection{\Glsfmtplural{attribute}}
indicate your past.
Strength might indicate working on a farm.
A high Charisma Bonus may indicate a creative past, such as poetry, or working with the public, trading or selling items.
Intelligent characters may have required to plan a lot -- perhaps working as \pgls{seneschal} for a town \gls{warden}, or planning a safe route for a travelling circus.
Low intelligence means never having to think, so they may come from \pgls{warden} family, or started as a high-ranking \gls{server}, with someone else doing the book-keeping.
Dexterous characters could have been an expert weaver, before joining the \gls{guard}; or perhaps they lived in a major city and stole money from others.

\subsubsection{Joining the \glsfmttext{guard}}
has only three possible reasons; as a punishment for crime, as a risky but lucrative career, or madness.
Most \glspl{guard} have a little of each.
Did you naively sign up for a thrilling adventure, and now regret it?
Or are you another, standard, criminal, in the only place without standards?

\index{Debt}
All criminals start in the \gls{guard} with the rank of `\gls{fodder}', and a debt to society of 100~\glspl{sp}, payable to their \gls{templeOfBeasts}.
The \gls{guard} generally pay off this debt by selling the bodies of the beasts they kill while on the \gls{lonelyRoad}.

\subsubsection{Other Temples}
sometimes send people out beyond the \gls{edge}, but be careful with these concepts.
Such a character must have a reason to routinely join \gls{guard} missions, and listen to \glspl{jotter}.

Take your character's history, and condense it into two words.

\begin{multicols}{2}
\begin{itemize}
  \item
  Betrayed \Gls{server}
  \item
  Crypto Zoologist
  \item
  Dispossessed Farmer
  \item
  Dishonoured \Gls{doula}
  \item
  Dauntless Youth
  \item
  Failed \Gls{scribe}
  \item
  Lazy Thief
  \item
  Lost \Gls{cartographer}
  \item
  Reformed Bandit
  \item
  \Gls{warden}'s bastard
\end{itemize}
\end{multicols}

\end{multicols}

\renewcommand\csComments{
    \commentary{[xshift=7em,yshift=-1em]TCBPOSTER@title.south west}{0em,0em}{{\huge\ref{sumCCconcept}:} Select a two-word concept.}

    \commentary{[xshift=3em,yshift=0em]TCBPOSTER@attributes.south east}{-3em,2.8em}{{\huge\ref{sumCCatt}:} Adjust Attributes with the racial bonuses (\autopageref{raceRoll}).}

    \commentary{[xshift=-3em,yshift=-3em]TCBPOSTER@title.north}{0em,0em}{{\huge\ref{sumCCrace}:} Write a name and race.}

    \commentary{[xshift=0em,yshift=2em]TCBPOSTER@skills.south}{0em,0em}{{\huge\ref{sumCCxp}:} Spend 50 \glspl{xp} to purchase Attributes, Skills, and Knacks.}

    \commentary{[xshift=-6em,yshift=2em]TCBPOSTER@knacks.north east}{0em,0em}{{\huge\ref{sumCCcoin}:} Note starting money, if any.}

    \ifodd\thepage
      \commentary{[xshift=-7em,yshift=-4em]TCBPOSTER@title.east}{3em,0em}{{\huge\ref{sumCCcode}:} Select a Code and place it on the next page (\autopageref{codes}).}
    \else
      \commentary{[xshift=-4em,yshift=0em]TCBPOSTER@attributes.south}{-3em,0em}{{\huge\ref{sumCCcode}:} Select a Code and place it on the next page (\autopageref{codes}).}
    \fi

    \commentary{[xshift=-3em,yshift=0em]TCBPOSTER@Derstats.south}{0em,0em}{{\huge\ref{sumCCder}:} Fill in the derived stats.}

    \commentary{[xshift=-6em,yshift=12em]TCBPOSTER@knacks.north east}{0em,.0em}{{\huge\ref{sumCCequip}:} Take a number of items equal to your total Skills, then fill in any stats for weapons and armour (\autopageref{commonWeapons}).}
}

\begin{filledCS}%
  {Keelfrak}% NAME
  {Human}% RACE
  {Lockpicking Proteg\'e}% CONCEPT
  {Chronicler}% CODE
  {{1}{0}{0}}% BODY
  {{1}{-1}{0}}% MIND
  {%
    \renewcommand\rank{Digger}

    \setcounter{Academics}{1}
    \setcounter{Crafts}{2}
    \setcounter{Larceny}{1}
    \setcounter{Melee}{1}
    \longsword
    \partialchain
  }% SKILLS
  {\specialist{locks}}% KNACKS
  {Lock picking set, torch, tinder box}% EQUIPMENT

\end{filledCS}

\renewcommand\csComments{}

