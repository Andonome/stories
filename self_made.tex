\section{Self Made Characters}

\glsresetall

\begin{multicols}{2}

\label{playerchosen}

\noindent
If players prefer, they can design their own characters by simply setting all Attributes to 0, then applying the racial modifiers.

\begin{enumerate}
  \item\label{sumCCrace}
  Select a race.
  \item\label{sumCCatt}
  Set each of your Attributes to 0, then apply the racial bonuses and penalties (\autopageref{raceRoll}).
  \item\label{sumCCconcept}
  Write down a concept and background culture from your campaign.
  \item\label{sumCCcode}
  Select a Code to follow, so you can gain \glspl{xp} (\autopageref{codes}).
  \item\label{sumCCxp}
  Spend 50 \gls{xp} on Attributes, Skills, and Knacks (see \autopageref{xpCosts} for the costs).
  \item\label{sumCCequip}
  Take 1 item per Skill level your character has from the list of starting equipment.
  \item\label{sumCCcoin}
  Starting money is $(3D6-5)\times 2^S$\gls{cp}, where S = combined levels in all Skills.
  \item\label{sumCCder}
  Fill in the derived stats, like \gls{dr} (0 if you have no armour), \glspl{ap}, \glspl{hp}, et c.
  \begin{itemize}
    \item
    \glspl{hp} are equal to 6 plus your Strength.
    \item
    \glspl{fp} are equal to total \glspl{xp} divided by 10 (rounded up), plus Charisma.
    \item
    \glspl{mp} are equal to the number of Spheres you have times 3.
  \end{itemize}
\end{enumerate}

\subsection{Concept}
\index{Concept}

Have a look at your character's Attributes and consider what kind of person they are.
Strength might indicate working on a farm.
A good Charisma may indicate a creative past, such as poetry, or working with the public, trading or selling items.
Intelligent characters may have required to plan a lot -- perhaps working as a seneschal for a town \gls{warden}, or planning a safe route for a travelling circus.
Low intelligence means never having to think, so they may come from a noble family, or always worked in a stonemason's guild, where someone else could plan their day for them.
Dexterous characters could have been an expert weaver, before joining the \gls{guard}; or perhaps they lived in a major city and stole money from others.

How did you end up in the \gls{guard}?
Did you naively sign up for a thrilling adventure, and now regret it?
What happened to your farm?
Torched by goblins, or did too many men leave the village?
Most join simply because they did not inherit their parents' land, so they hope to make some money, and perhaps be rewarded with land.
Some stole or robbed from people, and managed to plead to a judge to let them join the \gls{guard} instead of hanging.

Take your character's history, and condense it into two words.

\begin{multicols}{2}
\begin{itemize}
  \item
  Betrayed \Gls{server}
  \item
  Crypto Zoologist
  \item
  Dispossessed Farmer
  \item
  Dishonoured \Gls{doula}
  \item
  Dauntless Youth
  \item
  Failed \Gls{scribe}
  \item
  Lazy Thief
  \item
  Lost Writer
  \item
  Reformed Bandit
  \item
  \Gls{warden}'s bastard
\end{itemize}
\end{multicols}

\subsection{Starting Equipment}
\label{start_equipment}

\index{Starting Equipment}

Characters each start with one items per Skill level.
If your character has Combat 1, and Caving 2, they can select 3 items from standard Mission Equipment:

\begin{multicols}{2}
\begin{itemize}
\raggedright

  \item
  Longsword
  \item
  Maul
  \item
  Shortsword
  \item
  Partial leather armour
  \item
  Partial chain armour
  \item
  Chalk
  \item
  Lock picking set
  \item
  Medical equipment
  \item
  Mirror
  \item
  Rations for a day
  \item
  Rope
  \item
  Tinder box
  \item
  Torch
  \item
  Wine
  \item
  Writing equipment

\end{itemize}
\end{multicols}

\subsubsection{Starting Money}

The amount of bare money a character starts out with depends upon their Skills.
Starting money is $3D6-5$ \glspl{cp}, which doubles for every level the characters has in a Skill.

For example, a character with Empathy 1 and Tactics 2 might roll a total of 7.
$7\times2\times3 = 42$, so the character starts out with 42 \glspl{cp}.

\end{multicols}

\section{Introductions}

\begin{multicols}{2}

\noindent
Introduce your character to the group, and describe what they look like.
The standard \gls{guard} uniform is more of `look' than an actual uniform, with a few common elements.

\begin{itemize}
  \item
  Darkened leather (or a black tabard), helps hide them in the dark.
  \item
  A special scabbard with a `v-split' along one side, allows the \glspl{guard} to carry, and draw, a sword on their back.
  \item
  Long hair can be used as sutures.
  \item
  Large, black, backpack with side-pouches.
\end{itemize}

\subsubsection{First Words}

\begin{exampletext}
  You have arrived in \pgls{broch}, ready for your first mission.
  It's time for introductions, and to ask the \gls{jotter} if you can have one more piece of equipment\ldots
\end{exampletext}

\noindent
Ask the \gls{jotter} for an additional piece of equipment from the \nameref{start_equipment}, \vpageref{start_equipment}.
Roll \roll{Charisma}{Empathy} (\tn[7]).

\paragraph{If you succeed,}
think of how your character would do well in a social situation, and describe the result before taking the item.

\paragraph{If you fail,}
describe how your character might fumble the request.
Do they misunderstand the situation, or do they just have a bad attitude?

\paragraph{If you tie,}
think of how your character might fumble, then reverse course.
The jotter gives them the next item on the list, and explains why you should \emph{in fact} use that.

During the game, you can use a dice-roll to indicate that you have something to say.
Just roll, and use the result as a guide to your character's reaction.%
\exRef{core}{Core Rules}{rollForRoles}
Of course, the \gls{gm} may give you a rather different \gls{tn}\ldots

\end{multicols}

\renewcommand\csComments{
    \commentary{[xshift=22,yshift=-9em]TCBPOSTER@title.north}{-2em,2em}{{\huge\ref{sumCCconcept}} Select a two-word concept.}

    \commentary{[xshift=3em,yshift=0em]TCBPOSTER@attributes.south east}{-3em,2.8em}{{\huge\ref{sumCCatt}:} Adjust Attributes with the racial bonuses (\autopageref{raceRoll}).}

    \commentary{[xshift=15em,yshift=-6em]TCBPOSTER@title.west}{-2em,2em}{{\huge\ref{sumCCrace}:} Write a name and race.}

    \commentary{[xshift=-4em,yshift=-1em]TCBPOSTER@armoury.north east}{-18em,10em}{}
    \commentary{[xshift=-4em,yshift=0em]TCBPOSTER@armoury.north east}{1em,-23em}{}
    \commentary{[xshift=-4em,yshift=0em]TCBPOSTER@armoury.north east}{-4em,6.5em}{{\huge\ref{sumCCxp}:} Spend 50 \glspl{xp} to purchase Attributes, Skills, and Knacks.}

    \commentary{[xshift=0em,yshift=1em]TCBPOSTER@equipment.south}{0em,0em}{{\huge\ref{sumCCcoin}:} Note starting money, if any.}

    \commentary{[xshift=-7em,yshift=-4em]TCBPOSTER@title.east}{-2em,1em}{{\huge\ref{sumCCcode}:} Select a Code (\autopageref{codes}) and culture.}

    \commentary{[xshift=3em,yshift=2em]TCBPOSTER@derived.south}{0em,0em}{{\huge\ref{sumCCder}:} Fill in the derived stats.}

    \commentary{[xshift=5em,yshift=5em]TCBPOSTER@armoury.south west}{4em,-22em}{}
    \commentary{[xshift=5em,yshift=5em]TCBPOSTER@armoury.south west}{4em,.5em}{{\huge\ref{sumCCequip}:} Take a number of items equal to your total Skills, then fill in any stats for weapons and armour (\autopageref{commonWeapons}).}
}

\input{config/reset_cs.tex}

\setcounter{Strength}{1}
\setcounter{Wits}{-1}
\setcounter{Intelligence}{1}

\renewcommand\concept{Lockpicking Proteg\'e}
\renewcommand\race{Human}
\renewcommand\rank{Guildsman}
\renewcommand\name{Keelfrak}
\renewcommand\code{Chronicler}

\setcounter{Academics}{1}
\setcounter{Athletics}{0}
\setcounter{Caving}{0}
\setcounter{Crafts}{2}
\setcounter{Deceit}{0}
\setcounter{Empathy}{0}
\setcounter{Medicine}{0}
\setcounter{Performance}{0}
\setcounter{Larceny}{1}
\setcounter{Seafaring}{0}
\setcounter{Stealth}{0}
\setcounter{Tactics}{0}
\setcounter{Vigilance}{0}
\setcounter{Wyldcrafting}{0}
\setcounter{Combat}{1}
\setcounter{Projectiles}{0}

\renewcommand\knackOne{\specialist{locks}}

\setcounter{fp}{5}

\renewcommand\characterWeapon{\longsword}
\renewcommand\characterArmour{\partialchain}
\renewcommand\characterEquipment{Lock picking set, torch, tinder box}
\setcounter{gold}{288}

\settoggle{examplecharacter}{true}
\settoggle{genExamples}{true}
\input{config/CS.tex}
\settoggle{genExamples}{false}
\settoggle{examplecharacter}{false}
\renewcommand\csComments{}

