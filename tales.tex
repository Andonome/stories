\chapter[House of Stories]{Stories}
\label{stories}

Players `write' most of their backstory during play rather than before it.
\Glspl{pc} can start off as blank slates with no history, but the history comes out of the woodwork soon after as players can spend 5 \glspl{storypoint} to bring their history into the current mission.

The encounters must take place in a rational manner -- players might find the perfect sellsword in a town, but if they're in a labyrinth, fighting a hall of ghouls, there's little reason for a random sellsword to be present and looking for a job -- this is not an ability to magically summon useful tradesmen with a flash of smoke and plot.
As a result almost all stories will have to be told in populated areas such as towns and villages.

The \gls{gm} is, of course, free to veto any Story suggestions without explanation in order to maintain the integrity of the plot or stop cumbersome play issues.

\section{How to Tell Stories}
\label{listOfStories}

\begin{multicols}{2}

\begin{itemize}
  \item
  Players begin each with 5 \glspl{storypoint} and spend them at any point during the game.
  \item
  All stories should be noted down on the back of the character sheet, including any stats from companions, just in case they enter during a later mission.
  \item
  Each \gls{storypoint} spent earns the \gls{pc} 5 \glspl{xp}.
  \index{XP}
  \item
  Only one expenditure can be made per \gls{interval}.
  If two players want to go first, priority goes to
  \begin{itemize}
    \item
    whoever has the most \glspl{storypoint}, then
    \item
    whoever has the least \glspl{xp}.
  \end{itemize}
\end{itemize}


\begin{exampletext}
  \paragraph{Session 1}
  The troupe slump, tired and lost, at the first bridge on the Doulamarsh road.
  The map said nothing about the bridge, and less about the river.

  At this point, a player decides to spend \pgls{storypoint}.

  \begin{quotation}
    Nickdrawl knows someone in the area.
    Maybe a trader?
    Maybe a cousin?
  \end{quotation}

  The \gls{gm} allows this, but when she rolls up a random character, she creates a gnoll -- so clearly not her \gls{pc}'s cousin.

  \begin{quotation}
    Maybe he was a friend of the family -- the outsider who always brought the best things to the village.
  \end{quotation}

  That makes sense, as gnolls often trade meat.
  She'll roll up the rest of the character as the \gls{gm} begins narrating a wagon coming over the next hill, surrounded by massive dogs.
  The troupe soon find the village that needed their help, and takes notes about the goblin raids affecting them.

\end{exampletext}

\begin{exampletext}
  \paragraph{Session 2}
  Only three players arrive for the next session, and the \gls{gm} says they `might struggle'.
  That doesn't sit well when sending the troupe into their first underground labyrinth.

  Luckily, Nickdrawl has another friend.
  The player recognizes that the troupe needs someone with martial abilities, so she decides to create a character from scratch, without dice, and put most of their starting \glspl{xp} into Martial Skills.
  That won't leave them with much ability to do anything else.

  \begin{quotation}
    An old army friend will join us.
    He's not very bright, can't really plan well, but he \emph{has} survived plenty of missions beyond the \gls{edge}.
    He can lead us into the labyrinth.
  \end{quotation}

  She has a little time to go through the full character creation, given that the session has just begun.
\end{exampletext}

\subsubsection{Themes}

Good character stories tie in together.
Great character stories have a theme.
If all the stories emphasise a clear character trait, they become more memorable, and hammer-home the character's concept.

\paragraph{The passionate poet}
might have a long history of love-affairs.

\begin{itemize}
  \item
  When the troupe require a safe place to rest, he reveals a night spent with a noble woman while her husband had business in the city.
  He could ask her for refuge again, along with the rest of the troupe.
  \item
  Later, he reveals ties to the local Weavers' Guild, and an intimate history with the mistress of the house.
  \item
  The troupe really need some extra muscle.
  Of course, Tealmoor -- warrior of the Northern Plains -- will accompany them, to ensure nothing happens to her `boo'.
  He asks the troupe not to mention the recent week.
  \item
  When an ancient scroll comes up, he spends \pgls{storypoint}, and declares his character knows Gnomish, but says he doesn't want to talk about it\ldots
\end{itemize}

Other good themes include `theft', `gluttony', `death', 

\begin{description}
  \item[Theft:]
  including various stories of burglary gone amiss.
  \item[Mystery:]
  where the character wanted hidden knowledge, picked up a language, skill, or friend, but never quite found out the answers they sought.
  \item[Death:]
  where every anecdote inevitably ends with everyone but the character dying horribly.
\end{description}

\end{multicols}

\section{Sample Stories}
\index{Stories}
\index{Background Stories}

\begin{multicols}{2}

\noindent
The following is a suggested list of Stories the players can tell and their costs.
The players are strongly encouraged to suggest more to the \gls{gm} who will either veto them or give them an appropriate cost.

\story{1}{the Second Language}
\index{Languages}
You have spent a significant amount of time in another culture. You know their language and enough of their background to transfer over basic Skill knowledge. If you have the Performance Skill and are familiar with elvish culture then you also know some Elvish songs.
If you are familiar with gnoll culture and have the Empathy Skill then you know a range of details about gnoll etiquette and lineage.

\begin{itemize}
\item While the dwarves think they're sneakily planning to stab you in plain sight, you actually learnt dwarvish from a blacksmith dwarf who decided to live among humans.
\item Back in the circus, the gnomes could never figure out your `elvish arrow trick'.
You eventually convinced them to teach you how they communicate through whistles in return for teaching them the trick.
\end{itemize}

\story{1}{the Surprise Skill}
\label{surpriseSkill}
You have a surprising Skill or Knack which will comes in useful.
As you tell this story, you can buy a Skill level so long as you have the requisite \gls{xp}.
This cannot be a Skill which you have clearly lacked in the past, e.g. if your character has so far been illiterate then you cannot suddenly learn a level of Academics.
However, if you have never wanted for Craft ability then you could declare that you have always known how to forge iron, or that you have the Seafaring Skill.

\begin{itemize}
\item

While working as one of the \gls{guard} in the Pebbles islands, you got your sea-legs.
Swimming through this underground lake shouldn't be a problem.
You immediately buy two levels of the Seafaring Skill.
\item
You lost a younger sibling to a chitincrawler, and now that your troupe has finally come across one, your burning hatred kicks in as you rush forward.
You purchase the Knack `Chosen Enemy'.

\item Protecting your siblings from griffin attacks on multiple occasions gave you all the knowledge you need to defend them.
You reveal you have the Knack `Guardian'.
\end{itemize}

\story{1}{Guild Ties}
You declare ties to one of the guilds.
Perhaps you served in one, or know a good friend who can pull some strings.
Always check with the \gls{gm} if you can accomplish what you want to with this favour before spending \pgls{storypoint}.

\begin{itemize}
  \item
  You never could make it in the Paper Guild, but your old friends there will provide you with a free map of the labyrinth, in return for exclusive access to any information about the place\ldots assuming you return.
  \item
  The Armourer's Guild taught you everything you know, and you still know nothing.
  Still, they can get the troupe some leather armour for half price, as long as nobody's particular about the colour\ldots
\end{itemize}

\story{1}{the Random Fact}
When the \gls{gm} asks you to make a check to gain knowledge, you can spend a \gls{storypoint} and mention how you know this one particular fact about this topic.
You gain a +6 bonus to a single knowledge check.
This does not count again for the same domain of expertise -- it is only a bonus to knowing one, single fact about the subject.

\begin{itemize}
\item
The troupe want to know how the magical item works.
You've failed the roll, but then remember you've seen this magical item in your mother's book collection.
She had extensive shelves full of bizarre tomes, and all the leafing through those tomes finally paid off.
  \item
  It was unclear if the noble was telling the truth, but you recognise the dyes on his tunic; they come only from the Shale, which can only mean one thing\ldots
  \item
  The troupe have you idea where they are, but you suddenly remember your uncle's maps.
  They were always plastered all over the walls, and you used to imagine walking in those distant lands.
\end{itemize}

\story{1}{the Random Friend}
\label{oldFriend}
You meet an old friend or family member.
They may not be exactly the right person for the job, but they're willing to join you for the mission\ldots

Roll up a random character, as per \nameref{randomCharacterCreation}, \autopageref{randomCharacterCreation}.
They join the group, just like a regular \gls{pc}, for this one session.
They start with 50 \glspl{xp} total, and 50 \glspl{xp} already spent.

This new character joins the player's \gls{characterPool}, so the player can select this character at the start of a session.
However, only the primary character can spend \glspl{storypoint}.

\begin{itemize}
  \item
  The troupe want to traverse a dangerous trail, mostly abandoned.
  On \gls{pc} rolls a random family member\ldots and the world's thickest human says he knows just the way to go.
  \item
  Hunting through all the treasures lost in the sunken library, the players decide they really need an extra pair of hands to haul out the loot, so someone spends \pgls{storypoint}.
  \ldots unfortunately they receive the aid of an emaciated elf with Strength -3.
  At least she can put some scrolls in her pockets.
\end{itemize}

\story{2}{just the Guy}
\index{Allies}
\index{Henchmen}
Someone has precisely the skills you're looking for, and they're coming along for the ride, until the end of the session.

Create a custom character, as per `\nameref{playerchosen}', \autopageref{playerchosen}.

\begin{itemize}
  \item
  With no idea how to talk to the local lord, you suddenly find your old military friend guarding the gate.
  \item
  Everyone wants to buy expensive chainmail, and your dwarvish friend just so happens to have retired here, selling top-quality armours of all types.
  \item
  The troupe need an expert tracker, and on the road you meet your brother.
  He never liked people, so once he got out of the military, he began working independently as a bounty hunter.
\end{itemize}

As before, this character enters the \gls{characterPool}, and may become the player's primary character if their \gls{pc} ever dies.

\story{2}{the Resting Spot}
You know of a secluded and secret location where you will be safe.

\begin{itemize}
  \item
  The guards may be chasing after you, but the Mincing Pig is nearby.  It's famous for some nasty customers and a deep cellar where even the town guard don't want to enter.
  It's been your regular bar since you were twelve years old, and you're sure they'll put you up.
  \item
  The goblins have found your tracks, and they'll catch up soon.
  However, you recall a nearby cave in the forest where you slept the last time you came through here on a mission with the other guards.
  \item
  The bandits are catching up soon, but you recall a walled village nearby.
  Your grandfather was the chief noble before he died, and you're hoping the guards remember you, despite your new beard.
\end{itemize}

\end{multicols}
