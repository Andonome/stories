\chapter{Talisman Workshop}

\begin{multicols}{2}

\noindent
With the right ingredients, ritual casters
\exRef{core}{Core Rules}{ritualCaster}
can create one-use items (`talismans'), to store a single spell.

\subsubsection{Creating Talismans}

\begin{itemize}
  \item
  A talisman requires double the \glspl{mp} that the spell would require.
  \item
  At lest half the \glspl{mp} for the talisman must come from magical ingredients.
  \item
  Talismans must be made from the right ingredients for the elemental spheres involved.
  \begin{description}
    \item[Air]
    uses contained gasses, often held in a phial.
    \item[Earth]
    requires any solid materials dug from the ground.
    \item[Fate]
    talismans require the remains of someone who had \glspl{fp} in life (at least 1 per \gls{mp} cost of the spell).
    \item[Fire]
    requires flammable material.
    \item[Water]
    can use any liquid.
  \end{description}
  Spells which employ multiple spheres must employ multiple ingredients.
  \item
  The creator then decides upon a talisman's activation conditions (see below).
  \item
  The alchemical process requires an \roll{Intelligence}{Alchemy} check, \tn[7] plus the spell's cost.
  \item
  Spells are cast with a bonus equal to their \gls{mp} cost, so an Air spell with a cost of 3 \glspl{mp}, made into a talisman, cast with a straight +3 bonus, rather than a roll of \roll{Charisma}{Air}.
\end{itemize}

\subsubsection{Talisman Activation}

During the ritual, the alchemist emphasises the activation, over and over.
It might be a word, phrase, a concept, condition, or type of tree.
So some items activate when they see the light, others wait until the stars above match the stars when their creator made them.

The creator must impress a talisman's `waking conditions' upon it so well that the ritual always leaves a `telling scar'.
If air in a phial must activate when the right stars show, then the creator will likely carve the pattern of those stars into the phial.
If a scroll, containing a spell of Fire, should activate upon hearing the word `warden', then that word should be written, or heavily implied by writings upon the scroll.

Sometimes talismans `decide' to activate at just the right time, displaying a surprising understanding of context.
At other times, they seem to have a ridiculously literal interpretation of their own activation.
In any case, they have no minds of their own -- merely an echo of their creator's singular thought at the time of the ritual which made them.

\subsubsection{Identifying Talismans}

A \textit{Detailed Witness} spell, employing all of the correct spheres, can ask about whether the contained spell employs particular Enhancements or Effects.
The \gls{tn} for this effect is 7 plus the spell's \gls{mp}

Of course, this only identifies the \emph{effects} of a talisman, not the activation.

\end{multicols}


