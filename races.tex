\chapter[The Melting Pot]{Races}
\label{racesChapter}

\epigraph{
  You know of elves from the tales, dwarves from the news, gnolls from distant travels, and gnomes from all their footnotes.
}

\index{Cultures}
\index{Races}
\label{races}

\noindent
These facts summarize the customs and beliefs of the various peoples in the  local \glspl{region} of \gls{fenestra}.
We must assume the elves further inland, and the humans over the sea act and think much the same, but we have no way to tell.

\section[Dwarves]{Dwarves \Dw}
\index{Dwarves}
\label{dwarvishHomes}

\begin{multicols}{2}
\renewcommand\npcsymbol{\Dw}

\subsection{Settlements}

Far underground, below the soil or coiled up within mountains, the underwyrms roam.
Some are as long as a castle, while others stretch only the length of eight horses.
Their head is that of a streamlined lizard, and they snake limblessly through the bowls of the world, pushing or chewing up raw earth and stone.
They feed on a combination of minerals, rocks and underground fungi.
And in their path they leave wide, wide tunnels.

After underwyrms form tunnels, little dwarves follow on -- strengthening them with properly placed stone arrayed into an arch or packing the tunnel with clay and then setting a fire of mushrooms, underwyrm droppings and underground oil. Then they carve and chisel for decades until they have a hall or room fit to house a dwarf, or a deep fungal garden, powered by an underground lake or river.

Almost all dwarvish communities begin by underground lakes -- many are boating folk, though they do not understand the open sea, or its wind and tides. You know where you stand with a dwarvish lake -- you stand still. It is often at the centre of the lake that one finds the day-bell, a massive bell which forms the pride and heart of any dwarvish community. The day bell rings after 20 hours to say that work has finished and then again 8 hours later to say that work has started again. Many communities buck this trend one way or the other, depending upon the whims of their queen.

The outsides of a dwarvish citadel (or `undertown') are reinforced with metals and very dense clays to discourage outsiders digging in.
Dwarves know exactly what might collapse, how to reinforce walls, and pull them down in a hurry.

Alcohol forms a massive part of dwarvish culture.
They use it primarily for light or cooking, as it produces less smoke than other fuels.
Dwarves, they say, can ferment anything -- living oozes, fungi, goblins.
All life underground eventually converts to light.%
\footnote{Dwarvish saying: \textit{`If it lives, it can die. When it dies, it rots. If it rots, you can burn it'}.}

Commonly, dwarvish tunnels to the outside will end in a gnome-warren.
Direct contact with the outside world, opening into a forest or plain, is seen as `letting the sun in',
and generally frowned upon, but if the dwarvish tunnel ends in a gnomish warren and those gnomes happen to let the sun in, well -- that's \emph{their} business.
This persistent crossing of paths means that the dwarvish and gnomish languages have many common words, and patient speakers of one can mostly understand the other.

\subsubsection{The Language}
\index{Languages!Dwarves}
has an official form, unchanged since the beginning of time.
None of them speak it, and they disagree on every kind of pronunciation, but all dwarves write in the official language.
\index{Dwarves!Names}
To do otherwise would invite shame, as written errors persist longer than spoken errors, and most dwarven speech is an error, at least according to the `official' dwarvish stance.
Dwarves avoid the problems embedded in their own `regional' names by simply translating the meaning when travelling beyond the mountain.

\subsubsection{The Structure}
\label{dwarven_structure}
of their society is heavily matriarchal -- only around one in every ten dwarves is female, so most never marry.
Women stand at the heads of their society and are generally considered too precious to go above ground for the menial tasks of trading for food or cutting down wood.
Rich males compete in fashioning the most exquisite jewellery in order to win the hand of a fair, dwarvish maiden (or indeed, any dwarvish maiden).

\subsection{Commerce}

Underground trade focusses on farming mushrooms, glow-worms for lanterns, underground jellies which feed on water and slime; all manner of underground delicacies are created deep below the earth (though it seems only dwarves actually find them palatable).

Dwarves are famed for their exceptional armour, being the first to invent full plate armour, and still the best at creating it.
They can enter combat fearlessly, knowing that little except an underwyrm can penetrate their thick, steel plates.

\subsection{Warfare}

Dwarves use a lot of smoke when fighting; any enemy coming from a smokey tunnel will invariably suffocate before pushing through defences.

When defending a large entrance, dwarves set themselves up with crossbows, then hand the crossbows back.
Others behind them reload the crossbows in a production line, then hand it back.

While rudimentary crossbow-string might be made from watchers' tendrils, the best comes from hemp.
Dwarves can construct the rest of the item from wood or umberhulk chitin.

When narrower tunnels eventually demand toe-to-toe combat, dwarves always fight with spears or swords (which humans irritatingly refer to as `short swords').
They bring all the nastiest, burnable material they can to a battlefield, such as specially dried mushrooms, or wood, and lay it around the start of a narrow tunnel where they intend to fight.
They stab a little with their spears, then retreat while lighting the fires underneath them.

Dwarves often wet their beards before battle, to protect them from flames.

\needspace{12\baselineskip}
\subsection{Inheritance}

\subsubsection[Tenacity: dwarves take only half the usual penalties from rotten food, poisons, or foul air.]{Tenacity}
\label{dwarvenInheritance}
is learnt with every meal, as dwarves grow up eating the most acrid substances -- tough mushrooms and acidic jellies (well cooked, of course).
Dwarven ales are classified as spirits by any sane human and dwarven spirits are generally classified as poisons by all other races.
The same applies to bad air.

Dwarves take half Damage or \glspl{ep} from any given poison or gas.
They suffer no ill effects from eating rotten food (though it may not count as being nutritious) and the \gls{gm} is encouraged to allow them to eat anything that might otherwise be damaging, within reasonable limits.

\subsubsection{Taciturn}
dwarves trust others slowly, and like to remain formal when first meeting people.
In gaming terms, they cannot spend \glspl{storypoint} during their first session.

\subsubsection{Fire \Glsfmtplural{ingredient}}
\index[mana]{Fire!Dwarves}
can be made from dwarven beards.
Of course, dwarves never like to speak about this, and often dismiss this as a rumour, designed to make trouble between dwarves and \glspl{doula}.

\subsection{Enlistment}

Dwarves join the \gls{guard} for the same reasons as anyone else -- a criminal background, a propensity for violence, and hope of gold.
And each one in the \gls{guard} carries some plan for that gold, even if they never voice it.
Because at the end of the day, it's never about the gold -- it's about what you plan to do with it.

\subsection{Roleplaying Dwarves}
\index{Dwarves!Roleplaying}

Check then double-check.

\begin{itemize}
  \item
  Does this person really know where the lost temple lies?
  Ask him about the rooves, doors, and other items made of wood.
  If people abandoned the temple three centuries ago, those constructions must have degraded.
  Does his story match?
  \item
  Have you really made your point clear?
  Tell him again what will happen if he fails to pay your money back, but \emph{louder}.
  \item
  Does the beer taste good?
  A really good beer still tastes good when you drink three in a row.
  \item
  When the guide says he will lead you all to the lost city, does he mean `within visual range', or `up to the gate', or `to the actual monument, in the centre'?
  Is that written in the contract?
  \item
  Do we have enough torches for this mission?
  If the last crew took two hours to journey down, and three hours back up, and if each torch burns for one hour, then you will need at least five torches for the journey, and one to look around for an hour.
  Best bring ten.

  Share the torches among your companions, so that if you lose one, the group still has enough torches.
  \item
  Has the bandit really died?
  Stab him in the neck, just to make sure.
\end{itemize}

\emph{Write yourself a reminder to double-check this section at the start and end of every session, to make sure you have put it into practice.}

\end{multicols}

\section[Elves]{Elven Glades \El}
\index{Elves}
\label{elvenGlades}

\widePic{Studio_DA/elf_stalker}

\begin{multicols}{2}
\renewcommand\npcsymbol{\El}

\noindent
Elves are highly sexually dimorphic.
As females' shape-changing abilities improve, they gradually become more bestial, and eventually transform into a different kind of creature.
Males tend to become trees, or other plants.

Mating occurs only in their younger years -- within the first few centuries, before becoming bored of their standard bodies.
Their young initially look almost human, so many elves see humans as a kind of giant toddler which dies in infancy.

Elves young enough to still need a community to live in, array themselves in a circular fashion, in `underglass' houses.
They first excavate the entire house with two openings to the top -- one as an exit and the other as an above-ground window.
The window is composed of thick glass -- thick enough for a herd of deer to gallop across.
It lets in feint Sunlight during the day, and at night, when elvish hearths bloom, little lights can be seen across the forest bed as the fire-light shines out of the underglass houses.

Elvish homes are sometimes solitary but more often linked -- they will share chimneys (which leak above ground, sometimes through a tree), exits and often a couple of communal rooms.

Travelling elves often take griffins as their mounts. Rather than capture and tame them, they are expected, through natural magical talent, to instantly befriend them and leave them when the journey is over. The human method of keeping animals in a long-term manner, who then cannot fend for themselves is considered clumsy at best and cruel at worst. Elves pick up what they need as they go and discard it just as quickly.

Elves live for long years -- sometimes up to a millennium -- and as a result become skilled artisans.
Most of this time is often spent simply lounging about, but if they bother even once in five years to make an artistic piece then the forest is soon peppered with little artistic pieces.
Trees carved (or magically shaped) into depictions of battles, or the face of a famously handsome elvish enchanter, or just intricate patterns of knots and spirals carved into stone, so often make an elvish glade look like an art-show.
Some communities put the rubbish outside and leave the best pieces for the sacred centre of the community, where outsiders may not go.
Others leave the centre empty, saving the best pieces for the outskirts of the village and throw the mediocre pieces away.

Each community has its own long-term rhythm and patterns.
Anyone who visits for a day may see chaos, as the elves switch activities at random, and chatter about possibilities, without putting down plans.
But a few decades later, the community will capture the same type of animals to raise as pets, travel among the same dozen spots, and sing the different songs to the same old rhythm.

Elvish communities seldom reach above a population of one hundred.
Those that do are always based around some `Tree Singer' who can sing fruit out of a dead tree.
The majority stay as low as twenty folk who travel long distances between communities.

\subsubsection{The Language}
\index{Languages!Elves}
does not change much, as elven elders maintain old forms of speech for as long as they live.
The similarities between the various Elven languages suggests they come from a single, united source -- a `proto-Elvish'.
But in fact elves just swap songs so often that common elements become inevitable.

Also, note the oddities \vpageref{elfRoleplaying}, under `\nameref{elfRoleplaying}'.

\subsubsection{The Structure}
follows expertise, which often follows age.
In matters concerning hunting, the master hunter will make all group decisions.
In matters concerning statues, the master carver will make communal decisions.
Each expert has their own strict domain of influence.
Many elves translate these `masters' as `king' or `land warden' when speaking with human, and as a result nearly half the elves abroad in human lands claim to be the children of royalty -- exactly how accurate this is depends upon one's interpretation.

\subsection{Commerce}

People think that elves won't trade with anyone due to snobbery.
But in truth, the elves rarely trade with anyone, because nobody has the time.

\subsubsection{Jewellery}
shows how wealthy an elf is, so more wealth means more piercings.
Typically these will be in the ears, but torso piercings are also common.
Rings, necklaces, brooches and all manner of other precious art pieces adorn most elves with any interest in commerce.

The value of jewellery depends on its history -- having a famous maker increases the value, as does a history of being worn during a famous battle, or just by a famous elf.
While trading, elves will explain the complete history of each item, in order to ascertain its worth.
During this time, the seller expects the buyer to sit silently and listen, without reaction.
This custom arose so that buyers who already know about the history of an item can corroborate what the seller tells them, and then inform other elves if the seller's information matches, or does not match.
This chain of listening, comparing, and noting others' reputations keeps the system consistent.%
\footnote{Consistency is often more important than truth.}

While most traders don't have a day or two to simply look at people's wares, some of them manage to break the system.
Gnomes often set up deals through a series of letters (which they then show to others, rather than repeating a long conversation), and some gnolls manage to conclude deals simply by {\small speaking-incredibly-fast} and hurrying the seller along.

\subsubsection{Songs}
form a kind of second currency, as jewellery is rarely traded for music, and vice-versa.
Instead, elves trade a songs for songs.

Elves often store information in songs, including area-knowledge, gossip,%
\footnote{What human refer to as `history', elves refer to as `gossip'.}
recipes, hunting techniques and spells.%
\exRef{core}{Core Rules}{ritualCaster}
They do this partly to make something beautiful, but mostly to solidify their idea, and ensure nobody changes what they have made.
So while elves can change a recipe themselves, they can't pass those changes on to someone else unless they can fit their changes to a new rhyme.

\subsection{Warfare}

As a rule, elves have a wickedly individual mindset, which makes wars difficult.
They understand group-thinking in the abstract, but lack the instinct.

For a band of elves to instigate a battle, each one must have an individual reason why this fight is worth risking their long life.
Of course, battle still occurs, but the groups tend to be small, and usually rely on long-range spells.

Even when defending their own turf from an encroaching enemy, elves usually think about how they -- as an individual -- can deal with the situation before asking others.
They can still work together in defence, and take orders from a central point of command, but the order of their instincts matters a great deal, and often results in elves fleeing when they would likely be victorious.

Despite these limitations, elven \glspl{witch}, with their centuries of practice, ensure the safety of most settlements.
With sufficient motivation, superannuated casters can destroy entire settlements.

\subsection{Inheritance}

\subsubsection[Thermal Apathy: take no penalties from natural weather.]{Thermal Apathy}
\label{elvenInheritance}
means elves are immune to \glspl{ep} from natural heat levels -- they can sleep outside in the snow or wander deserts without sunburn.

Within their own land they might wander naked, or put on clothes just for the joy of adornment.
However, when visiting abroad, they always put on something, as a minimal effort to acclimate to others' cultures.%
\footnote{Nobody ever thanks an elf for all the effort they put in to purchase and maintain clothing, which the elves take as yet another sign that outsiders are all barbarians.}

\subsubsection{Longevity}
means the elves do not degrade, but they do change over the centuries, becoming progressively more fay-looking and alien.
Their minds sharpen, but their bodies degrade.
After 100 years, an elf's maximum Strength Bonus decreases from +2 to +1 but their maximum Dexterity increases to +4.
At 200 years old the elf's maximum Strength score becomes 0 but their maximum Speed Bonus raises to +4.
At 300 the elf's maximum Strength Bonus is -1 but they can move their Intelligence up to +4.
Finally, at 400 years old the elf's Charisma Bonus becomes +4 and their maximum Strength becomes -2.

  \begin{boxtable}[XcX]

    Age & Max. Strength & Increase \\\hline

    100 & +1 & Dexterity \\

    200 & 0 & Speed \\

    300 & -1 & Intelligence \\

    400 & -2 & Charisma \\

  \end{boxtable}

\subsubsection{Water \Glsfmtplural{ingredient}}
\index[mana]{Water!Elves}
can be harvested from elven tears.
Unfortunately, shedding tears also drains elves of all \glspl{mp}, so elves quickly learn to withdraw from unpleasant feelings.

\Glspl{pc} can elect to cry on command by spending \pgls{storypoint}, in order to call upon a tragic memory.

\subsection{Starting Characters}
Elven characters begin young, without the experience, keen intellect and amazing skill-set of their elders.
Some join the \gls{guard} in order to gain the experience they see in their elders.
Others want to learn a specific skill, perhaps to master the rapier or an elemental Sphere.
Most just want to see what the world has to offer.

Elves tend to view their own young as expendable.
They do not reproduce rapidly, but over long centuries a single elf can easily have many children.
Since the youth tend to be stronger than their elders, these young things are encouraged to perform the most dangerous of tasks such as hunting large animals or defending a village through m\^{e}l\'{e}e rather than with a bow.
As a result of this attitude, elves encourage their young to go out into the world and seek knowledge before they become old, delicate and strange.

\index{Elves!Starting \glsfmtlongpl{xp}}
When starting \pgls{campaign}, the \gls{gm} may allow elves to spend all of the \glspl{xp} they would gain from \glspl{storypoint} when creating the character (this usually totals 25~\glspl{xp}).
The character can spend \glspl{storypoint} as usual, but gains no further \glspl{xp} for using them.
Spending all \glspl{xp} up-front allow elven characters to begin with evidence of their lived-experience, then leaves the character broadly unchanged, while the other \glspl{pc} show growth-spurts when spending their initial \glspl{storypoint}.

\subsection{Roleplaying Elves}
\index{Elves!Roleplaying}
\label{elfRoleplaying}

Elven languages have no words for `good', `bad', or `evil'.
As a result, elves to not fully understand or use these words, even when speaking other languages.

Bread cannot `go bad' -- it has mould.
They will never call a song `good' -- the song feels lively, or sounds like a Sunrise, or makes one think of home.
They would never call someone `evil' -- they might say `destructive' or `useless', or `selfish', but never use language which characterizes anything with such a wide notion as `good' or `bad'.

If someone says `your plan sounds good', make sure to clarify if they mean that they want the results of the plan, or if the plan seems likely to succeed, or if the plan has been stated clearly.
And when you hear something is `bad', clarify that too.

\end{multicols}

\section[Gnolls]{Gnoll Hunting Grounds \Nl}
\index{Gnolls}
\label{GnollishGrounds}

\begin{multicols}{2}
\renewcommand\npcsymbol{\Nl}

\noindent
Only the gnolls have the strength and wits to live above-ground, without walls.
They keep and breed fierce hunting dogs, so a group of twenty gnolls will often have around fifty.
These dogs keep watch, sometimes prowling around a camp's outskirts, sometimes simply keeping their ears up, so the camp never lacks sentries.

Smaller groups hunt.
Larger groups generally herd animals, and can be heard a long way off, due to the combined noise of aurochs, goats, sheep, and gossip.

People change from one clan to another depending upon romantic partners or where they find themselves.
The various mobile clans sometimes fight, but always come together when an outsider invades their territory.

\subsubsection{The Language}
\index{Languages!Gnolls}
\index{Languages!Dragons}
sounds distinct from any other, due to the shape of gnolls' jaws.
Other people, with other jaws, struggle to make heads or tails of the gnolls' languages.

Gnollish speech consists of 20 to 50 percent sign-language.
Spoken words tend to relate to far-away communication, such as shouting for aid, or invitations to dinner; signs signal loyalty, subtle descriptions, or veiled warnings.

The Gnollish language shares a great deal of vocabulary with the standard speech of dragons.
According to legend, the gnoll hero Kshonk taught the dragons how to speak so that he could outwit them.

\subsubsection{The Structure}
emerges through hyperactive gossip.
It never ends -- the chatter is constant, but when serious decisions arise about where the tribe should go, or whether it should fight, the gossip reaches incredible speed.
Every gnoll speaks at once, to all sitting beside them, in a short, hurried fashion, acknowledging and expanding upon others' points.
This process goes on for anywhere for twenty minutes to a full night and day.

By the end, they reach a consensus.
Nobody knows exactly how the process works, or what kind of governance to call it.
Outsiders only know that when gnolls start talking, nobody else can keep up.

\subsection{Commerce}

Gnolls primarily trade meat with dwarves and humans, who can never get enough of their own.
They also trade hunting dogs, but charge a high price, and always make sure that the buyer promises to look after the animal properly.
The buyer should keep this promise, as gnolls take note of how buyers care for their animals.
Humans sometimes complain they don't really understand `property', and the gnolls generally don't sell to those humans.

Gnolls sometimes take coinage, but prefer jewellery, as it can be worn, and does not require additional preparation, like money-sacks.

\subsection{Warfare}

Gnolls almost universally employ guerilla tactics.
They set settlements on fire, attack supply lines, and generally poke at every weakness which comes from living in a fixed location.

Massive castle walls daunt gnolls deeply, so they prefer not to attack large civilizations, but if they must do so then they always focus on attacking supply lines, while moving in small groups around the area, encircling it with tiny groups.

\subsection{Inheritance}

\subsubsection[Teeth: grab and grapple in a single manoeuvre]{Fangs}
\label{gnollishInheritance}
mean that Gnolls, like wolves, can grab and damage in a single attack by sinking their teeth into a target.
This deals $1D6 + Str$ Damage.

\subsubsection{Air \Glsfmtplural{ingredient}}
\index[mana]{Air!Gnolls}
can be created from a gnoll's intestines.
Gnoll \glspl{witch} typically extract intestines from the dead.
Others refer to this as a `death ritual', but in fact, gnolls simply value practicality, and rarely bother with rituals.

\subsection{Enlistment}

Gnolls have no `criminals' within their own society (every crime has a fast ultimatum, and possible redemption or death), but those corrupted by human society still end up in the \gls{guard} when breaking the law.

Other gnolls sign up voluntarily.
Being faster and fitter than humans, they stand a better chance than most at survival.

\subsection{Roleplaying Gnolls}
\index{Gnolls!Roleplaying}

Let's go!

\centering{Gnolls get things done, then move on.}

{\raggedleft What's next?\par}

\end{multicols}

\section[Gnomes]{Gnomish Warrens \Gn}
\label{gnomishWarrens}

\begin{multicols}{2}
\renewcommand\npcsymbol{\Gn}

\widePic[t]{Roch_Hercka/illusion_trogdor}

\index{Gnomes}
\noindent
Gnomes live in little warrens, under the ground, but enjoy lots of sunlit openings near the edge of their warren.
Their network of tunnels and homes extend often up to fifty feet below the ground.
These little communities often keep two-level farms -- they tunnel beneath what others consider to be good farmland and then pull cabbages, potatoes, carrots and other rooting vegetables down from the ceiling rather than up from the earth.
They consider humans to be backwards, since root vegetables clearly grown downwards, to emerge at the bottom when ripe.

Gnomes take great pride in remaining `subtle' -- the openings to their houses are never glass but openings which can be closed in order to look as natural as possible -- the side of a hill may open to reveal a living room, or a large, apparently dead tree may have a door opening underground to a small pantry.
Often, the only way to spot a gnomish warren once the doors are closed is to note the bountiful fields of good crops.
Most gnomish gardens cannot support `heavy things', such as a human on horseback.
This leads to humans falling through the soil of a gnomish garden and into a warren, where a number of gnomes have to wonder what to do with a wounded horse and a bemused human rider, and whether or not to keep their warren's location a secret.

All warrens have many traps to secure them from predators and bandits.
\Glspl{woodspy} which probe a tentacle underground often find a razor-sharp edge which contracts as the tentacle withdraws.
\Glspl{crawler} searching the grounds will find myriad entrances, all of which have blades pointing harmlessly inwards\ldots until the creature turns back, and finds the blades far more of a problem when trying to leave the narrow corridor.
If rude dwarves decide to arrive fully-armoured, for a rude visit, they may find a hallway festooned with tiny hooks, just strong enough to snag on their helmets and distract from the cracks in the ceiling.

Gnomes make two distinct types of traps: those built for animals (which anyone with a little sense can see) and those build for people (which nobody can see, unless they understand how gnomes think).
Their \glspl{talisman} work similarly -- gnomes often write some activation work on their alchemical creations in the form of a riddle; this ensures that stupid people who don't speak their language cannot use the item, which functions to stop `bad people' using the item.%
\footnote{We all have our little prejudices, and the gnomish intellect does not make them an exception.}

The gnomish language is rather similar to dwarvish but can change almost as quickly as human languages.
They have three versions -- in addition to being able to speak and write, they can also whistle their language.
The language has a strict way of making sound shifts form normal sounds to whistling sounds.
This allows gnomes to communicate over massive distances -- over wide plains, mountains or through a mile or two of underground tunnels.
It also allows them to hold conversations between each other while standing right in front of people, as most people do not understand that when a gnome is whistling they are also probably saying something meaningful.
Or meaningless.
Gnomes are big fans of using language for its own sake. 

Upon greeting each other, gnomes do not give their names but ask for one -- customarily each person a gnome meets will have one name for them, and a group name will soon emerge for each different social circle. This causes no end of confusion when people ask a gnome what their name is, and the gnome takes this as a sign of an unimaginative companion, before giving the new friend a name without asking what they would like to be called.

\subsubsection{The Language}
has very little vocabulary.
Many assume that such an intelligent little people would develop an extremely complex and precise language.
In fact, the opposite case holds -- the Gnomish language has fewer than 200 words, which then create compound words.

$crazy + water = alcohol$

$flight + animal = bird$

$small + flight + animal = mosquito$

When gnomes become bored of making elaborate compound words, they revert to just one word per concept (or fewer) and expect people to just `get it'.

Despite the difficulties creating compounds, people learn the basics fast, so Gnomish has formed the basis of the \gls{tradeTongue}, which almost everyone uses while travelling.

\subsubsection{The Structure}
varies greatly from warren to warren.
Many Gnomish societies have complicated electoral systems where members cast differing numbers of votes in order to elect to create various positions of government.
These positions are then voted upon with different voting systems, and a third is in place to decide how often votes will take place and how to vote on bringing in new voting systems.
This can take place with warrens with as few as ten gnomes, and often every member of the warren will be in government in some sense or another.
Any time a decision is called upon, gnomes will be delighted to help, and will often return a month later with an ornately carved flowchart of exactly how to determine `Step A' in the `decision-optimization adventure'.
And if nearby dwarves and elves ignore this advice, it's just further evidence that the other races are both impatient and a little stupid.

\subsection{Warfare}

When gnomes can flee, they do so, but otherwise nobody knows what they might do ahead of time.
They dislike repeated tactics or methods.
They prefer unpredictable plans to reliable ones, and often rely on details that people think of as inconsequential, such as what the enemy's shoe-laces are made from, or what the maximum tunnel-size the enemy can comfortably run through.
They might stop and draw a perfect square into the dirt with their finger before running away, leaving any half-sensible enemy to conclude that some hidden trap lies on the ground.
And if they know illusion magic, they always mix together illusory common beasts for the area, illusions of spells they could theoretically cast (and in fact can cast), and ensure all of these illusion have far less detail than they could on the first casting.

\subsubsection{Gnomish traps}
\index{Gnomes!Traps}
\index{Traps}
are known across all of \gls{fenestra}, and demand a high price due to their rarity.
They would be more common, but the traps usually weigh too much for small people to carry them easily, and asking other to carry them would mean showing the big folk where the warren is.

\paragraph{Poison kites}
are kites with small nuggets of poisoned food.
If \pgls{griffin} sees the kite, it usually swoops down and eats it, but may alternatively attack the gnome.
As a result, only specialists fly kites, and gnomes consider the activity an extreme sport.

\paragraph{Water mazes}
are artificial ponds, with rusting spikes covering the bed.
A stone walkway, just inches under the surface, allows anyone who knows the path to enter freely.

\Glspl{basilisk} regularly walk away from them limping, \glspl{crawler} usually avoid bodies of water, and \glspl{digger} can't swim.
Unfortunately the mazes need someone to repeatedly defile the water, keeping it putrid and vile, or \pgls{woodspy} may move in.

\paragraph{\Gls{woodspy} chests}
have two or three latches which lock the box the moment \pgls{woodspy} enters.
Small pieces of wood keep the chest open until pushed aside.
The chests don't have any bait -- no meat or wriggling rodent.
Instead, the chest simply sits near a path or door, as if to say `gee, I sure hope nobody goes into my treasure chest and ambushes me later'.

These chests usually have a small opening at the back next to a bell.
The opening leaves just enough room for a flustered \gls{woodspy}'s  tentacle to squeeze the tip out, and feel around for any way out.
At this point, the tentacle-tip rings the bell, which lets the gnomes know they've caught something.

Trappers must empty the trap quickly, because \glspl{woodspy} are semi-intelligent, and complete psychopaths.
If they see another \gls{woodspy} caught in a chest, they will not only understand the trap, but wait beside it, in the hopes of catching a gnome.

\subsection{Commerce}

Gnomes use treasure maps as currency, as none of them want to walk long distances with bags of heavy coins.
Humans who try to use these maps to hunt down the treasure are always disappointed, as they lift the treasure-chest's lid, and find it's full of maps.
This is because, as covered earlier, gnomes use treasure maps as currency.

Some say that the maps comprise a grand treasure-hunt, which stretches across all of \gls{fenestra}.
Others say that someone has probably found every real treasure which once existed.
It makes no difference, because as far as gnomes are concerned, treasure maps represent value.

\needspace{12\baselineskip}
\subsection{Inheritance}

\subsubsection[Attentiveness: roll 2D6+3 for resting actions]{Attentiveness}
\label{gnomishInheritance}
means Gnomes often have a hard time focussing on things, but once they successfully do so, they focus to the exclusion of all else, often with amazing results.
When gnomes take a Resting Action, rather than rolling $1D6$ and adding +6, they roll $2D6+3$.
If they want to change a failed action into a Resting Action, they add $1D6-3$ to their roll.

\subsubsection{Earth \Glsfmtplural{ingredient}}
\index[mana]{Earth!Gnomes}
can be harvested from Gnome-bones with enough grinding.
Many theorize that this explains why \glspl{griffin} seem so eager to eat them.%
\footnote{Gnomes have their own theory, but it makes for a bad story, so nobody can remember their claims.}

\subsection{Enlistment}

Many join the \gls{guard} to find rare \glspl{ingredient} for \glspl{talisman}.
Others join the \gls{paperGuild}, then steal some valuable book when they think nobody can see them, and end up in the \gls{guard}.

\subsection{Roleplaying Gnomes}
\index{Gnomes!Roleplaying}

{\raggedleft Think sideways.\par}
\noindent
If Human things are `Human', and Dwarven things are `Dwarven', is my hat `Gnome' or `Gnomen'?
Can we apologize to the \gls{witch} and make amends instead of killing her?
Can you use a hammer to communicate?
What else do shoes do?

Gnomes see the world from a different perspective.
They look up people`s noses all day.
Gnomes see the ceiling while others look down at the ground.

Gnomes travel slowly but it looks like a large space to them.
From a relative perspective, a travelling Gnome has travelled farther than the rest of the troupe.
Are we counting footsteps or miles?
Did you know that every mile has 5.280 feet?

Where did the \gls{witch} commission her traps?
Is the architect still alive?
Does he have standard schematics for his traps in a workshop where he builds traps for people?

What kind of contract do you make when you sell someone a trap to guard their labyrinth?
What happens if I roll a boulder down the stairs?
Have these traps killed before?
Where do the bodies go?
Does someone climb down to get them out and do they use a ladder?
If we dig out the stream nearby, we could flood the labyrinth.

\end{multicols}

\section[Humans]{Human Towns \Hu}
\index{Humans}
\label{humanTowns}

\begin{multicols}{2}
\renewcommand\npcsymbol{\Hu}

\noindent
Humans are a massive, war-like race with round ears, who all live above ground, despite the dangers.
Their strong arms let them wield long, iron weapons for battle, and build high wooden walls and dig deep ditches around their settlements.

Their large size and tall walls don't protect all of the humans from being eaten by large creatures of the forest, so populations must survive by having many children.
Humans reproduce at an alarming rate -- instead of simply replacing themselves with two or three more humans, a couple might make as many as fourteen.
A good many will practice the hunting bow daily to protect their animals from forest predators.

The repetitive activities of their \glspl{village} hide a subtle chaos.
No matter how long one lives with them, each one does the same thing, at the same time of day, every day.
But when people return a dozen years later, they find half the humans have a different routine, and the other half have died.

When enough humans form together, a city often springs up in the centre, organically.
None of them organize where to build the city, it simply emerges around the middle of anywhere they can defend.
Once a city establishes itself in the area, humans will start to raise specialists who can practice at woodworking, book-binding, and other specialized skills.

\widePic[t]{Leonard/next_day}

\subsubsection{The Language}
\index{Languages!Humans}
looks very unified in speech, but with different styles of writing.
This illusion comes from the fact that every time humans travel, they pick up a few local words, and copy a little of the local accent.
By the time someone has travelled a thousand miles, they have arrived at a very different language, without learning anything entirely new at any step.

Exactly what counts as a new language depends a little on shared words, and a lot on the speed of travel.
Simply put:

$$ comprehension = \frac{cromulance}{velocity} $$

The local human languages share enough words with Gnomish that the two are mutually intelligible, as long as both speakers have some patience.
However, when humans speak their local dialects quickly, nobody can understand them except their relatives and others who come from a nearby \gls{village}, or from the same town.
Many use their dialects as a secret language, or `cant', when they want to speak privately.

\subsubsection{The Structure}
of any \gls{village} or town is extremely important, because humans love hierarchies and become confused about what they are doing if they cannot identify a nearby leader.
As a result, specialized decision-makers arise, usually inside cities, called `\glspl{warden}', who dictate what happens in a city, and distribute justice to criminals in a large court-house.

\subsection{Commerce}

Humans' massive feet and their habit of following each other creates massive roads.
Additionally, they trade live animals more often than hunted game, which creates more roads as cows, sheep, and goats trample down every possible route between human settlements.

They cannot weave quality spells, or make long-lasting armour, but the sheer quantity of goods they have to trade always lets them purchase these goods from others.

\subsection{Warfare}
Humans always rely on numbers in battle.
Coupled with their incredible size, they make a formidable force without much need for additional tactics.

Due to their slow minds, humans need to use simplified signals for battles, such as trumpets or flags, which can signal where everyone should go.

\needspace{12\baselineskip}
\subsection{Inheritance}

\subsubsection[Marching Legs: every \glsentrytext{ep} spent to march adds 2 miles]{Marching Legs}
\label{humanInheritance}
mean humans can walk a long way before they feel tired.
Instead of taking \pgls{ep} to cover an extra mile, they can take \pgls{ep} to cover an additional 2~miles.

Humans may seem slow and clumsy, and may not run terribly well.
But when time is measured in days, they are the fastest in \gls{fenestra}.

\subsubsection{Fate \Glsfmtplural{ingredient}}
\index[mana]{Fate!Humans}
can be made from human blood, if distilled into an ink.
Sometimes it makes for two \glspl{ingredient}, if the human had enough blood.

This process takes a long time, but the results speak for themselves, so many spell-casters will refuse to enter battle until they have enough humans write some bloody \glspl{boon}.

\subsection{Enlistment}

The \gls{guard} exists to ferry all the excess humans into the forest, to protect those who can plant and create.
They never need much of a reason to join the guard.

\subsection{Roleplaying Humans}
\index{Humans!Roleplaying}

If something doesn't work, humans just try a different method.
If they can't buy what they want, they often just steal.
When they can't figure out a spell to make the plants grow, they turn to studying natural cultivation.
And when they can't open a door, they start hammering the walls.

Humans may seem dim when you watch them work, but come back a year later and you can often find them with the same goal, using a new technique.

\end{multicols}

