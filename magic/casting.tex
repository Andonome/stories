\chapter{Spell Weaving}

Magic cannot create.
It warps, waxes, and wanes.
It gives information.
But nothing in the world comes from nothing, so spellcasters have to work with the environment, but they can only change what their spheres govern.

Casters create a spell's foundation by deciding what \emph{action} to perform on which element.
Fire casters may \textit{Wax Fire}, causing it to roar and explode.
Water casters can \textit{Wane the River}, making it evaporate.
\textit{Witness Fate} spells allow the caster to know if fate has plans for someone nearby.
So each of the five spheres allows the caster to create four basic spells -- wax, wane, warp, and witness.

More advanced casters can add enhancements, though they come with complications and costs.
The `duplicated' enhancement lets spells affect a wider area, or more targets, but the caster may inadvertently target allies.
`Distant' spells can affect anyone far away, but not someone close by.
Each enhancement added increases the spell's potency, at the potential cost of unwanted complications.

On top of the basic five low spheres, casters can employ high spheres by combining two of the Low Spheres together.
Air and Fate create Death magic, while Water and Earth create Life magic.

\vspace{2em}
\newlength{\magicCircle}
\setlength{\magicCircle}{.32\textwidth}
\newlength{\sphereBack}
\setlength{\sphereBack}{18pt}

\begin{centering}
\setcounter{track}{161}
\begin{tikzpicture}[very thick]

  \foreach \x in {1,...,5} {
    \node (\x) at (\arabic{track}:\magicCircle) {};
    \addtocounter{track}{72}
  }

  \foreach \x in {1,...,5} {
    \setcounter{enc}{\x}
    \addtocounter{enc}{2}
    \ifnum\value{enc}>5
      \addtocounter{enc}{-5}
    \fi
    \draw[fill=\pageSideColor] (\x) -- (\arabic{enc});
  }

  \foreach \x in {1,...,5} {
    \draw[fill=\pageSideColor] (\x) circle(\sphereBack){};
    \addtocounter{track}{72}
  }

  \addtocounter{track}{36}
  \foreach \x in {6,...,10} {
    \node (\x) at (\arabic{track}:.38\magicCircle) {};
    \draw[fill=\pageSideColor] (\x) circle(\sphereBack){};
    \addtocounter{track}{72}
  }

  \node (X) at  (6) {\Large\outline{\textgoth{Earth}}};
  \node (X) at  (7) {\Large\outline{\textgoth{Water}}};
  \node (X) at  (8) {\Large\outline{\textgoth{Fate}}};
  \node (X) at  (9) {\Large\outline{\textgoth{Air}}};
  \node (X) at (10) {\Large\outline{\textgoth{Fire}}};
  \node (X) at  (1) {\huge\outline{\textgoth{Force}}};
  \node (X) at  (2) {\huge\outline{\textgoth{Life}}};
  \node (X) at  (3) {\huge\outline{\textgoth{Mind}}};
  \node (X) at  (4) {\huge\outline{\textgoth{Death}}};
  \node (X) at  (5) {\huge\outline{\textgoth{Light}}};
\end{tikzpicture}
\end{centering}

\section{The Rules}

\begin{multicols}{2}

\begin{enumerate}
  \item
  Select an elemental Sphere.
  \begin{itemize}
    \item
    If you have a Sphere, you can target that element directly.
    \item
    Each of the Low Spheres can combine with two others to make a High Sphere, equal to the lowest of the two elements creating it.
    \item
    For example, a witch with Fire 1 and with Air 2 could cast \textit{Light} magic at level 1.
  \end{itemize}
  \item
  Select one of the four Weaves:
  \begin{description}
    \item[Waxing]
    spells encourage the element, making it \emph{more} like what it is.
    \item[Waning]
    magic does the opposite -- it reduces the element's nature, and often destroys the target.
    \item[Warping]
    an element alters some fundamental aspect, promoting strange behaviour and effects from the target.
    \item[Witnessing]
    means to find out whether or not the element exists somewhere.
    They always come in the form of `yes/ no' answers.
  \end{description}
  \item
  Spend \glspl{mp} to power your spell.
  \begin{itemize}
    \item
    You can spend \glspl{mp} up to your level in the Sphere.
    \item
    The spell's level equals the \gls{mp} cost.
    \item
    If you have 0 \glspl{mp}, each extra levels inflicts \pgls{fatigue}.
  \end{itemize}
  \item
  For each \gls{mp} spent beyond the first, you \emph{must} select another Enhancement.
  \begin{description}
    \item[Detailed]
    spells let you control the appearance and minute properties of a spell, or let you specify details of an element with \textit{Witness} spells. 
    \item[Distant]
    spells throw your magic far into the distance, and no less.
    These spells cannot target shorter distances.

    \begin{boxtable}[cLL]
      \textbf{\Glsentrytext{lv}} & Standard Distance & Enhanced Distance       \\
      \hline
                  1              & 20 \glspl{step} &                   \\
                  2              & 15 \glspl{step} & throwing distance \\
                  3              & 10 \glspl{step} & shouting distance \\
                  4              &  5 \glspl{step} & horizon           \\
                  5              &                 & line of sight     \\
    \end{boxtable}
    \item[Divergent]
    spells select an opposing element, and add the same Modes and Enhancements to produce a secondary effect.
    \begin{itemize}
      \item
      High spheres oppose high spheres, and the low oppose the low.
      \item
      Any spheres which do not neighbour each other, oppose each other.
      \item
      As usual, all spheres used must be equal in level.
    \end{itemize}
    \item[Duplicated]
    spells fork like lightning, affecting every available target nearby.
    The caster does not select the remaining targets.
    \begin{itemize}
      \item
      Fire spells target all nearby fires.
      \item
      Earth spells usually target a large patch of rock or ice, as all the substance around makes for other viable targets.
    \end{itemize}
  \end{description}
  \item
  Roll to resolve, with \roll{Charisma}{Sphere}, while stating your intention!
  \begin{itemize}
    \item
    The element's resistance may raise or lower the \gls{tn}.
    \item
    Breaking hard earth is difficult, but snow is easy.
    Blasting wind in a house is difficult, but redirecting a storm is not.
    \item
    If the effects target a person, then the person can \emph{also} resist the effects with any appropriate combination of Traits.
  \end{itemize}
  \item
  Apply effects.
  \begin{itemize}
    \item
    Low Spheres create mechanical effects equal to their level +2.
    \item
    High Spheres create a mechanical effect equal to their level +1.
  \end{itemize}
\end{enumerate}

\end{multicols}
