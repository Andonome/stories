\chapter{Interpreted Characters}
\index{Traits}
\index{Character Creation}
\label{character_rolls}

It's time you build some character.
Specifically, \pgls{pc}.
Take a blade and slice out the character sheet.

\setglossarypreamble[symbols]{
  \begin{multicols}{2}
  \noindent
  Grab $2D6$.
  You will need to roll them eight times to create this character.

  \input{config/rules/attributes.tex}

  Roll $2D6$ for each Attribute in order, and mark each result on your character sheet.
  \begin{exampletext}
  \noindent
  If you roll `3, 8, 9', then you have `Strength -2, Dexterity +0, Speed +1'.
  \end{exampletext}

  \label{randomCharacterCreation}
  \attributeChart

  Note the pictograms below -- people in \gls{fenestra} make heavy use of them, to ease communication across distance cultures and centuries.
  Now you can roll to find out what type of creature this character is by the images \vpageref{Roch_Hercka/five_races}.
  \end{multicols}
  \bigLine
}

\renewcommand{\glsmcols}{3}
\printglossary[
  type=symbols,
  title=Inheritance,
  style=mcolindex,
]
\renewcommand{\glsmcols}{2}
\setglossarypreamble[symbols]{}

\bigLine

\begin{multicols}{2}

\mapPic[\Large]{t}{Roch_Hercka/five_races}{
  {\Huge\Gn}/08/99,
  2-3/08/83,
  {$+1$ Dexterity}/02/24,
  {$+1$ Intelligence}/02/16,
  {$-2$ Strength}/02/08,
  {$-1$ Speed}/02/0,
  {\Huge\Dw}/25/99,
  4-5/25/86,
  {$+1$ Dexterity}/25/36,
  {$-1$ Speed}/25/26,
  {\Huge\Hu}/43/99,
  6-8/41/91,
  {$+1$ Strength}/49/34,
  {$-1$ Wits}/49/27,
  {\Huge\El}/64/99,
  9-10/57/92,
  {$+1$ Wits}/65/36,
  {$-1$ Strength}/65/26,
  {\Huge\Nl}/77/99,
  11-12/84/92,
  {$+1$ Strength}/82/24,
  {$+1$ Speed}/82/16,
  {$-1$ Intelligence}/82/0,
  {$-2$ Charisma}/82/08,
}

\subsubsection{Your First Name}
\label{raceRoll}%
A million cultures may inhabit little pockets in these dark, green, hunting-grounds, but we can only see the settlements within walking distance.
Traders, \glspl{warden}, and \glspl{seeker} walk along the tapestry of human roads which bridge the forest.
They tell stories of other lands, but the latter only wants to tell you to sell you, and the former don't speak to your type.

You can find out more about your character's home and customs in \autoref{races}, but for now, you need a name.
Whether you want to name your own character or roll randomly, start by looking at the conventions.

\end{multicols}

\newpage

\begin{multicols}{3}

\index{Names}%
\label{randomNames}%

\paragraph{Dwarven Names}
\index{Dwarves!Names}
come from landscape features, while their surname indicates their mother's occupation.
Occupations with more than two syllables take a short-form (dwarves are busy people).
So the surname `Spector' comes from `inspector', and if that dwarf's mother later finds a job as a cave surveyor, they would change his surname to `Veyor' (and would expect companions to supply a congratulatory drink).

\begin{boxtable}[l|>{\small}Y>{\small}Y>{\footnotesize}r]

\Dw & \textbf{\M} & \textbf{\F} & \textbf{Mother} \\\hline
\ifodd\value{r3}
  \dicef{1}   & Bight         & Tor             & Spector     \\
  \dicef{2}   & Dale          & Drumlin         & Llector     \\
  \dicef{3}   & Ben           & Glen            & Jeweller    \\
\else
  \dicef{1}   & Strand        & Bornhardt       & Binder      \\
  \dicef{2}   & Thalweg       & Morgaine        & Chanter     \\
  \dicef{3}   & Dell          & Maar            & Tographer   \\
\fi
\ifodd\value{r4}
  \dicef{4}   & Turlough      & Hogback         & Carver      \\
  \dicef{5}   & Graben        & Morgaine        & Chef        \\
  \dicef{6}   & Riffle        & Scree           & Dener     \\
\else
  \dicef{4}   & Foiba         & Mamelon         & Veyor       \\
  \dicef{5}   & Horst         & Karst           & Brewer      \\
  \dicef{6}   & Sandur        & Scarp           & Chitect   \\
\fi

\end{boxtable}

\paragraph{Gnollish Names}
\index{Gnolls!Names}
are short, to-the-point, and never require difficult lip-movements (assuming you have canine lips).
The meanings generally relate to the gnoll's primary joy, such as `hunting', or `biscuits'.

\begin{boxtable}[c|lY]
\Nl & \textbf{Prefix} & \textbf{Suffix} \\\hline
\dicef{1}  & Ksha & --dz  \\
\dicef{2}  & Ko   & --g   \\
\dicef{3}  & Sya  & --h   \\
\dicef{4}  & Tso  & --d   \\
\dicef{5}  & Yo   & --sh  \\
\dicef{6}  & Riye & --tse \\
\end{boxtable}

\paragraph{Elven Names}
\index{Elves!Names}
represent long stretches of their lives -- generally as long as the language survives.
Roll once for the prefix, and again for a female or male suffix.

The `\"e' symbol means you should pronounce the sound fully, as in `f\textbf{ai}rie', or `sel\textbf{e}ct'.

\begin{boxtable}[l|l|YY]
 \El & \textbf{\E\glsadd{E} Prefix} & \textbf{\F\glsadd{F} Suffix}   & \textbf{\M\glsadd{M} Suffix} \\\hline
\dicef{1} & Sind    & --\"e    & --on      \\
\dicef{2} & Atar    & --ink\"e & --inkon   \\
\dicef{3} & Ciry    & --inw\"e & --iel     \\
\dicef{4} & Tarin   & --\'ote  & --or      \\
\dicef{5} & Fin     & --uin    & --acil    \\
\ifodd\value{r3}
  \dicef{6} & It\'ar    & --w\"e   & --il      \\
\else
  \dicef{6} & Itar    & --il     & --ill\"e  \\
\fi
\end{boxtable}

\paragraph{Human Names}
\index{Humans!Names}
remain static throughout their lives, so they never have any relation to the person, or their accomplishments.
Humans believe that the forest takes the best part of everything, so they avoid giving their children any names which imply anything wonderful, or even positive.

\namesOfHumans

\paragraph{Gnomish Names}
\index{Gnomes!Names}
are given, not taken, so every gnome in a community will have another name from every other member of the community.
However, if the gnome feels generous, and does not want to trouble any of the `big folk' with the task of creating two or three syllables, they may provide a name.
Roll $1D6 \times 1D6$, and re-roll on doubles to add another part to the name.

\begin{boxtable}[rY|rY]
  \Gn & \textbf{Syllable} & \Gn & \textbf{Syllable} \\
  \hline
  1  & ni    & 12 & yelo  \\
  2  & lawa  & 15 & musi  \\
  3  & noka  & 16 & ma    \\
  4  & en    & 18 & leta  \\
  5  & ante  & 20 & nanpa \\
  6  & alasa & 24 & mute  \\
  8  & yan   & 25 & wan   \\
  9  & mu    & 30 & open  \\
  10 & kala  & 36 & tu    \\
\end{boxtable}

\end{multicols}

\pagebreak[3]

\begin{multicols}{2}

\subsection{Gifts}

The biggest difference that makes a difference between dwarves and gnolls isn't teeth, or beards -- it's the dwarven ability to live without Sunlight, and comfortably eat the things that live underground.

These small difference can threaten to divide a group of different people working together, so whatever aptitudes you have, best not to make too much of it around the others; just make a quiet note of your ability on your character sheet.

\begin{boxtable}[lL]
  Dwarves & \nameref{dwarvenInheritance}, \autopageref{dwarvenInheritance} \\
  Elves & \nameref{elvenInheritance}, \autopageref{elvenInheritance} \\
  Gnolls & \nameref{gnollishInheritance}, \autopageref{gnollishInheritance} \\
  Gnomes & \nameref{gnomishInheritance}, \autopageref{gnomishInheritance} \\
  Humans & \nameref{humanInheritance}, \autopageref{humanInheritance} \\
\end{boxtable}

\end{multicols}

\setglossarypreamble[\acronymtype]{
  Now you have a character, you should check it for suspicious abbreviations and empty boxes.
  Most of these terms have found interplaner adoption; but best make sure before you go.

  To find more on a rule, check the rules summary at the end.
  And if you have any spellcasting-Spheres, pick up three spells from the \textit{Core Rules}.
}

\printglossary[
  type=\acronymtype,
  style=mcolindex,
  title=Gnomenclature,
  nonumberlist,
]
\setglossarypreamble[\acronymtype]{}


\exampleRandomCharacter % in commands.tex

\section{Enlistment}
\index{Enlistment}
\index{Concept!Random}
\label{enlistment}

\begin{multicols}{2}

\noindent
Why has this character joined the \gls{guard}?
First check your `failure' -- the lowest Attribute you've rolled, and under that section check your highest Attribute.
If multiple Attributes compete for lowest or highest, stop at the first.

Write down the concept, code, stats and equipment.

\subsubsection{Strength Failure}

\begin{itemize}

  \showConcept{Dexterity}%Attribute
    {Thief}% Concept
    {
      You stole from your family, then stole from the neighbours, and finally took to snatching from slow, fat, and rich people.
      But stealing from rich people is a crime, so now you must repay your debt.

      The \gls{jotter} will dock your pay; receive only half the normal payment until your commander decides you have learnt your lesson.

    }% Description
    {Noble}% Code
    {Combat~1, Projectiles~1, Crafts~1, Larceny~2, Stealth~1, and the knack \specialist{locks}}% Stats
    {a lock-picking set, three daggers, and \arabic{r4} \glspl{sp}}% Equipment

  \showConcept{Speed}%Attribute
    {Coward}% Concept
    {
      You might at least have shouted or at least thrown a rock at the monster, but you just hid and cried quietly.
      Now the forest has eaten our best animals.
      Off you go to the \gls{guard}, to learn courage.
    }% Description
    {Noble}% Code
    {
      Projectiles 1, Athletics 2, Empathy 1, Vigilance 1, Wyldcrafting 1, the knack \textit{Last Stand} and raise any Attribute from -1 to 0 (or purchase the Caving Skill).
    }% Stats
    {
      four throwing-daggers, a buckler shield, three travelling rations, and an idea of which \gls{pc} you want to hide behind when the situation goes South.
    }% Equipment


  \showConcept{Intelligence}%Attribute
    {Layabout}% Concept
    {
      People like you can't just talk and recite poems all day.
      Nobody cares that you can read the stars if you won't help with the work.
      You think you're too good to work?
      Time to put that brain to use\ldots
    }% Description
    {Wanderer}% Code
    {%!
      Academics 2, Performance 1, Vigilance 1, and the Rituals Knack.
    Then fill in Earth 1, Fire 1.
    }% Stats
    {
      a piece of chalk, a tinder-box, a mirror, 50' of rope, writing equipment, and four letters you wrote (still undelivered).
    }% Equipment

  \showConcept{Wits}%Attribute
    {Gossiper}% Concept
    {
      While everyone else created, toiled, and cared for their animals, you set neighbour against neighbour with your incessant chatter.
      Talking time is \emph{over}.
    }% Description
    {Paladin}% Code
    {Deceit 3, Empathy 1, Larceny 1, Tactics 1, Vigilance 1}% Stats
    {a javelin, a dagger, a tinder-box, a pouch of pitch,%
      and a rumour you overheard about another \gls{pc}.}% Equipment

  \showConcept{Charisma}%Attribute
    {Swindler}% Concept
    {
      Those people trusted you with their money.
      They believed your ointments would cure gout, and the secret prayers would banish the next storm.
      Now reality knocks at the door, and it won't listen to your clever stories.

      \textit{Ask any other \gls{pc} to roll \roll{Wits}{Empathy}, (\tn[8]).
      If they pass, fill in their backstory with a time you tricked them out of something.
      If they fail, write down the event, and hand it to the \gls{gm}.}
    }% Description
    {Chronicler}% Code
    {Athletics 1, Deceit 2, Empathy 1, Performance 2, Seafaring 1, and Vigilance 1}% Stats
    {
      a dagger, a mirror, a pouch of unidentified seeds, and a flute.
    }% Equipment

\end{itemize}

\null

\needspace{12\baselineskip}

\subsubsection{Dexterity Failure}

\nobreak
\begin{itemize}

  \showConcept{Strength}%Attribute
    {Oaf}% Concept
    {300 \glsentrylongpl{gp} down the drain, because you couldn't pick up a vase properly.
      But there's a place they welcome heavy-handed people\ldots
    }% Description
    {Tribe}% Code
    {Brawl~2, Crafts~2, Vigilance~1}% Stats
    {a longsword, complete leather armour (covered in faint marks from tentacle suckers), three days of rations, 50 \glspl{cp}, and a knot-puzzle you cannot solve.}% Equipment

  \showConcept{Speed}%Attribute
    {Klutz}% Concept
    {You zip about and run into people.
      You climb houses, and knock off the roof.
      One too many misadventures and it was time to let you fall into the clutches of the forest.}% Description
    {Chronicler}% Code
    {Brawl~1, Athletics~2, Larceny~2, Seafaring~1, and the Knack \charge.}% Stats
    {a short sword, partial leather armour with fang-holes through the abdomen, and \arabic{r12} \glspl{cp}.}% Equipment

  \showConcept{Intelligence}%Attribute
    {Loner}% Concept
    {Too good to speak with the farmers, too useless to weave with the weavers.
      Time to find a place in the most accepting organization in all the land\ldots}% Description
    {Conqueror}% Code
    {
      Air~2, Academics~2, Tactics~1
    }% Stats
    {a bag of flour, a bag of chalk, a shortsword, a dagger, and an unopened letter from home.}% Equipment

  \showConcept{Wits}%Attribute
    {Traitor}% Concept
    {
      When guards came knocking, you blamed your crimes on the others in your hovel, until the others figured out your game.
      They want you gone for good, so in the end the \gls{guard} took you anyway.
    }% Description
    {Noble}% Code
    {
      Combat~1, Athletics~1, Deceit~2, Stealth~2, Vigilance~1,
    }% Stats
    {a shortsword, a dagger, and \arabic{r12} \glspl{cp}.}% Equipment

  \showConcept{Charisma}%Attribute
    {Critic}% Concept
    {
      You could tear anyone's work apart at the \gls{armourHall} with one comment.
      With your friends jeering with you, artisans fell to rage or tears, and you had plenty of friends.
      But when time came to make your own works, they never came out quite right.
      Nobody likes a critic\ldots
    }% Description
    {Jester}% Code
    {
      Projectiles~1, Empathy~1, Crafts~1, Deceit~2, Performance~2
    }% Stats
    {a broken sword, once-complete (now partial) leather armour (arms ripped off, helmet missing), 3 \glspl{sp}, and some helpful advice for the player to your left.}% Equipment

\end{itemize}

\needspace{10em}
\subsubsection{Speed Failure}

\begin{itemize}

  \showConcept{Strength}%Attribute
    {Fatty}% Concept
    {
      You just eat and eat, helping nobody too far from the table.
      Those limited connections won't get you too far, so it's time to earn all that food.

      Begin play with six rations, a great sword, partial chain (you cannot wash the basilisk-stench off the tunic), a frying pan, and a craving for something in particular\ldots.
    }% Description
    {Tribe}% Code
    {Projectiles~2, Academics~1, Crafts~1, Medicine~1, Wyldcrafting~1}% Stats
    {a shortsword and a sandwich}% Equipment

  \showConcept{Dexterity}%Attribute
    {Brigand}% Concept
    {
  Younger siblings never inherit, and your choice of guilds had no room to take you, so you took to the road before anyone could shunt you into the worst guild of all -- \gls{guard}.
  The only way to survive was to band together with others, stealing what you could from good folks houses at night, and running away.

  The \gls{guard} chased your group down and put half of them to the sword while you just gave up, knowing you couldn't run, and threw the rest into `the Pit of Justice'.
  The warden and his jester laughed as he sentenced you to join the \gls{guard}.
  }% Description
  {Noble}% Code
  {Combat~1, Projectiles~1, Caving~1, Crafts~1, Stealth~2, Wyldcrafting~1}% Stats
  {a javelin, partial leather armour with claw-holes through the chest, \arabic{r12}0 \glspl{sp} buried nearby, and nasty black eye (-1 \gls{hp}).}% Equipment


  \showConcept{Intelligence}
  {Upstart}% Concept
  {You write letters, and get everyone else to send them.
  You arrange new deals, but can't deliver the goods.
  Well nobody needs a king without a kingdom, and nobody's going to do your work for you.
  }% Description
  {Paladin}% Code
  {Combat~1, Brawl~1, Academics~1, Empathy~1, Medicine~1, Tactics~2}% Stats
  {writing equipment, a torch, and a dagger.}% Equipment

  \showConcept{Wits}
  {Imp}% Concept
  {You just sit there telling jokes and mocking people.
  You spot the problems, then wait for others to solve them.
  It's time you paired that sharp wit with a sharp blade.
  }% Description
  {Jester}% Code
  {Air~1, Fire~1, Empathy~1, Performance~2, Seafaring~1, and the Knack \snapcaster}% Stats
  {a short sword, a dagger, and \arabic{r4} \glspl{sp}.}% Equipment

  \showConcept{Charisma}
  {Cynic}% Concept
  {You see issues, and moan, moan, moan.
  We can't stand to see your ugly, down-turned face around here any longer.
  Go fix all the problems you see!
  }% Description
  {Chronicler}% Code
  {Combat~1, Academics~1, Empathy~2, Vigilance~2, and the Knack \adrenalinesurge}% Stats
  {a longsword, complete leather armour (chest covered with scorch-marks), and an attitude problem.}% Equipment


\end{itemize}

\needspace{10em}
\subsubsection{Intelligence Failure}

\begin{itemize}

  \showConcept{Strength}
  {Bastard}% Concept
  {Your mother never wanted to fight in the \gls{guard}, and needed an excuse to stay away from the \gls{edge}.
  That's what you are -- an excuse of a person, raised with the rejected.
  Now the laws have changed, and pregnancy's no excuse for a woman to avoid battle.
  Go pick up your stick, and enjoy your birthright.}% Description
  {Conqueror}% Code
  {Combat~2, Brawl~1, Athletics~1, and the Knack \adrenalinesurge}% Stats
  {a greatsword, partial chain armour,%
  \footnote{You keep finding chunks of webbing stuck in its creases.}
  and disgust for the weak.}% Equipment

  \showConcept{Dexterity}
  {Cretin}% Concept
  {You worked well enough, even with the most delicate materials, but lacked the vision to get ahead in life.
  Now the \gls{guard} can do your planning for you\ldots
  }% Description
  {Jester}% Code
  {Combat~1, Empathy~1, Crafts~2, Vigilance~1, Wyldcrafting~2}% Stats
  {a short sword, and a list of three things you don't understand about the world.}% Equipment


  \showConcept{Speed}
  {Barbarian}% Concept
  {You strike first and ask questions later, even when cooking soup.
  There's a special home for people who run about without thinking about anything.
  }% Description
  {Conqueror}% Code
  {Combat~2, Athletics~2, and the Knack \laststand}% Stats
  {a longsword, partial chain armour with an arrow-hole through the heart, and a reason you don't need to listen to people who think they're clever.}% Equipment


  \showConcept{Wits}
  {Faker}% Concept
  {You say you `get it', and you just want to `get on with it', but you never really understand a single task given to you.
  We're all tired of trying to explain simple ideas over your incessant babbling.
  }% Description
  {Tribe}% Code
  {Projectiles~1, Caving~1, Deceit~1, Performance~1, Medicine~1, Seafaring~1, Vigilance~2}% Stats
  {nothing -- packing is stressful, you can find what you need along the way.}% Equipment

  \showConcept{Charisma}
  {Brother}% Concept
  {Everyone loved your stories the other night, and we're glad you could join us in the \gls{guard}.
  Now you can tell stories with us on watch.
  It's going to be a riot!
  }% Description
  {Tribe}% Code
  {Combat~1, Projectiles~1, Athletics~1, Medicine~2, Wyldcrafting~1, and the Knack \guardian}% Stats
  {a short sword, 80 \glspl{cp}, and a joke for the builder.}% Equipment

\end{itemize}

\needspace{10em}
\subsubsection{Wits Failure}
\begin{itemize}

  \showConcept{Strength}
  {Dullard}% Concept
  {%
    Last to get the joke, and couldn't spot the rain in a thunderstorm.
    You're a dime a dozen.
    Replaceable.
    And it's time you replaced the last one who died.
  }% Description
  {Paladin}% Code
  {Combat~1, Projectiles~1, Athletics~1, Crafts~1, Stealth~1, Vigilance~1, Wyldcrafting 1, and the Knack \lucky}% Stats
  {a maul, partial chain armour (flakes from a giant egg make the tunic itch), and \arabic{r12}0 \glspl{cp}.}% Equipment

  \showConcept{Dexterity}
  {Airhead}% Concept
  {You play about with cards, dance with friends, and can't pay attention to anything else.
  You don't listen, or look.
  You're lost when not playing.
  We're tired of being ignored, but there's one place you'll listen\ldots
  }% Description
  {Noble}% Code
  {Projectiles~2, Caving~2, and the Knack \unstoppable}% Stats
  {any item which you could plausibly use as a toy to fidget with.}% Equipment

  \showConcept{Speed}
  {Headstrong}% Concept
  {Always rushing, never thinking.
  Always moving, never looking.
  Go run into the forest, and don't look where you're going too long.
  }% Description
  {Chronicler}% Code
  {Combat~1, Athletics~3, Deceit~1, and the Knack \charge}% Stats
  {a quarterstaff and a sob story about the time you should have looked ahead.}% Equipment

  \showConcept{Intelligence}
  {Babbler}% Concept
  {It's always tangents within tangents with no point.
  It's always ideas without observation.
  Off you go to put those ideas into practice, and we hope you can start looking around, or the \gls{guard} will feed you to the forest.
  }% Description
  {Chronicler}% Code
  {Combat~1, Academics~2, Crafts~1, Performance~1, Tactics~2}% Stats
  {a short sword and a book on gnomish cooking.}% Equipment

  \showConcept{Charisma}
  {Feckless}% Concept
  {Friendly, without principles.
  Eager, without a plan.
  Good intentions alone won't give you a life, so you can go make a new one in the \gls{guard}.
  }% Description
  {Wanderer}% Code
  {Combat~1, Athletics~1, Empathy~2, Performance~2, and the Knack \fasthealer}% Stats
  {a shiny longsword and a shaggy dog story.}% Equipment

\end{itemize}

\needspace{10em}
\subsubsection{Charisma Failure:}

\begin{itemize}

  \showConcept{Strength}
  {Reprobate}% Concept
  {Nobody likes you.
  Nobody wants your thoughts.
  But we know somewhere that can make use of your muscles\ldots
  }% Description
  {Jester}% Code
  {Combat~2, Athletics~2, Performance~1}% Stats
  {a maul, partial chain (large feathers are still stuck in the rings), and \arabic{r12}0 \glspl{cp}.}% Equipment

  \showConcept{Dexterity}
  {Rogue}% Concept
  {You argue and complain, then resort to theft.
  Who needs that?
  }% Description
  {Noble}% Code
  {Brawl~1, Athletics~1, Larceny~3, and the Knack \lucky}% Stats
  {a lockpick set, a shortsword, and a knock-knock joke.}% Equipment

  \showConcept{Speed}
  {Chicken-chaser}% Concept
  {Listen, the fox ate all the chickens, so we don't need you any more.
  But don't worry, someone still needs you\ldots
  }% Description
  {Tribe}% Code
  {Brawl~1, Athletics~2, Stealth~2, Wyldcrafting~1, and the Knack \lucky}% Stats
  {a shortsword, and a stolen chicken.}% Equipment

  \showConcept{Intelligence}
  {Pitiable}% Concept
  {Smart enough to see problems, but too inarticulate to have anyone else know.
  }% Description
  {Chronicler}% Code
  {Academics~2, Medicine~2, Tactics~2, Vigilance~1}% Stats
  {a short sword, 3 \glspl{sp}, and a conspiracy theory.}% Equipment

  \showConcept{Wits}
  {Crier}% Concept
  {Nobody appreciates your little `spin', on the news.
  Nobody appreciates your gossip about the warden's wife when we need to hear about events of the day.
  We have the fish-wives for that, and for you we have the \gls{guard}.
  }% Description
  {Jester}% Code
  {Projectiles~1, Academics~1, Empathy~1, Deceit~2, \chosenEnemy{outlaws}}% Stats
  {a shortsword, 50 \glspl{cp} and a disgusting rumour.}% Equipment

\end{itemize}

\commonArmourChart

\commonWeaponsChart %from config/rules/charts.tex
\label{commonWeapons}
\index{Weapons Chart}

\end{multicols}

