\chapter[Welcome Inn]{Character Creation}
\index{Traits}
\index{Character Creation}
\label{character_rolls}

\mapPic[\Large]{b}{Roch_Hercka/five_races}{
  {\Huge\Gn}/08/99,
  2-3/08/83,
  {+1 Dexterity}/02/24,
  {+1 Intelligence}/02/16,
  {-2 Strength}/02/08,
  {-1 Speed}/02/0,
  {\Huge\Dw}/25/99,
  4-5/25/86,
  {+1 Dexterity}/25/36,
  {-1 Speed}/25/26,
  {\Huge\Hu}/43/99,
  6-8/41/91,
  {+1 Strength}/49/34,
  {-1 Wits}/49/27,
  {\Huge\El}/64/99,
  9-10/57/92,
  {+1 Wits}/65/36,
  {-1 Strength}/65/26,
  {\Huge\Nl}/77/99,
  11-12/84/92,
  {+1 Strength}/82/24,
  {+1 Speed}/82/16,
  {-1 Intelligence}/82/0,
  {-2 Charisma}/82/08,
}

\section{Unlikely Characters}
\label{raceRoll}

\begin{multicols}{2}

\noindent
Grab $2D6$.
It's time you build some character.
Specifically, you will build your \gls{pc}.

\attributeChart

\noindent
Roll $2D6$ six times -- once for each Attribute -- and write down your result on the character sheet.
If you roll `3, 8, 9', then you have `Strength -2, Dexterity +0, Speed +1'.
Roll the dice thrice more for the Mental Attributes.

Next, roll $2D6$ to determine your character's race.

Whichever race you've landed on, go and have a look at \autoref{races} to find out about them, then come back here, and adjust your Attributes with the racial bonuses and penalties.

\input{config/rules/attributes.tex}

\subsection{Enlistment}

\label{enlistment}

Why has this character joined the \gls{guard}?
Check your lowest Attribute to see their failure, then look under your highest Attribute.
If multiple Attributes compete for lowest or highest, stop at the first.

Write down the concepts, stats and equipment.

\paragraph{Strength Failure}

\begin{itemize}

  \item
  \textbf{Dexterity:}
  Thief!
  You stole from your family, then stole from the neighbours, and finally took to snatching from slow, fat, and rich people.
  But stealing from rich people is a crime, so now you must repay your debt.

  Your pay will be docked -- receive only half the normal payment until your commander decides you have learnt your lesson.

  \textbf{Stats:}
  Combat +1, Projectiles +1, Crafts +1, Larceny +2, Stealth +1, and the knack Specialist (locks).

  \textbf{Equipment:}
  a lock-picking set, three daggers, and 4 \glspl{sp}.

  \item
  \textbf{Speed:}
  Coward!
  You might at least have shouted and thrown a rock at the monster, but you just hid and cried quietly.
  Now the forest has eaten our best animals.
  Off you go to the \gls{guard}, to learn courage.

  \textbf{Stats:}
  Write down Projectiles 1, Athletics 2, Empathy 1, Vigilance 1, Wyldcrafting 1, the knack \textit{Last Stand} and raise any Attribute from -1 to 0 (or purchase the Caving Skill).

  \textbf{Equipment:}
  Begin play with four throwing-daggers, a buckler shield, three travelling rations, and an idea of which \gls{pc} you want to hide behind when the situation goes South.
  \item
  \textbf{Intelligence:}
  Layabout!
  People like you can't just talk and recite poems all day.
  Nobody cares that you can read the stars if you won't help with the work.
  You think you're too good to work?
  Time to put that brain to use\ldots

  %!
  \textbf{Stats:}
  Write down Academics 2, Performance 1, Vigilance 1, and the Rituals Knack.
  Then fill in Earth 1, Fire 1.

  \textbf{Equipment:}
  Begin play with a piece of chalk, a tinder-box, a mirror, 50' of rope, writing equipment, and four letters you wrote (still undelivered).
  \item
  \textbf{Wits:}
  Gossiper!
  While everyone else created, toiled, and cared for their animals, you set neighbour against neighbour with your incessant chatter.
  Talking time is \emph{over}.

  Begin play with a rapier, a dagger, a tinder-box, a torch, and a rumour you overheard about another \gls{pc}.
  \item
  \textbf{Charisma:}
  Swindler!
  Those people trusted you with their money.
  They believed your ointments would cure gout, and the secret prayers would banish the next storm.
  Now reality knocks at the door, and it won't listen to your clever stories.

  \textbf{Stats:}
  Write down Athletics 1, Deceit 2, Empathy 1, Performance 2, Seafaring 1, and Vigilance 1.
  \textbf{Equipment:}
  Begin play with a dagger, a mirror, a pouch of unidentified seeds, and a flute.

\end{itemize}

\paragraph{Dexterity Failure}

\begin{itemize}

  \item
  \textbf{Strength:}
  Oaf!
  300 \glsentrylongpl{gp} down the drain, because you couldn't pick up a vase properly.
  But there's a place they welcome heavy-handed people\ldots

  Begin play with a maul, complete leather armour, three days of rations, 50 \glspl{cp}, and a knot-puzzle you cannot solve.

  \item
  \textbf{Speed:}
  Klutz!
  You zip about and run into people.
  You climb houses, and knock off the roof.
  One too many misadventures and it was time to let you fall into the clutches of the forest.

  Begin play with a short sword, partial leather armour, and 20 \glspl{cp}.
  \item
  \textbf{Intelligence:}
  Loner!
  Too smart to speak with the villagers, too useless to work for a guild.
  Time to find a place in the most accepting organization in all the land\ldots

  Begin play with a bag of flour, a bag of chalk, a shortsword, a dagger, and an unopened letter from home.

  \item
  \textbf{Wits:}
  Traitor!
  When guards came knocking, you blamed your crimes on the others in your hovel, until the others figured out your game.
  They want you gone for good, so in the end the \gls{guard} took you anyway.

  Begin play with a rapier, partial leather armour, a hidden vial of poison (deals 4 Fatigue per interval for 4 intervals) and an illegal map of a wealthy village master's keep in the exact hand-writing of any literate \gls{pc} (name them, secretly).
  \item
  \textbf{Charisma:}
  Critic!
  You could tear anyone's work apart at the guild with a few comments.
  With friends to jeer, you could bring anyone's work down, and you had plenty of friends.
  But when time came to make your own works, they never came out quite right.
  Nobody likes a critic\ldots
  
  Begin play with a broken sword, once-complete (now partial) leather armour, 3 \glspl{sp}, and some helpful advice for the player to your left.

\end{itemize}

\end{multicols}
\exampleRandomCharacter % in charts.tex
\begin{multicols}{2}

\paragraph{Speed Failure}

\begin{itemize}

  \item
  \textbf{Strength:}
  Fatty!
  You just eat and eat, helping nobody too far from the table.
  Those limited connections won't get you too far, so it's time to earn all that food.

  Begin play with six rations, a great sword, partial chain, a frying pan, and a craving for something in particular\ldots.

  \item
  \textbf{Dexterity:}
  Brigand!
  Younger siblings never inherit, and none of the guilds would have you, so you took to the road before anyone could shunt you into the \gls{guard}.
  The only way to survive was to band together with others, stealing what you could from good folks houses at night, and running away.

  The \gls{guard} chased your group down and put them to the sword while you fled.
  But you couldn't run fast enough.
  So now you must join the guard.

  Begin play with a javelin, partial leather armour, 80 \glspl{cp}, and nasty black eye.
  \item
  \textbf{Intelligence:}
  Upstart!
  You write letters, and get everyone else to send them.
  You arrange new deals, but can't deliver the goods.
  Well nobody needs a king without a kingdom, and nobody's going to do your work for you.

  Begin play with writing equipment, three torches, and a dagger.
  \item
  \textbf{Wits:}
  Jester!
  You just sit there telling jokes and mocking people.
  You spot the problems, then wait for others to solve them.
  It's time you paired that sharp wit with a sharp blade.

  Begin play with a short sword, a dagger, and 2 \glspl{sp}.
  \item
  \textbf{Charisma:}
  Cynic!
  You see issues, and moan, moan, moan.
  We can't stand to see your ugly, down-turned face around here any longer.
  Go fix all the problems you see!

  Begin play with a longsword, complete leather armour, and an attitude problem.

\end{itemize}

\paragraph{Intelligence Failure}

\begin{itemize}

  \item
  \textbf{Strength:}
  Thug!
  Actions have consequences, \emph{plural}, but you could never count past `one'.
  Luckily, we have a place where someone will put that muscle to good use.

  Begin play with a greatsword, partial chain armour, and disgust for the weak.
  \item
  \textbf{Dexterity:}
  Cretin!
  You worked well enough, even with the most delicate materials, but lacked the vision to get ahead in life.
  Now the \gls{guard} can do your planning for you\ldots

  Begin play with a short sword, and a list of three obvious things you don't understand about the world.

  \item
  \textbf{Speed:}
  Barbarian!
  You strike first and ask questions later, even when cooking soup.
  There's a special home for people who run about without thinking about anything.

  Begin play with a longsword, partial chain armour, and a reason you don't need to listen to people who think they're clever.

  \item
  \textbf{Wits:}
  Faker!
  You say you `get it', and you just want to `get on with it', but you never really understand a single task given to you.
  We're all tired of trying to explain simple ideas over your incessant babbling.

  Begin play with nothing -- packing is stressful, you can find what you need along the way.
  \item
  \textbf{Charisma:}
  Brother!
  Everyone loved your stories the other night, and we're glad you could join us in the \gls{guard}.
  Now you can tell stories with us on watch.
  It's going to be a riot!

  Begin play with a short sword, 80 \glspl{cp}, and a joke for the captain.

\end{itemize}

\paragraph{Wits Failure}
\begin{itemize}

  \item
  \textbf{Strength:}
  Dullard!
  Last to get the joke, and couldn't spot the rain in a thunderstorm.
  You're a dime a dozen.
  Replaceable.
  And it's time you replaced the last one who died.

  Begin play with a longsword, partial chain armour, and 10 \glspl{cp}.
  \item
  \textbf{Dexterity:}
  Airhead!
  You play about with cards, dance with friends, and can't pay attention to anything else.
  You don't listen, or look.
  You're lost when not playing.
  We're tired of being ignored, but there's one place you'll listen\ldots

  Begin play with any item which could plausibly be used as a toy to fidget with.
  \item
  \textbf{Speed:}
  Headstrong!
  Always rushing, never thinking.
  Always moving, never looking.
  Go run into the forest, and don't look where you're going too long.

  Begin play with a rapier and a sob story of the time you should have looked ahead.
  \item
  \textbf{Intelligence:}
  Babbler!
  It's always tangents within tangents with no point.
  It's always ideas without observation.
  Off you go to put those ideas into practice, and we hope you can start looking around, or the \gls{guard} will feed you to the forest.

  Begin play with a short sword and a book on gnomish cooking.
  \item
  \textbf{Charisma:}
  Feckless!
  Friendly, without principles.
  Eager, without a plan.
  Good intentions alone won't give you a life, so you can go make a new one in the \gls{guard}.

  Begin play with a shiny longsword and a shaggy dog story.
\end{itemize}

\paragraph{Charisma Failure:}

\begin{itemize}

  \item
  \textbf{Strength:}
  Reprobate!
  Nobody likes you.
  Nobody wants your thoughts.
  But we know somewhere that can make use of your muscles\ldots

  Begin play with a greatsword, partial chain, and 50 \glspl{cp}.
  \item
  \textbf{Dexterity:}
  Rogue!
  You argue and complain, then resort to theft.

  Begin play with a lockpick set, a shortsword, and a knock-knock joke.
  \item
  \textbf{Speed:}
  Chicken-chaser!
  Listen, the fox ate all the chickens, so we don't need you any more.
  But don't worry, someone still needs you\ldots

  Begin play with a shortsword, and a stolen chicken.
  \item
  \textbf{Intelligence:}
  Pitiable!
  Smart enough to see problems, but too inarticulate to have anyone else know.

  Begin play with a short sword, 3 \glspl{sp}, and a conspiracy theory.
  \item
  \textbf{Wits:}
  Crier!
  Nobody appreciates your little `spin', on the news.
  Nobody appreciates your gossip about the town master's wife when we need to hear about events of the day.
  We have the fish-wives for that, and for you we have the \gls{guard}.

  Begin play with a shortsword, 50 \glspl{cp} and a disgusting rumour.

\end{itemize}

\subsection{Names}

Roll a die twice -- once for the prefix, and once for the suffix.

\subsubsection[Dwarven Names]{\Dw}
Dwarven names show to their position in society, and generally relate to their mother's occupation.

\begin{nametable}[l|XYY]{Dwarven Names}

\textbf{Roll} & \textbf{Prefix} & \textbf{Suffix (M)} & \textbf{Suffix (F)} \\\hline
1 & Th   & --alin   & --oshell    \\
2 & D    & --urg    & --ragsi     \\
3 & M    & --eel    & --well      \\
4 & G    & --imlen  & --itten     \\
5 & B    & --enlak  & --lot       \\
6 & K    & --rindal & --lasi      \\

\end{nametable}

\subsubsection[Elven Names]{\El}
Elven names represent long stretches of their lives -- generally as long as the language survives.

Roll once for the prefix, and again for a female or male suffix.

The `\"e' symbol means the sound must be pronounced fully, as in `f\textbf{ai}rie', or `sel\textbf{e}ct'.
The stress goes on the last syllable of the prefix.

\begin{nametable}[l|lYY]{Elven Names}
  & \textbf{Prefix} & \textbf{Suffix (F)}   & \textbf{Suffix (M)} \\\hline
1 & Sind    & --\"e    & --on      \\
2 & Atar    & --ink\"e & --inkon   \\
3 & Ciry    & --inw\"e & --iel     \\
4 & Tarin   & --\'ote  & --or      \\
5 & Fin     & --uin    & --acil    \\
\ifodd\value{r3}
6 & It\'ar    & --w\"e   & --il      \\
\else
6 & Itar    & --il     & --ill\"e  \\
\fi
\end{nametable}

\begin{multicols}{2}

\subsubsection[Gnomish Names]{\Gn}

Roll $1D6 \times 1D6$, and re-roll on doubles to add another part to the name.

Gnomes receive names from everyone around them, so their name depends on context.
Reflexively, this therefore means they give others a name rather than asking.
However, if the gnome feels generous, and does not want to trouble any of the `big folk' with the task of creating two or three syllables, they may provide a name.

\begin{nametable}[l|Y]{Gnomish Names}
1  & ni    \\
2  & lawa  \\
3  & noka  \\
4  & en    \\
5  & ante  \\
6  & alasa \\
8  & yan   \\
9  & mu    \\
10 & kala  \\
12 & yelo  \\
15 & musi  \\
16 & ma    \\
18 & leta  \\
20 & nanpa \\
24 & mute  \\
25 & wan   \\
30 & open  \\
36 & tu    \\

\end{nametable}

\end{multicols}

\begin{multicols}{2}
\subsubsection[Gnollish Names]{\Nl}
Gnoll names are short, to-the-point, and never require difficult lip-movements (assuming you have canine lips).
The meanings generally relate to the gnoll's primary joy, such as `hunting', or `biscuits'.

\begin{nametable}[c|lY]{Gnollish Names}
\textbf{Roll} & \textbf{Prefix} & \textbf{Suffix} \\\hline
1  & Ksha & --dz  \\
2  & Ko   & --g   \\
3  & Sya  & --h   \\
4  & Tso  & --d   \\
5  & Yo   & --sh  \\
6  & Riye & --tse \\
\end{nametable}

\end{multicols}

\subsubsection[Human Names]{\Hu}
Human names remain static throughout their lives, so they never have any relation to the person, or their accomplishments.

\begin{nametable}[c|YY]{Human Names}
\textbf{Roll} & \textbf{Prefix} & \textbf{Suffix} \\\hline
\ifodd\value{r4}
1 & Lex    & --ograf \\
\else
1 & Gla    & --dun   \\
\fi
2 & Steer  & --kuff  \\
\ifodd\value{r3}
3 & Choir  & --nail  \\
4 & Flick  & --bor   \\
\else
3 & Pros   & --flay  \\
4 & Cart   & --pike  \\
\fi
\ifodd\value{page}
5 & Gors  & --meen  \\
\else
5 & Keel   & --frak  \\
\fi
6 & Moc    & --drag  \\
\end{nametable}


\end{multicols}
