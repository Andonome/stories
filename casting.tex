\chapter{Spell Weaving}

Magic does not exactly create.
It warps, waxes, and wanes.
It gives information.
But nothing in the world comes from nothing, so spellcasters have to work with the environment, but they can only change what their spheres govern.

Casters create a spell's foundation by deciding what \emph{action} to perform on which element.
Fire casters may \textit{Wax Fire}, causing it to roar and explode.
Water casters can \textit{Wane the River}, making it evaporate.
\textit{Witness Fate} spells allow the caster to know if fate has plans for someone nearby.
So each of the five spheres allows the caster to create four basic spells -- wax, wane, warp, and witness.

More advanced casters can add enhancements, though they come with complications and costs.
The `duplicated' enhancement lets spells affect a wider area, or more targets, but the caster may inadvertently target allies.
`Distant' spells can affect anyone far away, but not someone close by.
Each enhancement added increases the spell's potency, at the potential cost of unwanted complications.

On top of the basic five low spheres, casters can employ high spheres by combining two of the low spheres together.
Air and Fate create Death magic, while Water and Earth create Life magic.

\section{The Rules}

\begin{multicols}{2}

\noindent
Spell casters speak to the elements by rolling Charisma plus some Sphere (Earth, Air, Mind, et c.).
The \gls{tn} is always 7, adjusted for the target's resistance.
The resistance from a rock, or the wind depends on the state of the target (making wind more windy is easy, but summoning a hurricane underground is not).
The resistance to targeting a person depends on the person; they may resist in any number of ways.

People can resist Mind spells by focussing their mind, or resist necromancy by tensing their muscles.
And all spells might be resisted by simply stabbing the spellcaster.

\vspace{1em}
\begin{exampletext}
  An archer draws back his bow, intent on hitting the witch. At the same time, she speaks a curse, hoping to make him fumble, and miss his shot, while depleting his \glspl{fp}.
  The \gls{tn} is 7 plus the archer's \roll{Dexterity}{Projectiles}.
\end{exampletext}

\begin{exampletext}
  A water-mage attempts to Warp Water, freezing the water surrounding everyone in the flood. Bandits slosh through the flooded city, attempting to stab at him. The \gls{tn} is 7 plus the bandits' \roll{Dexterity}{Combat}.
\end{exampletext}

\begin{exampletext}
  The enchanter wants to control the elf's mind. He can't fight well, but his mind is strong. Instead of attacking the caster, he guards his mind, and resists with \roll{Wits}{Vigilance}.
\end{exampletext}

\subsubsection{Casting Range}
More powerful spells fizzle after shorter distances.
A basic spell has a range of 20 steps, and every level increase decreases the range by 5 steps.

\sidebox[3]{
  \begin{boxtable}[YL]
    \textbf{Level} & \textbf{Range} \\
    \hline
    1 & 15 steps \\
    2 & 10 steps \\
    3 & 5 steps \\
    4 & touch \\
  \end{boxtable}
}

A level 4 spell demands the mage touch the first target.
The other targets must also be touching each other, allowing the spell to flow through them like lightning.

\subsubsection{Magical Ingredients}

Various special items in the world can enhance a caster's abilities.
Almost all of them require a lot of preparation to turn into a fine dust which can permeate the air around the spell.
Anyone can throw the dust in the air to aid spell-casting.

Griffin feathers, ground to dust, grant a bonus to Air magic.
Dragon eggs, once split open, can enhance any caster's Fate Skill.

Casters, or non-casters, do not need a minimum level in any sphere to use these enhancements.
Someone with no magical abilities can use these items.
However, the use of items can only grant plus \emph{one} to any sphere, never more.

For more on these ingredients, see \autopageref{magicIngredients}.

\subsection{Enhancements}

Enhancements bolster some aspect of a spell, while adding to a spell's level.
Duplicated spell affects more targets, depending on the spell's level, and Detailed spells add subtlety, which also increase the spell's level.

Each Enhancement added increases the effects of the other dramatically, but can also bring the burden of an unwanted side effect.
Someone with the ability to cast a fourth-level spell may want to create a heavy blast of wind to throw a troublemaker at the tavern back through the door he came through, but they \emph{must} gain three enhancements to raise the spell to the fourth level. 
If they make the spell Duplicated, everyone in the tavern will feel the blast; if they make it Divergent, they must select some other element, which will create yet another spell; if they make it Distant then the wind's force will begin far away, at the horizon and do nothing to anyone next to the caster.
Most casters make a spell Detailed when looking for a safe Enhancement to add, but even then, unexpected problems often arise.

At their peak, all spheres of magic display the same problems: the spell will target only the farthest distances, will affect many targets (even if the caster doesn't want them to), will always display some peculiar appearance (which usually identifies the caster), and will create some other spell with the same action (`Wax', `Wane', et c.), but new targets.
Casters who want to give someone magical wings cannot make wings -- they will end up bestowing wings on someone while they stand upon a distant mountain-peak, and all the earth in the area becomes as brittle as glass.

\subsubsection{Detailed}

Detailed spells let the caster select how the spell appears.
Instead of making a bonfire simply explode, the caster can make it explode while looking like a mythical bird, or ice can melt into water while taking on the form of humans writing in pain.
The caster can also make spells more particular about targets, so that fire spell attacks only enemies, or a Mind spell might make someone forget what they wanted to do about that one things\ldots{} what was it again?

\subsubsection{Divergent}

Casters can add secondary effects to spells by adding another sphere.
The second spell's effects are identical, except for the additional sphere.

For example:

\begin{itemize}
  \item
  A `Wax Earth' spell hardens the bog around an enemy's feet.
  By adding `Air', the spell gains the same effects as `Wax Air', and the complete spell becomes `Divergent Wax Air and Earth'.
  \item
  A doula hopes to break a cheeky customer's mind with a `Detailed Wane Mind` spell, which makes them forget about what they came into the shop for.
  With a crooked smile, she decides to make it also `divergent', with the `Life' sphere.
  In total, the spell will cost 3 \glspl{mp}, and will send the customer out with no memory of how they lost the ability to move their toes.
\end{itemize}

\subsubsection{Distant}

Standard spells reach out only a few steps, but \textit{distant} spells can reach across worlds.
They also give spells a \emph{minimum} distance; casting a spell at `throwing distance' means the spell must fly beyond the standard maximum of a spell's distance (15 steps).
Similarly, spells with a range of `shouting distance' \emph{must} find a target beyond throwing distance.

\sidebox[3]{
  \begin{boxtable}[Yl]
    \textbf{Level} & \textbf{Range} \\
    \hline
    2 & Throwing distance \\
    3 & Shouting distance \\
    4 & Horizon \\
    5 & Line of Sight \\
  \end{boxtable}
}

If the caster misjudges the range while weaving a spell, the spell will find some other target inside the proper range.

Spells targeting the \gls{ainumar} result in the caster's instant death from a responding spell.
Nobody knows why, but they assume a god or gods are responsible.

\subsubsection{Duplicated}

Most spells have a single target, but by making a spell \textit{duplicated}, the spell goes up to level 2, and affects double the usual targets.
Earth spells might affect double the usual \gls{weight}, Mind spells affect two minds instead of one, and duplicated Fire spells can target a multitude of nearby torches, candles, and anything else currently alight.
If the caster can employ another Enhancement, then the spell multiplies \emph{again} by 3, and then by 4, and so on.

\sidebox[2]{
  \begin{boxtable}[YY]
    \textbf{Level} & \textbf{Targets} \\
    \hline
    2 & 2 \\
    3 & 6 \\
    4 & 24 \\
    5 & 120 \\
  \end{boxtable}
}

With enough Enhancements pushing a spell's level up, spells can affect huge numbers of targets.
The incredible force of a caster with a strong stock of magical items has serious implications for warfare across \gls{fenestra}.
While armies can fall prey to a single enchanter's song, a small, strong-willed group of veterans can push through and succeed.

\end{multicols}

\section{Spheres of Magic}
\index{Spheres of Magic}

\begin{multicols}{2}

\subsection{Elemental Spheres}
\index{Elemental Spheres}

\Gls{fenestra} has no need for philosophers to debate the fundamentals of the world.
Spellcasters can alter five basic building-blocks -- Air, Earth, Fate, Fire, and Water -- so they have first-hand knowledge of the world's essential nature.

\subsubsection{Modes}
\index{Modes}
Each sphere a character gains gives them access to four modes of casting -- four ways to affect the element.

\begin{description}
  \item[Waxing]
  spells encourage the element, making it \emph{more} like what it is.
  \item[Waning]
  magic does the opposite -- it reduces the element's nature, and often destroys the target.
  \item[Warping]
  an element alters some fundamental aspect, promoting strange behaviour and effects from the target.
  \item[Witnessing]
  means to find out whether or not the element exists.
  These spells let the caster know if the element lies nearby.
\end{description}

\subsubsection{Air}
\hint{wind, smoke, steam, fog, mist, cloud}

\begin{description}
  \item[Wax]
  spells increase the air's motion, making even a still room fast enough to push someone back.
  \item[Wane]
  spells make air putrid enough to make people choke and retch.
  Breathing the smoke in inflicts 1 \gls{fatigue} per level of the spell.
  \item[Warp]
  spells can reshape air into a more solid `bubble'.
  \item[Witness]
  air spells don't come up very often, but the spell remains, nevertheless.
\end{description}

\subsubsection{Earth}
\hint{dirt, ice, snow, sand, ash}

\begin{description}
  \item[Wax]
  spells harden sand into stone, or bind snow into ice.
  \item[Wane]
  softens earth, melts ice, turns dirt to mud, and calms magma.
  \item[Warp]
  spells twist earth in unnatural ways, making it brittle.
  These spells can increase the \gls{tn} to damage a wall or door, but if they receive a single point of Damage, the brittle mass may shatter.
\end{description}

\subsubsection{Fate}
\hint{luck, curses, prophecy}

\begin{description}
  \item[Wax]
  fate spells grant 2 \glspl{fp} plus the spell's level.
  The \glspl{fp} convert to dice, just like Damage, so a level 1 \textit{Wax Fate} spell would grant $1D6-1$ \glspl{fp}.
  \item[Wane]
  spells inflict curses, removing the same number of \glspl{fp}.
  \item[Warp]
  fate spells inflict \emph{encounters} on a target.
  Each level adds one encounter per interval.%
  \iftoggle{judgement}{%
    \footnote{\Glspl{gm} looking for encounter ideas can find a system in \textit{Judgement}: \nameref{encounters}, \autopageref{encounters}.}%
  }{}
  \item[Witness]
  spells tell a caster if someone has \glspl{fp}.
  Witches once informed people when they had found `the chosen one', but so many fated for great things end up eaten by the forest that nowadays nobody puts much stock in someone with an aura of luck.
\end{description}

\subsubsection{Fire}
\hint{flame, lightning, magma, furnaces}

\begin{description}
  \item[Wax]
  spells make fires flare up and burn through their fuel in an explosive instance.
  \item[Wane]
  spells put fires out.
  Miracle workers who truly understand fire can often do more damage through darkness than damage.
  \item[Warp]
  spells can remove a fire's light, make them flicker downwards rather than up, or alter the colours.
\end{description}

\subsubsection{Water}
\hint{rivers, rain, ale, magma}

\begin{description}
  \item[Wax]
  spells make rivers flow faster, rain fall harder and purify any liquid, from ales to straight-up poisons.
  \item[Wane]
  encourages water's natural evaporation.
  Small-scale spells might encourage a cup of water to evaporate, while greater magics could turn a river into steam.
  \item[Warp]
  spells freeze liquids, or encourage any mildly impure substance to become putrid, or even acidic.
\end{description}

\subsection{High Spheres}

High spheres combine two of the lower, elemental spheres to produce something new.
If someone can cast both of the lower spheres they need, then they can also cast from that high sphere at a level equal to the lowest of the constituents.

\begin{exampletext}
  For example, a miracle worker with Water 2, and Fate 1 can automatically cast spells from the Mind sphere at level 1.
  If they gain the Earth sphere at level 2, they could then combine Water and Earth magic to make Life magic, also at level 2.
\end{exampletext}

\subsubsection{Death}
\hint{rot, health, souls, fatigue, regeration}

\textbf{Elements:}
\roll{Air}{Fate}

\begin{description}
  \item[Wax]
  spells increase any deathly effects, inflicting \glspl{fatigue} equal to the spell's level.
    \begin{description}
      \item[Detailed]
        versions can remove \glspl{hp} instead of adding \glspl{fatigue}.
    \end{description}
  \item[Wane]
    spells remove degradation.
    The target ignores a number penalties from \glspl{fatigue} equal to the spell's level.

    These spells always cause problems in the long-run.
    Each interval the target engages in any physical activity whatsoever, the spell weakens a little, and after a number of \glspl{interval} equal to the spell's level, it ends, crushing the target with \gls{fatigue} penalties.
    \begin{description}
      \item[Detailed]
        spells focus this slowing of \glspl{fatigue} onto a single problem, so the target ignores \gls{fatigue} penalties gained through hiking, poison, fighting, or any other source, equal to double the spell's level.
        For example, a level 3 spell would allow the target to ignore up to a -6 \gls{fatigue} penalty.
      \item[Objects]
        targeted by this spell rot more slowly.
        \footnote{Some use the spell to preserve food, though it always loses its flavour.
        This has resulted in the nickname `death rations' for any food affected by the spell.}
      \item[Undead corpses,]
        animated by angry spirits, cannot move properly in their dead shell.
        This spell allows a corpse to become supple enough to move, although the body continues to rot whenever the creature moves.%
        \footnote{For more on the duration of death magic, see \autopageref{deathDuration}.}
        This leaves mindless undead falling apart after they spend long enough searching for souls, but sentient undead always feel aware of this slow degradation and guard their ability to sleep viciously.
        Many spend centuries standing perfectly still, fearing any situation which might make them move, and bring them closer to the afterlife which still awaits them.
    \end{description}
  \item[Warp]
    spells pull pull people into a kind of limbo between life and death.
    Their heart slows, but they keep moving just as fast.
    People so affected bleed less, and gain an unnatural \gls{dr} of 2, just like the undead.

    The target also becomes detached from the world around them, and begin to treat the world like a kind of strategy game.
    The uncaring mindset, glazed eyes, and strange conversation reduces their Charisma bonus by a number equal to the spell's level.
    \begin{description}
      \item[Detailed]
        spells allow \ldots a redesign. What do they do?
    \end{description}
  \item[Witness]
    spells track dead spirits who somehow cling to this realm, evading their proper afterlife.%
    \exRef{judgement}{Judgement}{godsOfDeath}
    These spells allow mages to target spirits with other spells, but only a moment before the spirit moves.
    \begin{description}
      \item[Detailed]
        Witness spells allow the caster to receive more specific information, such as the location of a relative who passed away, and then (if that spell succeeds) to ask if that spirit wants to express something about a particular subject.

        As usual, only `yes' or `no' answers return.
        The caster cannot simply ask a spirit to divulge information, because they cannot hear the spirit -- they can only confirm or disconfirm guesses about what the spirit might want to say.
    \end{description}
\end{description}

\subsubsection{Force}
\hint{momentum, portals, teleportation}

\textbf{Elements:}
\roll{Fire}{Earth}

\begin{description}
  \item[Wax]
  force means a heavy force, often used in combat.
  These spells let anything deal Damage equal the spell's $level + 3$, but the spell must be cast at \emph{exactly} the same time as the target weapon hits something.
  In mechanical terms, the caster must release the spell when they have exactly the same number of \glspl{ap} as the attack they want to aid.

  \begin{exampletext}
    An alchemist has Fire 2 and Earth 1, which implies access to the Force sphere at level 1.
    He wants to aid his companion's arrows, with a \textit{Wax Force} spell.
    Unfortunately they already deal $1D6+3$ Damage, 
  \end{exampletext}
  \item[Wane]
  force means to remove momentum or gravity.
  The target may lose a number of \glspl{ap} equal to the spell's level, or gain a bonus to jumping equal to to the spell's level +2.%
  \footnote{The target's \gls{weight} also reduces by 1 per spell level.}
  While people break free from the spell's effects through their own motions, inanimate objects do not.
  A sword stripped of its own weight will simply float around indefinitely.
  Likewise, someone who becomes near weightless can simply relax, and float along with the wind indefinitely.
  Unfortunately for them, they have little say in which direction they will float, so flight is not on the cards without some very clever Air magic prepared beforehand.
  \item[Warp]
  forces crate tears in space\ldots or possibly just one tear in space.
  This spell creates a magical `rift' which joins one place to another, allowing anyone to step through one side, and step out the other.
  This spell may initially appear to have two targets (the entrance, and exit), but since both places become one, the spell has only one target -- one `rift'.
  These rifts have two symmetric sides, so a portal placed over a kitchen's door might lead to the garden outside, and the other side of that door would lead to the same place in the garden, but out the other side of the rift.

  These portals have delicate edges, so any damage to them will destroy them.
  Creatures moving through them must make a \roll{Dexterity}{Larceny} check, with a \gls{tn} equal to double their \gls{weight} rating, minus the spell's level.
  So in total, a gnome with a \gls{weight} of 4 would have a \gls{tn} of 6 to move through a portal, while human with Strength +1 would be at \tn[12].
  Building any kind of wall around the portal to protect the border ensures the maximum \gls{weight} which can move through is equal to the spell's level.

  As usual, a \textit{duplicated} spell allows for a larger \gls{weight} in total (as it effectively increases the spell's level).
  Spells of level 4 or more will allow creatures with up to a \gls{weight} of 24, which should allow just about anything through without any bother.
  \item[Witness]
  of force allows the caster to know where any heavy forces act (flying arrows, heavy falling objects, or magical rifts).
\end{description}

\subsection{Spell Duration \& Banishment}

All spells last forever, or perhaps they vanish instantly but leave the world changed.

\begin{description}
  \item[Air]
  dissipates eventually, although moving air remains in motion, and choking fogs remain noxious.
  \item[Earth]
  which breaks remains broken, and ice statues melt in the Sun.
  \item[Fate]
  runs its course.
  \item[Fire]
  can burn forever, so some rich houses keep an enchanted fire going, burning in all of the colours of the rainbow.
  \item[Water]
  always returns to its natural form in time, once mist settles, or acid mixes with normal water, and joins the natural cycles of the sea and rivers.
  \item[Death]
  \begin{itemize}
    \item
    spells which simply debilitate heal normally.
    \item
    Warping effects reduce by one as the target embraces life, and starts to push themselves.
    At the end of any \gls{interval} in which the target endures \pgls{fatigue} penalty, the spell disappears.
    \item
    Waning effects, which reduce the rate of death, dissipate with movement.
    Each \gls{interval} in which the target engages in exercise, they reduce any effects by a single point.
    \label{deathDuration}
  \end{itemize}
  \item[Force]
  \begin{itemize}
    \item
    effects cast on people dissipate as the person moves.
    Every \gls{ap} spent reduces the effects by 1.
    \item
    Warping effects (portals) vanish once the sides are touched, as the rift in space collapses.
  \end{itemize}
  \item[Light]
  spells suffer from any type of push or movement -- even a strong wind will eventually crack an illusion.
  \item[Life]
  \begin{itemize}
    \item
    Life enhancing magic remains fuelled by the body of the affected (or `afflicted').
    These spells end only through starvation or tiredness, once the \gls{fatigue} penalty reaches -4.
    \item
    Warping spells work the same, although the impatient will often simply cut the wings off.
    \item
    Life spells which `Wane' will leave the afflicted until they heal.
    People will heal at the same rate as they heal \glspl{hp}, and only heal once all \glspl{hp} have healed.
  \end{itemize}
  \item[Mind]
  altering effects depart as quickly as a habit, meaning that they might linger for some time.
  Whatever the effects, when someone gains \pgls{xp} for following their code, they can shake off the spell.%
  \footnote{Spending money does not count.}
\end{description}

\end{multicols}

