\chapter{Casting Rules}

Magic does not exactly create.
It warps, waxes, and wanes.
It gives information.
But nothing in the world comes from nothing, so spellcasters have to work with the environment, but they can only change what their spheres govern.

Casters create a spell's foundation by deciding what \emph{action} to perform on which element.
Fire casters may \textit{Wax Fire}, causing it to roar and explode.
Water casters can \textit{Wane the River}, making it evaporate.
\textit{Witness Fate} spells allow the caster to know if fate has plans for someone nearby.
So each of the five spheres allows the caster to create four basic spells - wax, wane, warp, and witness.

More advanced casters can add enhancements, though they come with complications and costs.
The `duplicated' enhancement lets spells affect a wider area, or more targets, but the caster may inadvertently target allies.
`Distant' spells can affect anyone far away, but not someone close by.
Each enhancement added increases the spell's potency, at the potential cost of unwanted complications.

On top of the basic five low spheres, casters can employ high spheres by combining two of the low spheres together.
Air and Fate create Death magic, while Water and Earth create Life magic.

\section{Weaving}

\section{Spheres of Magic}
\index{Spheres of Magic}

\begin{multicols}{2}

\subsection{Casting Spells}

\begin{itemize}
  \item
  Casters roll Charisma + Sphere at \tn[7].

  \textit{For example, a caster with Charisma +1 and Air +2, casting a first level Air spell would roll with a +3 Bonus.}
  \begin{itemize}
    \item
    Various things can increase the \gls{tn}.
    Living creatures make a resisted action as usual, and inanimate things resist in their own ways.
    \item
    The Snap Caster Knack allows casters to use their Wits Attribute instead of Charisma.
    Similarly, the Ritual Caster Knack allows a caster to use Intelligence.%
    \exRef{core}{Core Rules}{ritualCaster}
  \end{itemize}
  \item
  Casters can gain bonuses to their low-spheres from rare ingredients, when properly prepared.
  These bonuses count as if the caster had that higher level.
  \item
  Special situations can also grant a bonus to high-spheres (see below).
\end{itemize}

\subsection{Elemental Spheres}
\index{Elemental Spheres}

\Gls{fenestra} has no need for philosophers to debate the fundamentals of the world.
Spellcasters can alter five basic building-blocks -- Air, Earth, Fate, Fire, and Water -- so they have first-hand knowledge of the world's essential nature.

\subsubsection{Modes}
\index{Modes}
Each sphere a character gains gives them access to four modes of casting -- four ways to affect the element.

\begin{description}
  \item[Waxing]
  spells encourage the element, making it \emph{more} like what it is.
  \item[Waning]
  magic does the opposite -- it reduces the element's nature, and often destroys the target.
  \item[Warping]
  an element alters some fundamental aspect, promoting strange behaviour and effects from the target.
  \item[Witnessing]
  means to find out whether or not the element exists.
  These spells let the caster know if the element lies nearby.
\end{description}

\subsubsection{Air}
\hint{wind, smoke, steam, fog, mist, cloud}

\begin{description}
  \item[Wax]
  spells increase the air's motion, making even a still room fast enough to push someone back.
  \item[Wane]
  spells make air putrid enough to make people choke and retch.
  Breathing the smoke in inflicts 1 \gls{fatigue} per level of the spell.
  \item[Warp]
  spells can reshape air into a more solid `bubble'.
  \item[Witness]
  air spells don't come up very often, but the spell remains, nevertheless.
\end{description}

\subsubsection{Earth}
\hint{dirt, ice, snow, sand, ash}

\begin{description}
  \item[Wax]
  spells harden sand into stone, or bind snow into ice.
  \item[Wane]
  softens earth, melts ice, turns dirt to mud, and calms magma.
  \item[Warp]
  spells twist earth in unnatural ways, making it brittle.
  These spells can increase the \gls{tn} to damage a wall or door, but if they receive a single point of Damage, the brittle mass may shatter.
\end{description}

\subsubsection{Fate}
\hint{luck, curses, prophecy}

\begin{description}
  \item[Wax]
  fate spells grant 2 \glspl{fp} plus the spell's level.
  The \glspl{fp} convert to dice, just like Damage, so a level 1 \textit{Wax Fate} spell would grant $1D6-1$ \glspl{fp}.
  \item[Wane]
  spells inflict curses, removing the same number of \glspl{fp}.
  \item[Warp]
  fate spells inflict \emph{encounters} on a target.
  Each level adds one encounter per interval.%
  \iftoggle{judgement}{%
    \footnote{\Glspl{gm} looking for encounter ideas can find a system in \textit{Judgement}: \nameref{encounters}, \autopageref{encounters}.}%
  }{}
  \item[Witness]
  spells tell a caster if someone has \glspl{fp}.
  Witches once informed people when they had found `the chosen one', but so many fated for great things end up eaten by the forest that nowadays nobody puts much stock in someone with an aura of luck.
\end{description}

\subsubsection{Fire}
\hint{flame, lightning, magma, furnaces}

\begin{description}
  \item[Wax]
  spells make fires flare up and burn through their fuel in an explosive instance.
  \item[Wane]
  spells put fires out.
  Miracle workers who truly understand fire can often do more damage through darkness than damage.
  \item[Warp]
  spells can remove a fire's light, make them flicker downwards rather than up, or alter the colours.
\end{description}

\subsubsection{Water}
\hint{rivers, rain, ale, magma}

\begin{description}
  \item[Wax]
  spells make rivers flow faster, rain fall harder and purify any liquid, from ales to straight-up poisons.
  \item[Wane]
  encourages water's natural evaporation.
  Small-scale spells might encourage a cup of water to evaporate, while greater magics could turn a river into steam.
  \item[Warp]
  spells freeze liquids, or encourage any mildly impure substance to become putrid, or even acidic.
\end{description}

\subsection{High Spheres}

High spheres combine two of the lower, elemental spheres to produce something new.
If someone can cast both of the lower spheres they need, then they can also cast from that high sphere at a level equal to the lowest of the constituents.

\begin{exampletext}
  For example, a miracle worker with Water 2, and Fate 1 can automatically cast spells from the Mind sphere at level 1.
  If they gain the Earth sphere at level 2, they could then combine Water and Earth magic to make Life magic, also at level 2.
\end{exampletext}

\subsubsection{Force}
\hint{momentum, portals, teleportation}


\textbf{Elements:}
\roll{Fire}{Earth}

\begin{description}
  \item[Wax]
  force means a heavy force, often used in combat.
  These spells let anything deal Damage equal the spell's $level + 3$, but the spell must be cast at \emph{exactly} the same time as the target weapon hits something.
  In mechanical terms, the caster must release the spell when they have exactly the same number of \glspl{ap} as the attack they want to aid.

  \begin{exampletext}
    An alchemist has Fire 2 and Earth 1, which implies access to the Force sphere at level 1.
    He wants to aid his companion's arrows, with a \textit{Wax Force} spell.
    Unfortunately they already deal $1D6+3$ Damage, 
  \end{exampletext}
  \item[Wane]
  \item[Warp]
  \item[Witness]
\end{description}
\end{multicols}

\section{Building spells}

\begin{multicols}{2}

\begin{enumerate}
  \item
  Every spell starts by selecting one or more spheres.
  \item
  The caster then selects one of the four \textit{modes} to apply to the element:
  \begin{description}
    \item[Detailed]
    spells change appearance, or grant more information.

    \textit{Crumbling stone with a `detailed wane earth' spell may create a statue on the inside of the shattered rock, revealed by brushing off the debris.
    To `Witness Fate` could highlight those with 8 or more \glspl{fp}.}
    \item[Distant]
    spells outstretch the usual range, \emph{but} cannot be limited in their range.
    \item[Divergent]
    spells combine an extra sphere (low or high). The two spheres have the same   Enhancements, but can have separate targets.

    \textit{A spell to `wax water' can make a river bulge, so adding fire would make a spell to `wax water and fire'; as the river swells, a bonfire flares up.}
    \item[Duplicated]
    spells can affect many targets -- typically the number of targets equals the cost, squared.
  \end{description}
\end{enumerate}

\subsection{Spell Duration \& Banishment}

All spells last forever, or perhaps they vanish instantly but leave the world changed.

\begin{description}
  \item[Air]
  dissipates eventually, although moving air remains in motion, and choking fogs remain noxious.
  \item[Earth]
  which breaks remains broken, and ice statues melt in the Sun.
  \item[Fate]
  runs its course.
  \item[Fire]
  can burn forever, so some rich houses keep an enchanted fire going, burning in all of the colours of the rainbow.
  \item[Water]
  always returns to its natural form in time, once mist settles, or acid mixes with normal water, and joins the natural cycles of the sea and rivers.
  \item[Death]
  \begin{itemize}
    \item
    spells which simply debilitate heal normally.
    \item
    Warping effects reduce by one as the target embraces life, and starts to push themselves.
    At the end of any \gls{interval} in which the target endures \pgls{fatigue} penalty, the spell disappears.
    \item
    Waning effects, which reduce the rate of death, dissipate with movement.
    Each \gls{interval} in which the target engages in exercise, they reduce any effects by a single point.
  \end{itemize}
  \item[Force]
  \begin{itemize}
    \item
    effects cast on people dissipate as the person moves.
    Every \gls{ap} spent reduces the effects by 1.
    \item
    Warping effects (portals) vanish once the sides are touched, as the rift in space collapses.
  \end{itemize}
  \item[Light]
  spells suffer from any type of push or movement -- even a strong wind will eventually crack an illusion.
  \item[Life]
  \begin{itemize}
    \item
    Life enhancing magic remains fuelled by the body of the affected (or `afflicted').
    These spells end only through starvation or tiredness, once the \gls{fatigue} penalty reaches -4.
    \item
    Warping spells work the same, although the impatient will often simply cut the wings off.
    \item
    Life spells which `Wane' will leave the afflicted until they heal.
    People will heal at the same rate as they heal \glspl{hp}, and only heal once all \glspl{hp} have healed.
  \end{itemize}
  \item[Mind]
  altering effects depart as quickly as a habit, meaning that they might linger for some time.
  Whatever the effects, when someone gains \pgls{xp} for following their code, they can shake off the spell.%
  \footnote{Spending money does not count.}
\end{description}

\end{multicols}

