\chapter[The Soldier's Code]{Codes \& Experience}
\label{codes}

\section{Codes of Belief}
\index{Codes of Belief}
\label{listOfCodes}

\begin{multicols}{2}
\noindent
Each character has a Code of belief.
Fulfilling the Code grants \glspl{xp} to the player to spend on the character.

Each of the four tasks grants 1~\gls{xp} once fulfilled, but players can only claim them once per session.

Characters can also gain \glspl{xp} for spending their \glspl{sp} on what they value.
Most gain either 1, 2, or 3 \glspl{xp}, depending on how much they spend.

\subsection{Chronicler}
\label{chronicler}
\index{Chronicler (Code)}

Chroniclers want information -- to have knowledge that others don't, and to uncover all they can about the world.

\begin{itemize}
  \item
  Providing a recap at the start of the session.%
  \footnote{Any number can take this prize, if they all help with the recap.}
  \item
  Uncovering a secret.
  \item
  Solving a puzzle.
  \item
  Preserving knowledge that would otherwise have been lost.
\end{itemize}

Chroniclers spend their money on maps, bribery, encryption equipment, and education.

\begin{enumerate}
  \item
  Spending 1 \gls{sp} on information.
  \item
  Spending 5 \glspl{sp} on information.
  \item
  Spending as many \glspl{sp} on information as the character has \glspl{xp}.
\end{enumerate}

\null

\subsection{Conqueror}
\label{conqueror}
\index{Conqueror (Code)}

We go beyond the \gls{edge}, not to drink or talk, but for the unending war with the forest.
We leave to kill and subdue every creature, and to conquer our enemies who live beyond the walls.

The conqueror hears the call to adventure, and focusses on the mission.

\begin{itemize}

  \item
  Helping a fellow player improve their combat tactics.
  \item
  Losing \glspl{hp}.
  \item
  Going first into a dangerous situation.
  \item
  Slaying the most powerful beast you have slain so far (measured by \gls{cr}).

\end{itemize}

Conquerors can see value in their money when they manage to spend it on better weapons which see real use.
They might buy a new sword, or arrows, poisons, or just take their armour to receive some quality care with a blacksmith.

\begin{enumerate}
  \item
  Buying 1 \gls{sp}'s worth of weapons.
  \item
  Buying 5 \glspl{sp}'s worth of weapons.
  \item
  Buying as many \glspl{sp}'s worth of weapons as the character has \glspl{xp}.
\end{enumerate}

\subsection{Jester}
\label{jester}
\index{Jester (Code)}

Life should be lived.
Some people don't know that, but you can help them see the joy in life, whether they want to or not.

\begin{itemize}
  \item
  Executing a prank set up on a previous session.
  \item
  Making a painful pun.
  \item
  Lifting the spirits of the downtrodden.
  \item
  Finding a new type of food or drink.
\end{itemize}

Jesters live to eat, drink, and make others merry.
Whether having a drinking competition, or feeding the poor, they like seeing perishables go unperished.

\begin{enumerate}
  \item
  Spending 1 \gls{sp} on food and drink.
  \item
  Spending 5 \glspl{sp} on food and drink.
  \item
  Spending as many \glspl{sp} on food and drink as the character has \glspl{xp}.
\end{enumerate}

\null
\subsection{Noble}
\label{noble}
\index{Noble (Code)}

A job well done deserves a reward.
A good worker deserves the best.
And what's the point in all this wandering if you can't return to a hot bath, the best food, and some god-damned appreciation?

\begin{itemize}

  \item
  Helping a stumped \gls{gm} come up with a quick name for a tavern, \gls{village}, or \gls{npc}.
  \item
  Being addressed deferentially.
  \item
  Disarming a potential conflict before it escalates.
  \item
  Acquiring more wealth than ever before.

\end{itemize}

Nobles like to spend their money on lavish, quality equipment -- something which does the job better than the rest -- the best-tasting wine, the most enduring bow, and the sharpest daggers (possibly with jewels in the hilt).
Lavish items include sweet food, jewel-encrusted weapons, silk robes, and top-quality wine.

\begin{enumerate}
  \item
  Spending 1 \gls{sp}'s on lavish goods.
  (The goods must cost at least double their standard values)
  \item
  Using 5 \glspl{sp}'s on unnecessary pampering.
  \item
  Using as many \glspl{sp}'s worth of disgusting indulgences as the character has \glspl{xp}.
\end{enumerate}

\subsection{Paladin}
\label{paladin}
\index{Paladin (Code)}

The best of the \gls{guard} exist to uphold the law and make sure the local populace can trust that civilization always wins in the end.

\begin{itemize}
  \item
  Declaring an action with an in-character voice.
  \begin{itemize}
    \item
    Instead of `I go to the tavern, if the \gls{village} has one', try `\textit{My throat feels parched, and is that a tavern I see?}'.
    \item
    Instead of saying `I loot the bodies. Do I find anything?', just declare `\textit{I wonder if their pockets have made this little deviation worth the trouble and effort}'.
    \item
    Instead of `I hit the goblin with my axe', say `\textit{let us see how he likes the taste of axe!}'.
  \end{itemize}
  \item
  Punishing a law-breaker.
  \item
  Fulfilling an oath set up on a previous \gls{interval}.
  (characters can make unlimited oaths without penalty)
  \item
  Capturing a more troublesome law-breaker than ever before (measured by \gls{cr}).
\end{itemize}

Paladins donate their additional coins to anyone in need.

\begin{enumerate}
  \item
  Donating 1 \gls{sp}.
  \item
  Donating 5 \glspl{sp}.
  \item
  Donating as many \glspl{sp} as the character has \glspl{xp}.
\end{enumerate}


\subsection{Tribalist}
\label{tribalist}
\index{Tribalist (Code)}

The tribalist supports the group, honours the dead, and leaves no one behind.
Exactly who counts as `in the tribe' depends upon the player's interpretation, but it will generally include any \glspl{pc} who have been involved in three or more sessions with them, and anyone the player decides to spend \glspl{storypoint} on (see \autopageref{stories}).

\begin{itemize}
  \item
  Bringing snacks for the table.
  \item
  Helping a member of the tribe.
  \item
  Helping \pgls{pc} gain \glspl{xp}.
  \item
  Honouring the memory of the fallen.
\end{itemize}

Tribalists spend money to help their own tribe in any way they can.

\begin{enumerate}
  \item
  Spending 1 \gls{sp} on the tribe.
  \item
  Spending 5 \glspl{sp} on the tribe.
  \item
  Spending as many \glspl{sp} on the tribe as the character has \glspl{xp}.
\end{enumerate}

\subsection{Wanderer}
\label{wanderer}
\index{Wanderer (Code)}

Some just want to see the world, and everything it has to offer.

\begin{itemize}

  \item
  Mapping a new area. (the player must draw the map).
  \item
  Remembering the name of \pgls{npc} from a previous \gls{interval}.
  \item
  Travelling somewhere new.
  \item
  Seeing a new type of creature.

\end{itemize}

Wanderers have little focus on money.
They tend to spend what they have and move on, instead of remaining shackled to their items.

\begin{enumerate}
  \setcounter{enumi}{1}
  \item
  Spending all of their coinage on anything which seems fun.
\end{enumerate}


\end{multicols}

\section{\Glsfmtlongpl{xp}}
\label{xpCosts}

\begin{multicols}{2}

\sidepic{Roch_Hercka/xp-1}

\noindent
\Glsentrylongpl{xp} let characters grow over time, and reward players for sticking to their Code.
Low-level Traits cost very little, so players can buy a lot of Skills, or remove Attribute penalties easily.
Higher levels become increasingly expensive, so having a specialized character will take some patience.

\begin{itemize}
  \item
  Increasing a negative Attribute costs 5~\glspl{xp}.
  After that, the price increases sharply.
  \item
  Combat Skills also increase sharply, so learning a little of everything comes easier than mastering a single Skill.
  \item
  Standard Skills cost half as much as Attributes, so characters can pick up a few before long.
  \item
  The first Knack is cheap, at 5~\glspl{xp}, but each Knack costs 5 more \glspl{xp} than the last.
\end{itemize}

All characters begin with 50 \textit{spent} \glspl{xp}
(self-made characters get to spend theirs, standard characters don't).
These spent \glspl{xp} make a difference as each character's total \glspl{fp} equals their total \glspl{xp} divided by ten.
Players should always take care to record how many \glspl{xp} they have earned in total.

\subsubsection{Progression}

Players can earn a maximum of 7 \glspl{xp} per mission from following their Codes, but will most often earn less.
Purchasing a +3 Attribute Bonus takes many sessions to build up the \glspl{xp}.

\sidepic{Roch_Hercka/xp-2}

\label{racial_limits}
\index{Attribute Limitations}
\Glspl{pc} cannot raise any Attribute above 3 (plus the racial modifiers).
So gnolls could raise their maximum Speed Bonus to +4, but gnomes cannot raise their Speed Bonus above +2, due to their natural -1 penalty.
No matter how much a gnome practices, they will never outrun another biped twice their size.

Once a player spends all 5~\glspl{storypoint}, they gain 25~\glspl{xp}, so any \gls{pc} can quickly shoot up to 75~spent \glspl{xp} in total, giving them 8~\glspl{fp} (plus the Charisma Bonus).

Most \glspl{pc} will die after a few sessions, but if one survives, it might go like this.

\newcommand\describeFullSeason[1]{%
  \addtocounter{season}{#1}%
  \ifnum\value{season}>12%
    \setcounter{season}{1}%
  \fi%
  \setTemperature\showTemperature\ \showSeason%
}

\begin{boxtable}[YLcc]
  \textbf{Real Time} & \textbf{\Glsfmttext{fenestra} Time} & \textbf{\Glsfmttext{xp}} & \textbf{\Glsfmttext{fp}} \\
  \hline
          0          & \describeFullSeason{0}             &    50               &   5  \\
     1 month         & \describeFullSeason{1}             &    87               &   8  \\
     2 months        & \describeFullSeason{1}             &    99               &   9  \\
     3 months        & \describeFullSeason{1}             &   111               &   11 \\
     \ldots          & \ldots                             &   \ldots            &   \ldots \\
     6 months        & \describeFullSeason{3}             &   147               &   15 \\
     \ldots          & \ldots                             &   \ldots            &   \ldots \\
     9 months        & \describeFullSeason{6}             &   183               &   18 \\
\end{boxtable}

The chart assumes the player spends all \glspl{storypoint} within the first four sessions, and then gains an average of 3~\glspl{xp} per session.

\end{multicols}

\XPchart
